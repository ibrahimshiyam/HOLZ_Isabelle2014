\section{Example: Cash Dispenser}

As an example, we look at the fragments of (simplified) dialogues
between a user and a cash dispenser.  The
following basic constructors for fragments will be used:
\begin{itemize}
\item $actor \ucmk action$ constructs a fragment containing the single
  interaction where $actor$ performs $action$.
\item $actor \ucmk action \uctrans rel$ is the general version of
  constructing singleton fragments. In addition to $actor \ucmk action$, a
  transformation on the system state is given by the relation $rel$.  We
  have $actor \ucmk action = actor \ucmk action \uctrans \id$.
\item $frag \ucthen frag'$ is the sequential composition of fragments.
  After the dialogues described by $frag$ the ones of $frag'$ must
  follow.  
\item $\ucselect frags$ describes a choice between several fragments.
  $frags$ can be an enumeration of fragments, but also a
  set-comprehension: we use the pattern $\ucselect\{x:A @ frag\}$
  to introduce a locally bound variable $x$ in fragments, which
  semantically is the choice between all possible instantiations
  of $x$.
\item $\ucrepeat frag$ describes the repetition of $frag$ for 
  zero or more times.
\item $frag \ucexcept frag'$ describes that $frag$ can be ``interrupted''
  at some interaction which overlaps with the first interaction of $frag'$.
  It is then continued with the behavior of $frag'$.
\end{itemize}

For the cash dispenser, the type $ACTOR$ defines the actors, $user$
and $dispenser$. The type $ACTION$ lists the actions performed: 
\begin{zedgroup}
\begin{zdirectives}
  \zsection[UseCases]{CashDispenser} 
\end{zdirectives} \\
\begin{zed}
  ACTOR ::= user | dispenser \\
  ACTION ::= \<askCard | putCard | ejectCard | takeCard | \\
             askAmount | putAmount \ldata \num \rdata | \\
             ejectMoney \ldata \num \rdata | takeMoney\\\>
\end{zed}
\end{zedgroup}

The type ${State}$ defines the system state, which is the money
reserve of the dispenser (in an extended version, we might represent
here the accounts of individual card holders).  We define two
operations $Draw$ and $CantDraw$ to be used in the fragments which
work on the system state. With $\mkrel$ operations on ${State}$ are
lifted to relations.

From a methodological point of view, we have to justify that the
visible representation of $reserves$ is not a violation of our
principle that inner system state must \emph{not} be modelled in Use
Cases. We argue that the reserves \emph{are} observable, since the
user may be confronted with the situation that the dispenser cannot
satisfy his requests because of low reserves:

\begin{zedgroup}
\begin{schema}{State}
  reserves: \num
\end{schema}
\begin{schema}{CantDraw}
  \Xi State; amount? : \num
\where
  reserves < amount? 
\end{schema}
\begin{schema}{Draw}
  \Delta State; amount? : \num
\where
  reserves \geq amount?;
  reserves' = reserves - amount?
\end{schema}
\begin{axdef}
  \mkrel == (\lambda Op:\power(\Delta State) @ 
                         \{Op @ (\theta State, \theta State ')\})
\end{axdef}
\end{zedgroup}

The fragments are given as follows. $Normal$ describes the usual
process of a disposition.  $InvalidCard$ is an exceptional fragment
which can interrupt $Normal$ at the point where the user has inserted
his card; the card is immediately rejected and the dialogue ends if
the user has removed his card.  $NoReserves$ is a further exceptional
behavior; it can continue at the point where the user enters the
amount to draw, \emph{and} where this amount cannot be served because
of low reserves. Note the use of the choice, $\ucselect$, to bind the
input variable $amount?$ in $Normal$ and $NoReserves$.  The overall
behavior of the system is described by a repetition of the $Normal$
fragment which can be interrupted by $InvalidCard$ or by $NoReserves$:


\begin{axdef}
  Normal,InvalidCard,NoReserves, System: \\\t1
                                 Fragment[ACTOR,ACTION,State]
\where
  \zblock{
  Normal & = & dispenser \ucmk askCard \\
        & \ucthen & user \ucmk putCard \\
        & \ucthen & dispenser \ucmk askAmount \\
        & \ucthen & \ucselect\{amount?: \num @ \\
        &         & user \ucmkt \<putAmount(amount?) \uctrans 
                           \mkrel [\Delta State | Draw] \\\> 
        & \ucthen & dispenser \ucmk ejectCard \\
        & \ucthen & user \ucmk takeCard \\
        & \ucthen & dispenser \ucmk ejectMoney(amount?) \\
        & \ucthen & user \ucmk takeMoney \}\\
  InvalidCard & = &  user \ucmk putCard \\
              & \ucthen & dispenser \ucmk ejectCard \\
              & \ucthen & user \ucmk takeCard \\
  NoReserves & =        & \ucselect\{amount?: \num @ \\
             &   & user \ucmkt \<putAmount(amount?) \uctrans 
                              \mkrel [\Delta State | CantDraw] \\\> 
             & \ucthen  & dispenser \ucmk ejectCard \\
             & \ucthen  & user \ucmk takeCard\} \\
  System & = &\ucrepeat(Normal ~ ~ \<\ucexcept ~ InvalidCard 
                                 \ucexcept ~ NoReserves)\>
}
\end{axdef}

We can now test satisfaction of given dialogues regarding the cash
dispenser's uses cases. Let some test
dialogues be defined as follows:

\begin{zedgroup}
\begin{zed}
  d_1 == \langle \< 
            dispenser \ucmki askCard,
            user \ucmki putCard, \\
            dispenser \ucmki askAmount,
            user \ucmki putAmount(400), \\
            dispenser \ucmki ejectCard,
            user \ucmki takeCard, \\
            dispenser \ucmki ejectMoney(400),
            user \ucmki takeMoney \rangle \\ \>
  d_2 == \langle \<
            dispenser \ucmki askCard,
            user \ucmki putCard, 
            dispenser \ucmki ejectCard,
            user \ucmki takeCard  \rangle \\ \> 
  \sigma_1 == \lbind reserves==600 \rbind \\
  \sigma_2 == \lbind reserves==800 \rbind \\
\end{zed}
\end{zedgroup}

\noindent
Here are some query results:

\begin{zexecexpr}
  (d_1,\sigma_1) \ucin System
\yields
  *true*
\end{zexecexpr}

\begin{zexecexpr}
  (d_1 \cat d_2,\sigma_1) \ucin System
\yields
  *true*
\end{zexecexpr}

\begin{zexecexpr}
  (d_1 \cat d_2 \cat d_1,\sigma_1) \ucin System
\yields
  *false*
\end{zexecexpr}

\begin{zexecexpr}
  (d_1 \cat d_2 \cat d_1,\sigma_2) \ucin System
\yields
  *true*
\end{zexecexpr}

In the last case, the reserves are too low to serve two subsequent
requests of the amount of $400$ units. In the fourth case, the
reserves are raised, such that the requests can be satisfied.

In the next evaluation, we demonstrate the efficiency of
evaluation. We construct a dialogue of length $1200$
and check for conformance: 

\begin{zexecexpr}
  (\dcat ((\lambda n:\nat @ d_1 \cat d_2) \circ \id(1 \upto 100)),
   \lbind reserves == 400 * 100 \rbind
  ) \ucin System
\yields
  *true*
  (7.86 seconds, 910827 steps, 115881 steps/sec)
\end{zexecexpr}

This numbers are measured on a PII400 running Linux.

%%% Local Variables: 
%%% mode: latex
%%% TeX-master: "ucases"
%%% End: 
