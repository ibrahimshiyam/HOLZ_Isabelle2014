\section{Model of Use Cases}

We define the formal model of our version of use cases in Z. An
\emph{interaction} is given by the schema $Interaction$, generic over
the type of actors $\alpha$ and of actions $\pi$.  A dialogue is
a sequence of interactions:

\begin{zedgroup}
\begin{zdirectives}
  \zsection{UseCasesModel}   
\end{zdirectives} \\
\begin{schema}{Interaction[\alpha,\pi]}
  actor: \alpha; action: \pi
\end{schema}
\begin{zed}
  Dialogue[\alpha,\pi] == \seq Interaction[\alpha,\pi] 
\end{zed}
\end{zedgroup}

A \emph{pattern} is a sequence of interactions paired with a state
transition relation over the state type $\Sigma$. A \emph{fragment}
is a set of patterns:
\begin{zedgroup}
\begin{zed}
  Pattern[\alpha,\pi,\Sigma] == 
    \seq (Interaction[\alpha,\pi] \cross (\Sigma \rel \Sigma)) \\
  Fragment[\alpha,\pi,\Sigma] == 
    \power Pattern[\alpha,\pi,\Sigma]
\end{zed}
\end{zedgroup}

Our constructor functions for fragments are defined as follows:

\begin{zedgroup}
\begin{zdirectives}
  \zfunction{65 (\_ \ucmki \_)} \t3 \\
  \zfunction{60 (\_ \ucmk \_)} \\
  \zfunction{60 (\_ \ucmkt \_ \uctrans \_)} \t2 \\
               %% CHECKME: ESZ MIXFIX BUG (\ucmk should be usable in both)
  \zfunction{50 \rightassoc (\_ \ucthen \_)} \\
  \zfunction{40 \leftassoc (\_ \ucexcept \_)}
\end{zdirectives}
\begin{axdef}[\alpha,\pi]
  \_ \ucmki \_ ==
    \lambda actor: \alpha; action: \pi @ % \\\t9
      \theta Interaction[\alpha,\pi]  \\
\end{axdef}
\begin{axdef}[\alpha,\pi,\Sigma]
  \_ \ucmkt \_ \uctrans \_ == 
    \lambda actor: \alpha; action: \pi; r: \Sigma \rel \Sigma @ \\\t9
      \{\langle (actor \ucmki action, r) \rangle\}  \\
\end{axdef}
\begin{axdef}[\alpha,\pi,\Sigma]
  \_ \ucmk \_ ==
    \lambda actor: \alpha; action: \pi @ 
         actor \ucmkt action \uctrans \id[\Sigma]
\end{axdef}
\begin{axdef}[\alpha,\pi,\Sigma]
  \_ \ucthen \_ == 
    \lambda f,f': Fragment[\alpha,\pi,\Sigma] @ \\\t9
      \{ p: f; p': f' @ p \cat p' \}
\end{axdef}
\begin{axdef}[\alpha,\pi,\Sigma]
 \ucrepeat == \lambda f: Fragment[\alpha,\pi,\Sigma] ~ @ \\\t1
    \{p_1: Pattern[\alpha,\pi,\Sigma]; p_2 : f @ (p_1, p_1 \cat p_2)\}\star
     \limg \{\langle\rangle\} \rimg  
\end{axdef}
\begin{axdef}[\alpha,\pi,\Sigma]
  \_ \ucexcept \_ == 
    \lambda f,f': Fragment[\alpha,\pi,\Sigma] @ f \cup ~ \\\t1
       \{ \< p: f; p': f'; i: \nat | 
            i \in \dom p;
            first(p~i) = first(p'~1) \\
          @ ((1 \upto i-1) \dres p) \cat p'\}\>
\end{axdef}

\begin{axdef}[\alpha,\pi,\Sigma]
  \ucselect == 
    \lambda fs: \power Fragment[\alpha,\pi,\Sigma] @ \bigcup fs
\end{axdef}
\end{zedgroup}


The satisfaction relation on fragments, $(d,\sigma) \ucin f$, holds iff
the dialogue $d$ and the initial state $\sigma$ confirm to~$f$,
which is defined as follows:

\begin{zedgroup}
\begin{zdirectives}
  \zrelation{(\_ \ucin \_)}
\end{zdirectives}
\begin{axdef}[\alpha,\pi,\Sigma]
  \_ \ucin \_ : Dialogue[\alpha,\pi] \cross \Sigma \rel 
                  Fragment[\alpha,\pi,\Sigma]
\where
  \forall \<d: Dialogue[\alpha,\pi]; \sigma: \Sigma; 
          f: Fragment[\alpha,\pi,\Sigma]@\\\>\\\t1
    (d,\sigma) \ucin f \iff \\\t2
    (\exists \<p: f @ \\ % \bigcup (glues\star \limg \{ f \} \rimg  @ \\
                     \sigma \in \dom(fold(\_\comp\_)(second \circ p)) 
                            \land 
                       first \circ p = d)\>
\end{axdef}
\end{zedgroup}

\noindent
Thus, a dialogue confirms to a fragment if there exists a pattern in
the fragment such that the initial state is in the domain of the
composition of all state transitions in the pattern and the
interactions of the pattern match the dialogue.

