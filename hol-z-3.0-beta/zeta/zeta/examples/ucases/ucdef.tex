\section{Executable Encoding of Uses Cases}

The satisfaction relation for fragments, $(d,\sigma) \ucin f$, where
$d$ is a dialogue, $\sigma$ a system state and $f$ a fragment, is not
executable in its descriptive definition from the previous section.
We could perhaps define an executable version of the constructors for
fragments and of the satisfaction relation: however, the
representation of fragments as sets of patterns is a dead-end
regarding \emph{efficient} executability.  The problem comes apparent
in the definition of $\_ \ucthen \_$ which equals to $$\lambda f,f':
Fragment[\alpha,\pi,\Sigma] @ \{ p: f; p': f' @ p \cat p' \}$$
A
common prefix $p$ is not shared in the composition and needs to be
``parsed'' again for every $p'$. A better representation would use
\emph{trees}, preserving common prefixes in fragments.  In this
section, we develop an encoding of fragments with executable
satisfaction based on this idea.

For a tree-like representation, we encode fragments as a set of
\emph{branches}. A branch is either $\uceod$ -- indicating that a
dialogue may end here -- or $\ucbr(i,r,f)$, where $i$ is the interaction
at the head of this branch, $r$ the state transition, and $f$ the
followup fragment:

\renewcommand{\assumed}{}

\begin{zedgroup}
\begin{zdirectives}
  \zsection{UseCases}    
\end{zdirectives} \\

%%pregen \cpower
%% \begin{zed} \cpower X == \assumed \power X \end{zed}

\begin{schema}{Interaction[\alpha,\pi]}
  actor: \alpha; action: \pi
\end{schema}
\begin{zed}
  Dialogue[\alpha,\pi] == \assumed \seq Interaction[\alpha,\pi] 
\end{zed}
\begin{zed}
  Branch[\alpha,\pi,\Sigma] ::= % \\\t1
    \uceod | 
    \ucbr \ldata \assumed Interaction[\alpha,\pi] \cross
                    (\Sigma \rel \Sigma) \cross
                    Fragment[\alpha,\pi,\Sigma] \rdata \\
  Fragment[\alpha,\pi,\Sigma] == 
    \assumed \cpower Branch[\alpha,\pi,\Sigma] \\
\end{zed}
\end{zedgroup}
\noindent
This definition makes use of an extension of the Z of the \Zeta{}
system, allowing generic free types. Note that the power operator used for
fragments cannot be the general powerset in order to let the
construction be consistent -- with $\cpower$ we denote the
``executable'' power-sets.

Based on the tree encoding, we redefine the operations on fragments.
The operator templates and basic constructors for fragments
nearly remain the same:
\begin{zedgroup}
\begin{zdirectives}
  \zfunction{65 (\_ \ucmki \_)} \\
  \zfunction{60 (\_ \ucmk \_)} \\
  \zfunction{60 (\_ \ucmkt \_ \uctrans \_)} \\
  \zfunction{50 \rightassoc (\_ \ucthen \_)} \\
  \zfunction{40 \leftassoc (\_ \ucexcept \_)}
\end{zdirectives}
\begin{axdef}[\alpha,\pi]
  \_ \ucmki \_ ==
    \lambda actor: \alpha; action: \pi @ % \\\t9
      \theta Interaction[\alpha,\pi]  \\
\end{axdef}
\begin{axdef}[\alpha,\pi,\Sigma]
  \_ \ucmkt \_ \uctrans \_ == 
    \lambda actor: \alpha; action: \pi; r: \Sigma \rel \Sigma @ \\\t9
      \{\ucbr (actor \ucmki action, r, \{\uceod\})\}  \\
\end{axdef}
\begin{axdef}[\alpha,\pi,\Sigma]
  \_ \ucmk \_ ==
    \lambda actor: \alpha; action: \pi @ 
         actor \ucmkt action \uctrans \id[\Sigma]
\end{axdef}
\end{zedgroup}

For the definition of sequential composition, we use a technique
which is paradigmatic for the tree encoding of fragments: the
composition is lazily ``pushed'' through the construction of the
tree:

\begin{zedgroup}
\begin{axdef}[\alpha,\pi,\Sigma]
  \_ \ucthen \_ : \<Fragment[\alpha,\pi,\Sigma] \cross
                   Fragment[\alpha,\pi,\Sigma] ~ \fun \\\t1
                   Fragment[\alpha,\pi,\Sigma] \> 
\where
 (\_ \ucthen \_) = \lambda f_1,f_2: Fragment[\alpha,\pi,\Sigma] @ \\\t1
    \<
     % \{b: Branch[\alpha,\pi,\Sigma] | \uceod \in f_1; b \in f_2\} ~ \cup \\
     (\IF \uceod \in f_1 \THEN f_2 \ELSE \emptyset) ~ \cup \\
     \{\<
       i: Interaction[\alpha,\pi]; r: \Sigma \rel \Sigma \\
       f_1': Fragment[\alpha,\pi,\Sigma] % \\
       | \ucbr(i,r,f_1') \in f_1 \\
       @ \ucbr(i,r,f_1' \ucthen f_2)
       \}\>\>
\end{axdef}
\end{zedgroup}

%\noindent
%Since our execution model of Z uses a strict evaluation
%order, we use a suspension such as $$\{b: Branch[\alpha,\pi,\Sigma] |
%\uceod \in f_1; b \in f_2\}$$ instead of the more intuitive 
%formulation $$\IF \uceod \in f_1 \THEN f_2 \ELSE \emptyset$$
%which on the level of the Z semantics represents the same meaning.

The definition of $f_1 \ucexcept f_2$ uses similar techniques:
\begin{zedgroup}
\begin{axdef}[\alpha,\pi,\Sigma]
  \_ \ucexcept \_ : \<Fragment[\alpha,\pi,\Sigma] \cross
                   Fragment[\alpha,\pi,\Sigma] ~ \fun \\\t1
                   Fragment[\alpha,\pi,\Sigma] \> 
\where
 (\_ \ucexcept \_) = \lambda f_1,f_2: Fragment[\alpha,\pi,\Sigma] @ \\\t1
    \<
     (\IF \uceod \in f_1 \THEN \{\uceod\} \ELSE \emptyset) ~ \cup \\
     \{\<
       i: Interaction[\alpha,\pi]; r_1,r_2: \Sigma \rel \Sigma \\
       f_1',f_2': Fragment[\alpha,\pi,\Sigma] \\
       | \ucbr(i,r_1,f_1') \in f_1; \ucbr(i,r_2,f_2') \in f_2 % \\
       @ \ucbr(i,r_2,f_2')
     \} ~ \cup \\\>
     \{\<
       i: Interaction[\alpha,\pi]; r_1: \Sigma \rel \Sigma \\
       f_1': Fragment[\alpha,\pi,\Sigma] \\
       | \ucbr(i,r_1,f_1') \in f_1 % \\
       @ \ucbr(i,r_1,f_1' \ucexcept f_2)
     \} \\\> \>
\end{axdef}
\end{zedgroup}
\noindent 
The third case in the set union describes the actual interruption,
where we continue with $f_2$, provided that there is an overlapping
between a current interaction of $f_1$ and the first interaction of $f_2$.

In the definition of the $\ucrepeat$ operator, again we use a suspension
to embed the recursive expansion of the operator. In order to avoid
looping of $\ucrepeat \{\uceod\}$, $\uceod$ is excluded in the repetition:

\begin{zedgroup}
\begin{axdef}[\alpha,\pi,\Sigma]
  \ucrepeat : \<Fragment[\alpha,\pi,\Sigma] 
                   ~ \fun 
                   Fragment[\alpha,\pi,\Sigma] \> 
\where
 \ucrepeat = \lambda f: Fragment[\alpha,\pi,\Sigma] @ \\\t1
    \<
    (\mu f': Fragment[\alpha,\pi,\Sigma] |
      f' = 
     \{\uceod\} ~ \cup \\
     \{b: Branch[\alpha,\pi,\Sigma] |
        b \in (f \setminus \{\uceod\}) \ucthen f'\})
     \>
\end{axdef}
\end{zedgroup}

The definition of the choice, $\ucselect$, is the same as in the
model semantics: 

\begin{zedgroup}
\begin{axdef}[\alpha,\pi,\Sigma]
  \ucselect == 
    \lambda fs: \power Fragment[\alpha,\pi,\Sigma] @ \bigcup fs
\end{axdef}
\end{zedgroup}
\noindent
Generalized union is executable by the definition
$$ \bigcup SS = \{S: \power A; ; x: A | S \in SS; x \in S @ x\}$$
as provided by the standard Z toolkit.

We finally define satisfaction: $(d,\sigma) \ucin f$
``parses'' a dialogue and initial state by trying the branches of a
fragment tree. This is a typical depth-first backtracking algorithm:

\begin{zedgroup}
\begin{zdirectives}
  \zrelation{(\_ \ucin \_)}
\end{zdirectives}
\begin{axdef}[\alpha,\pi,\Sigma]
  \_ \ucin \_ : \power(\<(Dialogue[\alpha,\pi] \cross \Sigma) \cross 
                  Fragment[\alpha,\pi,\Sigma])\>
\where
  (\_ \ucin \_) = \\\t1
  \<
  \{ \< 
     \sigma: \Sigma; f: Fragment[\alpha,\pi,\Sigma] | \uceod \in f % \\
     @ ((\langle\rangle, \sigma), f) \} ~ \cup \\\>
  \{ \< 
     d: Dialogue[\alpha,\pi]; \sigma,\sigma': \Sigma;
     f,f': Fragment[\alpha,\pi,\Sigma] \\ 
      r: \Sigma \rel \Sigma % \\
     | \<d \neq \langle\rangle; \ucbr(head~d,r,f') \in f \\ 
       (\sigma,\sigma') \in r;
       (tail~d, \FORCE \sigma') \ucin f' \\\>
     @ ((d, \sigma), f)
   \}\>\>
\end{axdef}
\end{zedgroup}


To model concurrency, we conservatively extend our current encoding by
a new operator for parallel composition.  The definition uses the same
``trick'' as before, pushing the composition lazily through tree
construction:

\begin{zedgroup}
\begin{zdirectives}
  \zfunction{45 (\_ \ucpar \_)}
\end{zdirectives} \\
\begin{axdef}[\alpha,\pi,\Sigma]
  \_ \ucpar  \_ : 
      \<Fragment[\alpha,\pi,\Sigma] \cross
                   Fragment[\alpha,\pi,\Sigma] ~ \fun \\\t1
                   Fragment[\alpha,\pi,\Sigma]\\ \> 
\where
 (\_ \ucpar \_) = \lambda f_1,f_2: Fragment[\alpha,\pi,\Sigma] @ \\\t1
   \<
%   \{b: Branch[\alpha,\pi,\Sigma] | \uceod \in f_1; b \in f_2\} ~ \cup \\
%   \{b: Branch[\alpha,\pi,\Sigma] | \uceod \in f_2; b \in f_1\} ~ \cup \\
   (\IF \uceod \in f_1 \THEN f_2 \ELSE ~ \IF \uceod \in f_2 \THEN f_1 
                                    \ELSE \emptyset)~\cup \\
   \{\<
     i: Interaction[\alpha,\pi]; 
     r_1,r_2: \Sigma \rel \Sigma \\
     f_1',f_2': Fragment[\alpha,\pi,\Sigma] \\
     | \ucbr(i,r_1,f_1') \in f_1; \ucbr(i,r_2,f_2') \in f_2 \\
     @ \ucbr(i, r_1 \cap r_2, f_1' \ucpar f_2')\} \cup ~ \\ \>
   \{\<
     i: Interaction[\alpha,\pi]; 
     r_1: \Sigma \rel \Sigma \\
     f_1': Fragment[\alpha,\pi,\Sigma] \\
     | \ucbr(i,r_1,f_1') \in f_1 % \\
     @ \ucbr(i,r_1, f_1' \ucpar f_2)\} \cup ~ \\ \>
   \{\<
     i: Interaction[\alpha,\pi]; 
     r_2: \Sigma \rel \Sigma \\
     f_2': Fragment[\alpha,\pi,\Sigma] \\
     | \ucbr(i,r_2,f_2') \in f_2 % \\
     @ \ucbr(i,r_2, f_1 \ucpar f_2')\}\> \>
\end{axdef}
\end{zedgroup}


%%% Local Variables: 
%%% mode: latex
%%% TeX-master: "ucases"
%%% End: 
