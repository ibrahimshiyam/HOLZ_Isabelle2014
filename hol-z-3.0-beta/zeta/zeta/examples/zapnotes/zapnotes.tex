\documentclass{article}

\sloppy
%% those definitions to complex for latex2html

\def\uncatcodespecials{\def\do##1{\catcode`##1=12 }\dospecials}
\def\obeytab{\catcode`\^^I=\active}
\def\setupverbatim{\tt%
    \def\par{\leavevmode\endgraf}%
    \obeylines \uncatcodespecials \obeyspaces\obeytab%
    %\everypar{\advance by1 \llap{\sevenrm\the\lineno\ \ }}
}
{\obeyspaces\global\let =\ }
\def\eightspaces{\ \ \ \ \ \ \ \ }
{\obeytab\global\let^^I=\eightspaces}

\catcode`�=\active
\def�{\bgroup\color{red}\def�{\egroup\egroup}\mbox\bgroup\setupverbatim}
%\def�{\bgroup\def�{\egroup\egroup}\mbox\bgroup\setupverbatim}

\newenvironment{mszsyntax}
    {\begin{center}\zedindent=0pt\begin{tabular}{p{3.4cm}>{$}c<{$}lm{9.0cm}}}
    {\end{tabular}\end{center}}
\newenvironment{latexsyntax}
    {\begin{center}\zedindent=0pt
     \def~{\hskip3pt}
     \def\{{\big\lbrace}
     \def\}{\big\rbrace}
     \begin{tabular}{p{3.0cm}>{$}c<{$}>{$}m{9.5cm}<{$}}}
    {\end{tabular}\end{center}}

\newcommand{\ns}[1]{\mbox{\sffamily #1}}
\newcommand{\os}[1]{\mbox{\sffamily #1}}
\newcommand{\ms}[1]{\mbox{\sffamily #1}}


 \def\[{\mbox{\kern2pt\slshape\large[\kern2pt}}
    \def\]{\mbox{\kern2pt\slshape\large]\kern1pt}}
    \def\rlist{\[ \ns{Renaming-List} \]}

    %\def\docsym#1{$#1$&{\color{red}\tt\string#1}}
    \def\emit{\uparrow}
    \def\MCOL#1{\\\hline\multicolumn6c{\bf #1} \\\hline}

    \def\sfCom#1{\expandafter\def\csname #1\endcsname{\mbox{\sf #1}}}
    \def\DataSpace{DataState}               \def\Control{Control}
    \sfCom{State}
    \sfCom{Default}                         \sfCom{History}
    \sfCom{DeepHistory}                     \sfCom{ev}


\def\dummy{}



\begin{document}
\title{Notes on \ZAP}

\author{Wolfgang Grieskamp\thanks{Technische Universit\"at Berlin,
    Fachbereich Informatik, Franklinstr. 28/29, 10587 Berlin,
    e-mail:~\texttt{wg@cs.tu-berlin.de}}}

\date{\today{} -- for version $2$ of \ZAP{} (DRAFT)}

\maketitle

\begin{abstract}
  This (incomplete and draft) document is an attempt towards a
  documentation for \ZAP, the compiler for
  Executable Z, as it is part of the \Zeta{} system. 
\end{abstract}

\tableofcontents

\section{Introduction}

\ZAP{} (version 2) realizes an execution model for Z based on concepts
described e.g.~in \cite{gri99b,gri00a}. It provides a
combination of higher-order, functional logic execution principles,
adapted to the set-based environment of Z. All idioms of Z which are
related to \emph{functional} and to \emph{logic} programming languages
are executable with \ZAP.

This document is a (first) attempt towards a documentation of \ZAP.
It cannot substitute a manual and tutorial.  For some examples and
tutorial introduction, see also the \verb:examples: directory of the
\Zeta{} distribution, and the publications \cite{gl00a,gl00b}.

\begin{remark}
  In order to replay the examples in this document, you may load its
  source into \zetalink{\Zeta{}}{zeta}. Environments named
  \verb:zexecexpr: and \verb:zexecpred: contain the execution queries.
  The source of this document is found in the examples directory of
  the \Zeta{} distribution, subdirectory \verb:zapnotes:.
\end{remark}

\section{Principles}


\subsection{Recursive Sets}

As sets are paradigmatic for the specification level of Z, they are
for the execution level. Set objects -- relations or functions -- are
eventually defined by (recursive) equations, as in the following
example, where we define natural numbers as a free type\footnote{See
  section \ref{sec:freetypes} why we use $S \ldata \{x:N\} \rdata$
  instead of $S \ldata N \rdata$.}, addition on these numbers and an
order relation:

\begin{zedgroup}
\begin{zdirectives}
  \zsection{Nat}
\end{zdirectives} \\
\begin{zed}
  N ::= Z | S \ldata \{x: N\} \rdata    
\end{zed}
\begin{axdef}
  three == S(S(S~Z))
\end{axdef}
\begin{axdef}
  add: N \cross N \fun N
\where
  add = \< 
     \{ y: N @ (Z,y) \mapsto y \} \cup % ~ ~ \cup \\
     \{ x,y,z: N | (x,y) \mapsto z \in add @ (S~x,y) \mapsto S~z \}
   \>
\end{axdef}
\begin{zdirectives}
  \zrelation{(\_ less \_)}
\end{zdirectives} \\
\begin{axdef}
  \_ ~less~ \_ == \{x,y: N | (\exists t: N @ (x,S~t) \mapsto y \in add)\}
\end{axdef}
\end{zedgroup}

We may now execute under \Zeta{} queries such as the following, 
where we ask for the numbers less then three:

\begin{zexecexpr}
  \{x:N | x ~ less ~ three\}
\yields
{Z,S(Z),S(S(Z))}
\end{zexecexpr}

These capabilities are similar to logic programming.  In
fact, we can give a translation from any clause-based system to a
system of recursive set-equations in the style given for $add$, where
we collect all clauses for the same relational symbol into a union of
set-comprehensions, and map literals $R(e_1,\ldots,e_n)$ to membership
tests $(e_1,\ldots,e_n) \in R$.

\subsection{Encapsulated Search}
\label{sec:esearch}

A binary relation $R$ can be \emph{applied}, written as $R~e$, which
is syntactic sugar for the expression $\mu t:X | (e,t) \in R$.  This
expression is defined iff their exists a unique $t$ such that the
constraint is satisfied; it then delivers this $t$. The set $add$ is a
binary relation (since it is member of the set $\power((N \cross N)
\cross N)$), and therefore we can for example evaluate

\begin{zexecexpr}
 add(three,three) 
\yields
S(S(S(S(S(S(Z))))))
\end{zexecexpr}

Note the semantic difference of $(e,y) \in R$ and $y=R~e$: the first
is not satisfied if $R$ is not defined at $e$, or produces several
solutions for $y$ if $R$ is not unique at $e$, whereas the second is
\emph{undefined} in these cases. 

This difference is resembled in the implementation: application,
$\mu$-expressions, and related forms are realized by
\emph{encapsulated search}.  When resolving the constraint $y = \mu t:X
| (e,t) \in R$ (the unsugared form of $y = R~e$), the constraint
$(e,t) \in R$ is resolved in a different search space then the outer
constraint $y = \ldots$ (which just waits for the completion of
the inner search). 

Encapsulated search is not allowed to produce bindings for variables
belonging to outer contexts. If an encapsulated search requires the
binding of a context variable, it residuates until the context
provides the binding. One reason for this behavior is that we may use
the context variables in negative positions (i.e.~$\mu t:X @ \lnot
(x,t) \in S$, where a binding for $x$ obviously is a semantic error).
Another reason is that encapsulated search gives us a tool to control
resolution order in a natural way -- it introduces \emph{functional}
reduction into our framework.

As a consequence, in our above example, $y = \mu t:X | (e,t) \in R$,
if we have $e = x$, where $x$ is a variable from an outer context
(i.e. if we have originally written $y = R~x$), the encapsulated
search residuates until $x$ is bound in the context -- if the
resolution of $(x,t) \in R$ actually requires a binding for $x$
\footnote{A situation where the value of $x$ is not required is
  f.i.~given if $R = \{x:\num@(x,x)\}$.}.

For illustration, we redefine the $less$ relation from the
previous section as follows:

\begin{zedgroup}
\begin{zdirectives}
  \zrelation{(\_ less_1 \_)}
\end{zdirectives} \\
\begin{axdef}
  \_ ~less_1~ \_ == \{x,y: N | (\exists t: N @ y = add(x,S~t))\}
\end{axdef}
\end{zedgroup}

\begin{zexecexpr}
  \{x:N | x ~ less_1 ~ three\}
\yields
 unresolvable constraints in toplevel goal {
    resolving LTX:zapnotes.tex(136.21-136.64)
    with variables:
     x = _x
     y = S(S(...))
     t~1 unbound
    unresolved constraints:
     LTX:zapnotes.tex(136.52-136.61) waiting for subgoal {
       resolving '{_y|(_x,_y) IN _S}'
       with parameters:
        _S = LTX:zapnotes.tex(70.6-70.33)+LTX:zapnotes.tex(71.6-71.67)
        _x = (_x,S(...))
       with variables:
        _y unbound
       unresolved constraints:
        '(_x,_y) IN _S' waiting for variable _x
     }
  }
{*unknown*}
\end{zexecexpr}

Here, the encapsulated search for $add(x,y)$ cannot continue, since it
is not allowed to produce bindings for the context variables $x$. In
the above diagnostics, \verb:{_y|(_x,_y) IN _S}: is the encapsulated
search goal for application, which is parameterized over \verb:_x: and
\verb:S:.  The parameter names are in a different name scope, hence
the parameter binding \verb:_x = (_x,S(...)): is \emph{not} cyclic.
The constraint residuates for the variable \verb:_x:, which is the
value assigned to \verb:x:.



\subsection{Higher-Orderness}

As functions are first-order citizens in functional languages, sets
are in \ZAP.  For example, we define a function describing relational
image as follows:

\begin{axdef}[X,Y]
   image == 
      \lambda R: \power(X \cross Y); S: \power X @ % \\\t1
        \{x: X; y: Y | x \in S; (x,y) \in R @ y \}
\end{axdef}

A query for the relational image of the $add$ function over the
cartesian product of the numbers less then three yields in:

\begin{zexecexpr}
  \LET ns == \{x: N | x~less~three \} @ image(add,ns \cross ns)
\yields
{Z,S(Z),S(S(Z)),S(S(S(Z))),S(S(S(S(Z))))}
\end{zexecexpr}

In a similar style as above most constants of the Z toolkit (domain
restriction and so on) can be naturally defined.


\subsection{Quantors}\label{sec:quantors}

Universal quantification is executable if it deals with finite ranges.
For example, we can define a generic constant denoting the set of partial
functions:

\begin{axdef}[X,Y]
  pfun ==  
    \{R: \power(X \cross Y) | % \\\t4
            (\forall x:X| x \in \dom R @ \exists_1 y:Y @ (x,y) \in R)\}
\end{axdef}

Universal and unique existential quantification are resolved by
enumeration (an instance of encapsulated search).  Thus, if we try to
check whether $add$ is a partial function, we get:

\begin{zexecpred}
  add \in pfun[N \cross N,N]
\yields
  ...
   unresolved constraints:
      LTX:zapnotes.tex(205.58-205.68) waiting for variable y
\end{zexecpred}

The reason for this behavior is that the domain of $add$ is not
finitely enumerable:

\begin{zexecpred}
  \dom add 
\yields
{(Z,y),(S(Z),z),(S(S(Z)),y),(S(S(S(Z))),y),...}
\end{zexecpred}

However, if we restrict $add$ to a finite domain it works:

\begin{zexecpred}
  \exists_1 ns == \{x: N | x~less~three \} @
  ((ns \cross ns) \dres add) \in pfun[N \cross N,N]
\yields
*true*
\end{zexecpred}

Above, $A \dres R$ restricts the domain of $R$ to the set $A$;
the existential quantor is used to introduce a local name in
the predicate.

\subsection{Schema Calculus}

In Z, a schema denotes a set of \emph{bindings}, where bindings are
tuples (records) with named components. The schema calculus operators,
then, are essentially set operations with special implicit signature
translations. 

Therefore, the schema calculus is fully executable. For instance

\begin{zexecexpr} \relax
  [x,y: \num | y \in 1 \upto 2] \land [y,z:\num | y = z]
\yields
  {<x==x,y==1,z==1>,<x==x,y==2,z==2>}   
\end{zexecexpr}

For more examples see the birthday-book specification in the
\verb:examples: directory of the \Zeta{} distribution.


\subsection{Declarations}
\label{sec:decls}

In Z specifications we often find declarations of the kind $f: A \pfun
B$, imposing that $f$ has type $\power (A \cross B)$ on the one hand,
and that $f \in A \pfun B$ on the other. As we have seen in section
\ref{sec:quantors}, the check whether a value is a partial function is
only executable for finite domains.

In order to conveniently support this specification style, \ZAP{}
treats declarations $x:E$ as \emph{assumptions}. Hence the membership
test $x \in E$ implied by $x:E$ is discarded for execution. 

If $x:E$ is actually a non-redundant property, it has to be
explicitely denoted in the constraint part of schema text.
For this reason, in the definition of $pfun$ in the
previous section, we have written $\forall x: X | x \in \dom R @ \ldots$
instead of $\forall x:\dom R @ \ldots$.



\subsection{Definition Forms}

When providing definitions of relations and functions one has to use
equational forms.  For example, the factorial function would be
typically specified in Z by axioms of the kind:

\zsection{facNonExec}

\begin{axdef}
  fac : \nat \fun \nat
\where
  fac~0 = 1; \forall x:\nat | x > 0 @ fac(x) = x*fac(x-1)
\end{axdef}

A definition of this form is unamenable to execution. Instead one
may write:

\zsection{facExec1}

\begin{axdef}
  fac : \nat \fun \nat
\where
  fac = \{(0,1)\} \cup (\lambda x:\nat | x > 0 @ x*fac(x-1))
\end{axdef}

or

\zsection{facExec2}

\begin{axdef}
  fac : \nat \fun \nat
\where
  fac = \lambda x:\nat @ \IF x = 0 \THEN 1 \ELSE x*fac(x-1)
\end{axdef}

The automatic conversion of axioms of the first form
to definitions of the later forms may become a feature of future
versions of \ZAP\footnote{\ZAP{} version 1 provided such
conversions, which are, however, not yet ported.}.

\subsection{Free Types}
\label{sec:freetypes}

\paragraph{Generators}

Free types are translated by \ZAP{} to sets which generate the type's
domain.  Consider the following definition:

\begin{zedgroup}
\begin{zdirectives}
  \zsection{FreeTypes}
\end{zdirectives}
\begin{zed}
  MySeq ::= nil | cons \ldata 1 \upto 3 \cross MySeq \rdata
\end{zed}
\end{zedgroup}

We can execute:

\begin{zexecexpr}
  MySeq
\yields
{nil,cons(1,nil),cons(1,cons(1,nil)),...}
\end{zexecexpr}


\paragraph{Using Constructors}

Constructors are functions as other functions are, which have
the property that they are \emph{(partial) injections}. Partiality
may be present because of the restrictions given for the domain 
in a free type declaration. 

We can use $cons\inv$ to convert a constructor to a projection,
and $x \in \ran cons$ to test whether a value is a variant of
this constructor.

Regarding the ``matching'' of constructed values.  in general the
constraint $ys' = cons(x,ys)$ can not be used if $x$ and $ys$ are
free.  The call to $cons$ will residuate until $x$ and $ys$ are bound,
since the encapsulated search implementing the call tries to execute
the constraint $(x,ys) \in \dom cons$, to which end it is not allowed
to produce bindings for $x$ and $ys$ (cf.~section \ref{sec:esearch}).
Thus, for matching one has to generally write $(x,ys) \mapsto ys' \in
cons$.

However, the constraints on the domain of a constructor can be
completely discarded, using the following declaration form
(cf.~section \ref{sec:decls}):

\begin{zedgroup}
\begin{zdirectives}
  \zsection{FreeTypes2}
\end{zdirectives}
\begin{zed}
  MySeq ::= nil | cons \ldata \{x: 1 \upto 3 \cross MySeq\} \rdata
\end{zed}
\end{zedgroup}

Now it becomes possible to write $ys' = cons(x,ys)$ for matching,
since there are no constraints imposed on $x$ and $ys$ which need
to be resolved in the encapsulated search of the constructor
application. Note, however, that now the assertion that
the first component of $cons$ is in the range $1 \upto 3$ is
not longer checked for a constructor call at runtime. 

Note that the above treatment is even necessary if we have total
constructor functions, e.g. $cons \ldata \num \cross MySeq \rdata$:
the reason is that $ys \in MySeq$ is a constraint which though
semantically is true, acts as a generator for values of $MySeq$.


\subsection{Genericity}

The genericity concept of Z allows to provide instantiations
with particular sets. For instance, given the definition
of the identity relation:

\begin{zedgroup}
\begin{zdirectives}
  \zsection{Genericity}
\end{zdirectives}
\begin{axdef}[X]
  id == \{x:X | x \in X @ (x,x)\}
\end{axdef}
\end{zedgroup}

\ldots we can provide instantiations as follows:

\begin{zexecexpr}
  id[\{1,2\}]
\yields  
{(1,1),(2,2)}
\end{zexecexpr}

On the implementation level this feature requires to represent each
generic constant as a function over its instantiation. Since in most
cases the instantiation is universal and imposes no 
constraints this is a serious efficiency problem.

For this reason, \ZAP{} compiles for each generic constant two
versions: one as a function over the instantiation argument, and the
other as a constant optimized for the case that the instantiation is
universal.  Which of these versions is selected is controlled by
whether an instantiation of a generic constant is \emph{inferred} or
\emph{provided}: for inferred instantiations the optimized version is
used.

\subsection{Numbers}

Though it is possible in our model to represent e.g.~natural numbers
by constructors $Z$ and $S \ldata N \rdata$, as shown in the previous
sections, enabling resolution techniques on them, this approach is
hopelessly inefficient for real applications.  Thus numbers are
actually integrated by a native implementation, and resolution
techniques for arithmetic constraints are not available for them
\footnote{It is planned to integrate resolution techniques for
  arithmetic constraints in future versions of \ZAP{}.}

To generate extensional number ranges, $n \upto m$ may be used:

\begin{zexecexpr}
  \{x:\num | x \in 1 \upto 1024; x \mod 2 = 0\}
\yields
{2,4,6,8,10,12,14,16,18,20,22,24,26,28,30,32,34,36,38,40,42,44,46,48,50,52,
 54,56,58,60,...}
\end{zexecexpr}

Note that $n \upto m$ forces the storage of a set extension
in memory. Thus $n$ and $m$ should be in a moderate range.
Use $\{x:\num|n \leq x \leq m\}$ if you do not actually want
to enumerate a range.


\subsection{Power-Sets and the Like}

The resolution techniques for subset constraints in our model
are relatively weak (they capture just functional and logic
computation, not more). For example, we are not able to enumerate the
set $\{x: \power A | x \subseteq S\}$ (which equals $\power S$), even
if $A$ is finitely enumeratable (indeed, we can test whether a ground
value is in this set).

Technically, the constraint resolution system of \ZAP{}
has problems with the following kind of constraints:

\begin{itemize}
\item subset constraints where the left or the right hand side is a
  free variable: $x \subseteq v$ or $v \subseteq x$.  These
  constraints can generally not be resolved before $x$ becomes bound.
  A typical instance is the powerset example given above.
\item constraints which represent unification problems on sets,
  for instance $x \cup y = \{1,2\}$. These
  constraints cannot produce bindings for $x$ and $y$, since set-unification
  is currently not employed in our model. Instead, they must residuate
  until other constraints provide bindings.
\end{itemize}




\section{The User Interface}

\ZAP{} is embedded in the diverse interfaces (graphical, XEmacs,
batch) provided by \zetalink{\Zeta{}}{zeta.html}. Here, we look
at the batch-oriented (command line) interface. Similar principles
apply to the other interfaces.

\subsection{Queries}

Execution is involved by the two single-letter commands \verb:e:
(execute expression) and \verb:p: (execute predicate). The output
of the \verb:help: command gives the following syntax for these commands:

\begin{verbatim}
  e [-u <UNITNAME>] [ ( -r | -t | -f ) ] ( -n | -d | -o | <EXPR> ) 
  p [-u <UNITNAME>] <PREDICATE> 
\end{verbatim}

The optional unit name specifies the section focus, as usual with
\Zeta{} commands. The first group of options to the \verb:e: command
has the following meaning:

\begin{itemize}
\item \verb:-r: -- ``raw mode'' -- indicates that sets in the
  result should not be enumerated.
\item \verb:-t: -- ``try mode'' -- indicates that sets in the
  result should be enumerated to a certain extend. This may
  fail, producing ``unresolved constraints'' diagnostics. This
  is the default.
\item \verb:-f: -- ``force mode'' -- indicates that sets in the
  result should be forced to be completely enumerated. This may lead
  to non-termination.
\end{itemize}

The last group of options of the \verb:e: command allows us
to step through the solutions to the entire specification
or a set-result. Consider the (loose) specification:

\begin{zedgroup}
\begin{zdirectives}
  \zsection{Stepping}   
\end{zdirectives}
\begin{axdef}
  S: \power \num
\where
  S \in \{\{1,2\},\{3,4\}\}
\end{axdef}
\end{zedgroup}

Evaluating $S$ yields in:

\begin{zexecexpr}
S
\yields
{1,2}
(possibly more solutions)
\end{zexecexpr}

Typing \verb:e -n: shows the next solution to $S$, \verb:{3,4}:.
From the current value we are observing, we can also step
downwards (if the current value is a set value): typing \verb:e -d:
now yields \verb:3 (possibly more solutions):, and \verb:e -n: afterwards
\verb:4:. With \verb:e -o: we return to the outer target.


\subsection{Profiling}

\ZAP{} provides two options for timing and profiling.  First, with the
command \verb:xconf ton:/\verb:xconf toff: timing of execution can
beactivated and deactivated.  

Second, the command \verb:xprof: can be used to display a profile
which counts the number of steps executed in each individual
constraint.



\section{Controlling Search}

The execution model of \ZAP{} uses concurrent constraint resolution,
employing the so-called \emph{Andorra Principle} to schedule search.
This means that deterministic computation is preferred for
non-deterministic computation: on resolving the constraints of
a goal, those constraints which try to create a choice point
are suspended until no other constraints can continue. 

If several choices remain in such a situation, those which are textual
``first'' are preferred. However, for recursive definitions, also
those choices which are ``older'' are preferred for those which are
``younger''. This is necessary in order to provide fairness. It may
lead to unexpected behavior, which is discussed in this section.

Consider a specification of the 8-queens problem:

\begin{zedgroup}
\begin{zdirectives}
  \zsection{queens}
\end{zdirectives} \\
\begin{axdef}
  N == 8 
\end{axdef}
\begin{zed}
  Pos == \num \cross \num 
\end{zed} \\
\begin{axdef}
  col == \lambda p: Pos @ p.1 \\
  row == \lambda p: Pos @ p.2 
\end{axdef}
\begin{axdef}
  up == \lambda p: Pos @ col(p) - row(p) \\
  down == \lambda p: Pos @ col(p) + row(p) \\
\end{axdef}
\begin{axdef}
  feasible == \{\<ps: \power Pos; p: Pos |
                \forall p': Pos | p' \in ps @ \\\t1
                 col(p) \neq col(p') \land
                 row(p) \neq row(p') \land
                 up(p) \neq up(p') \land
                 down(p) \neq down(p') \}\>
\end{axdef}
\begin{axdef}
  queens : \nat \rel \power Pos
\where
  queens = \\\t1
    \{(0, \emptyset)\} \cup \\ \t1
    \{\<
      n,c: \nat; qs,qs': \power Pos | \\
      n > 0;
      (n-1,qs) \in queens;
      c \in 1 \upto N \\ 
      (qs, (c,n)) \in feasible; 
      qs' = qs \cup \{(c,n)\}
       @ 
      (n, qs') \} \>
\end{axdef}
\end{zedgroup}

Execution of the goal 

\begin{zexecexpr}
  \{qs: \power Pos | (8,qs) \in queens\}
\end{zexecexpr}

leads to virtual endless search. What is wrong with the specification
(resp.~\ZAP{}'s current execution model)?

The problem is that $queens$ is computed again and again for the same
$n$. Though on the same recursion level the constraint $(n-1,qs) \in
queens$ preceed the constraint $c \in 1 \upto N$ and is thus
priorized, $(n-2,qs) \in queens$ is not priorized for $c \in 1 \upto
N$ on the upper level, since older constraints are priorized over younger
ones. Thus for each choice point of $c \in 1 \upto N$ the constraint
$(n-2,qs) \in queens$ is backtracked and recomputed.

In fact, there is no actual dependency between $(n-1,qs) \in queens$
and $c \in 1 \upto N$, and thus backtrack is not necessary.  However,
\ZAP{} is currently not capable of recognizing the independency.

There are several ways how a user can control search, discussed below.

\subsection{Sequentialization}

There are two primitive predicates, $\BEGINSEQ$ (\verb:\BEGINSEQ:)
and $\ENDSEQ$ (\verb:\ENDSEQ:) which can be used to enforce
sequential execution. Used for queens, this looks as follows:

\begin{axdef}
  queens_1 : \nat \rel \power Pos
\where
  queens_1 = \\\t1
    \{(0, \emptyset)\} \cup \\ \t1
    \{\<
      n,c: \nat; qs,qs': \power Pos | \\
      n > 0;
      \BEGINSEQ;
      (n-1,qs) \in queens_1;
      c \in 1 \upto N;
      \ENDSEQ \\ 
      (qs, (c,n)) \in feasible; 
      qs' = qs \cup \{(c,n)\}
       @ 
      (n, qs') \} \>
\end{axdef}

The constraints enclosed in $\BEGINSEQ$ and $\ENDSEQ$ are
strictly executed sequential. Note that this means, if one
of them residuates, the entire group residuates. For the
example, we thus ensure that choices for $c$ are always
appended to the bottom of the choice tree for $qs$.


\subsection{Extensionalization}

We can enforce the complete enumeration of the solutions on the
next recursion level using the $\EXT: \power A \pfun \power A$
(\verb:\EXT:) function from the toolkit. This functions takes
a set and yields its extensional (enumerated) version:

\begin{axdef}
  queens_2 : \nat \rel \power Pos
\where
  queens_2 = \\\t1
    \{(0, \emptyset)\} \cup \\ \t1
    \{\<
      n,c: \nat; qs,qs': \power Pos | \\
      n > 0;
      qs \in \EXT \{qs: \power Pos | (n-1,qs) \in queens_2 \};
      c \in 1 \upto N \\
      (qs, (c,n)) \in feasible; 
      qs' = qs \cup \{(c,n)\}
       @ 
      (n, qs') \} \>
\end{axdef}

Here, $qs \in \{\ldots\}$ is still backtracked if $c \in 1 \upto N$ makes
a choice, but since the result of $\{\ldots\}$ is already computed, this
does not hurts very much.

The $\EXT$ function can enhance efficiency in many other situations
which deal with finite sets. In general, if a set is defined
by a set-comprehension, its contents is recomputed at every
application point. This behavior is intended, since 
the application context determines what we actually
need to compute of a set (in a constraint $e \in S$, the
``pattern'' of the expression $e$ narrows what 
to compute from $S$). However, for problems such as $queens$
this behavior is not adequate.


\section{Special Definitions}

%% \zsection{Specials}

The mathematical toolkit which comes with \Zeta{} contains
some special definitions which are tailored for working with \ZAP{}
and are documented below.

\subsection{Debug Printing}

The function $\TRACE~x$ (\verb:\TRACE:) can be used to print the value
of $x$ to the session log. It just returns $x$. The function
$\PRINT(x,y)$ (\verb:\PRINT:) prints the value of $x$ and returns $y$.
The semantic definition, apart of these side-effects, is as follows:

\begin{zedgroup}
\begin{axdef}[X]
  \TRACE == \lambda x:X @ x
\end{axdef}
\begin{axdef}[X,Y]
  \PRINT == \lambda x:X; y:Y @ y
\end{axdef}
\end{zedgroup}

\subsection{Forcing}

The function $\FORCE~x$ (\verb:\FORCE:) residuates the executing
constraint until $x$ is bound (if $x$ is not a variable, it immediately
returns). The function $\DEEPFORCE~x$ (\verb:\DEEPFORCE:) does so
until \emph{all} free variables in $x$ are bound. Semantically,
both are identity:

\begin{zedgroup}
\begin{axdef}[X]
  \FORCE == \lambda x:X @ x 
\end{axdef}
\begin{axdef}[X]
  \DEEPFORCE == \lambda x:X @ x
\end{axdef}
\end{zedgroup}

\subsection{Assumptions}

The generic constant $\assumed X$ (\verb:\assumed:) takes a set and
converts it into a unversal set. Semantically, it is identity: 

\begin{zedgroup}
\begin{zdirectives}
  \zgeneric{1 (\assumed \_)}
\end{zdirectives}
\begin{axdef}[X]
  \assumed \_ == X
\end{axdef}
\end{zedgroup}

However, the compiler treats $\assumed X$ as $\{x:X\}$, discarding
any constraints involved by the instance of $X$.

This allows a usage as follows:

\begin{schema}{S}
  x: \num
\where
  \assumed [|x>0] \\
\end{schema}

The constraint $x>0$ is not checked at execution time.

There is also a \Zeta{} macro which supports the
above usage in a more natural notation:

\begin{verbatim}
%%macro \assert 1 [|#1]
\end{verbatim}

\begin{schema}{S}
  x: \num
\where
  \assert{x>0}
\end{schema}



\subsection{Extensionalization}

The function $\EXT~x$ (\verb:\EXT:) enforces the enumeration
of the set $x$, denotating its extensional internal representation.
It is undefined for sets which cannot be enumerated. The semantic
definition is identity:

\begin{zedgroup}
\begin{zdirectives}
  \zfunction{90 (\EXT \_)}
\end{zdirectives}
\begin{axdef}[X]
  \EXT \_ == \lambda S: \power X @ S
\end{axdef}
\end{zedgroup}

\subsection{Reduction Over Sets}

The function $setreduce$ implements homomorphisms over sets:

\begin{axdef}[X,Y]
  setreduce : (X \cross Y \pfun Y) \cross Y \cross \power X \pfun Y
\where
  setreduce = 
     \lambda f: X \cross Y \pfun Y; y: Y; S: \power X @ \\\t1
        \IF S = \emptyset \THEN y 
        \ELSE (\mu y': Y | 
                \forall x: X @ y' = setreduce(f,f(x,y),S \setminus \{x\}))
\end{axdef}

Semantically, it is thus only defined for functions $f$ which are
commutative in the first argument ($f(x,f(x',y)) =
f(x',f(x,y))$. However, this condition is not checked in the actual
implementation.

\subsection{Reduction Over Sequences}

Folding on sequences, as well-known from functional
programming, is provided:

\begin{axdef}[X,Y]
  foldr : (X \cross Y \pfun Y) \cross Y  \fun \seq X \pfun Y
\where
  foldr = \lambda f: X \cross Y \pfun Y; y: Y @
           \lambda s: \seq X @
              \IF s = \langle\rangle \THEN y 
              \ELSE f(head~s,foldr(f,y)(tail~s))
\end{axdef}

\begin{axdef}[X,Y]
  foldl : Y \cross (Y \cross X \pfun Y) \fun \seq X \pfun Y
\where
  foldl = \lambda y: Y; f: Y \cross X \pfun Y @
           \lambda s: \seq X @
              \IF s = \langle\rangle \THEN y 
              \ELSE foldl(f(y,head~s),f)(tail~s)
\end{axdef}



\subsection{Extensional Powerset}

The function $\efinset X$ (\verb:\efinset:) constructs the
extensional powerset of the set $X$, which must be enumerable.
Semantically, it is the same as $\finset X$:

\begin{zedgroup}
\begin{zdirectives}
  \zfunction{90 (\efinset \_)}
\end{zdirectives}
\begin{axdef}[X]
  \efinset \_ == \lambda S: \power X @ \finset X
\end{axdef}
\end{zedgroup}

We do not define $\finset$ itself to deliver an extension since for
even small domains powersets easily explode -- $\efinset
(1 \upto 16)$ contains $65536$ elements.


\subsection{Reflection}

A simple concept of reflection is provided, which is currently only
used for type-safe conversion of data given in a denotation or a file
into a runtime value (next section).

The function $\typeinfo T$ results a value representing
information of the base type of the set $T$. The function
$\giventypeinfo T$ results information about a free type
definition associated with the base type of $T$, which must
be a simple type name. These functions can only be used
if $T$ does not contains generic types.


\begin{zedgroup}
\begin{zed}
  [TYPEINFO[A]]
\end{zed}
\begin{zed}
  [GIVENTYPEINFO[A]]
\end{zed}
\begin{zdirectives}
  \zgeneric{10 (\typeinfo \_)} \\
  \zgeneric{10 (\giventypeinfo \_)}
\end{zdirectives} \\
\begin{axdef}[A]
  \typeinfo \_ : TYPEINFO[A] \\
\end{axdef}
\begin{axdef}[A]
  \giventypeinfo \_ : GIVENTYPEINFO[A]
\end{axdef}
\end{zedgroup}

\subsection{Data Input}

For the purpose of e.g.~test evaluation, \ZAP{} provides
a type-safe conversion of data found in a file or in a denotation
into values. This is achieved by the following functions:

\begin{zedgroup}
\begin{zdirectives}
 \zfunction{9 (\_ \fromfile \_)} \\
 \zfunction{9 (\_ \fromdeno \_)}   
\end{zdirectives} \\
\begin{axdef}[A]
  \_ \fromfile \_ : TYPEINFO[A] \cross \denotation \pfun \seq A
\end{axdef}
\begin{axdef}[A]
  \_ \fromdeno \_ : TYPEINFO[A] \cross \denotation \pfun \seq A
\end{axdef}
\end{zedgroup}

These functions parse the denotation or the contents of the file
according to the following grammar, parameterized over types,
with $P_{\tau}$ the start symbol, reading a sequence of values,
and $V_{\tau}$ the non-terminal reading a single value:

\begin{zdirectives}
  P_{\tau}  & \rightarrow & \epsilon | P_{\tau} ~ V_{\tau} \\
  V_{\num}  & \rightarrow & \mathtt{number} \\
  V_{\power \tau}  & \rightarrow & \mathtt{\{} V_\tau \ldots V_\tau \mathtt{\}} \\
  V_{\tau_1 \cross \ldots \cross \tau_n}
            & \rightarrow & V_{\tau_1} \ldots V_{\tau_n} \\
  V_{[x_1:\tau_1; \ldots; x_n: \tau_n]}
            & \rightarrow & V_{\tau_1} \ldots V_{\tau_n} \\
  V_{T ::= cons_1 \ldata \tau_1 \rdata | \ldots | cons_n \ldata \tau_n \rdata}
            & \rightarrow & \mathtt{cons}_1 V_{\tau_1} | \ldots |
                          \mathtt{cons}_n V_{\tau_n} 
\end{zdirectives}

Tokens are numbers, constructor names, and braces.  Arbitrary white
space is allowed between tokens. Here is an example:

\begin{zedgroup}
\begin{zdirectives}
  \zsection{DataInput}
\end{zdirectives}
\begin{zed}
  Seq[A] ::= nil | cons \ldata A \cross Seq[A] \rdata 
\end{zed}
\end{zedgroup}

\begin{zexecexpr}
  \typeinfo Seq[\power [x: \num; y: \num]] ~ \fromdeno 
            \ZD{cons \{1 2\} cons \{\} cons \{5 6 7 8\} nil nil}
\yields
<cons({<x==1,y==2>},cons({},cons({<x==5,y==6>,<x==7,y==8>},nil))),nil>
\end{zexecexpr}

\bibliographystyle{plain}
\bibliography{compu,lang,new,eabpub,eabtech,uebb,z98,statecharts} 

\end{document}

