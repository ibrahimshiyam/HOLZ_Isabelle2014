\documentclass{article}

\usepackage{a4wide}
\usepackage{zeta}

\parindent0pt
\parskip3pt

\newenvironment{remark}
{\begin{small}\begin{quote}\emph{Remark}: }{\end{quote}\end{small}}
  
\begin{document}
\author{Wolfgang Grieskamp}
\title{Test-Case Evaluation for the Birthday Book Example}
\maketitle
\begin{abstract}

Below, the well-known example of the \emph{birthday book}
specification is used to illustrate the basic operation of \ZAP.
Starting with the usual abstract specification of the birthday book,
we refine it for the purpose of executability (mainly fixing given
types in the specification), and add a framework for evaluating
\emph{test-data} described by input/output behavior.
The test-data is supposed to be be observed from an
implementation of the birthday book. The example illustrates most
of the capabilities of \ZAP{}: the ability to execute the schema
calculus, to lift schemas to functions, and to define recursive
higher-order functions and relations. 

\end{abstract}




\section{The Basic Specification}

The specification is partitioned into several Standard Z sections. The
first one, $BBSpec$, contains the basic, or ``abstract'' specification
of the birthday book example, as usually found in the literature.

The section $BBSpec$ is opened with the following declaration (all
paragraphs following this declaration until the next section
declaration will belong to $BBSpec$):

\begin{zdirectives} % an environment which behaves for LaTeX like zed, 
                    % but contains special Z paragraphs such as 
                    % sections and operator template declarations.
  \zsection{BBSpec}
\end{zdirectives}

We first define the given types $NAME$ and $DATE$, and the data
state of the birthday book:

\begin{zedgroup} % using this environment lets the LaTeX style
                 % try to layout the enclosing definitions side-by-side
                 % (if they fit)

\begin{zed}
  [NAME,DATE]
\end{zed}

\begin{schema}{BirthdayBook}
  known: \power NAME \\
  birthday: NAME \pfun DATE
\where
  known = \dom birthday
\end{schema}

\end{zedgroup}

An \emph{initial state} of the birthday book is given by the following
schema:

\begin{axdef}
 InitBirthdayBook == [BirthdayBook | birthday = \emptyset] 
\end{axdef}

The value of this schema can be evaluated:

\begin{zexecexpr}
 InitBirthdayBook
\yields
 {<birthday == {},known == {}>}
\end{zexecexpr}

\begin{remark}
  In order to perform evaluation of $InitBirthdayBook$, either type in
  the name in the \texttt{Execute} panel of the GUI commander, or in
  the command line of the XEmacs commander. Under XEmacs, you may also
  do the following: select the literal text \texttt{InitBirthdayBook}
  in the source of this document (\texttt{bbook.tex}) using the mouse,
  and then either choose \texttt{Region as Expression} from the
  \Zeta{} menu, or, as a shortcut, click button 2 while pressing the
  meta key.
\end{remark}

Further evaluation of birthday book values fails because the types
$NAME$ and $DATE$ are still abstract -- we don't know yet how
to denote values of these types. Abilities for doing so will be
introduced in the next Section.

For the basic birthday book specification, we define some operations:

\begin{zedgroup}
\begin{schema}{AddBirthday}
  \Delta BirthdayBook \\
  name?: NAME; date?: DATE
\where
  name? \notin known \\
  birthday' = birthday \cup \{name? \mapsto date?\}
\end{schema}
\begin{schema}{FindBirthday}
  \Xi BirthdayBook \\
  name?: NAME; date!: DATE 
\where
  name? \in known \\
  date! = birthday(name?)
\end{schema}
\begin{schema}{Remind}
  \Xi BirthdayBook \\
  today?: DATE; cards!: \power NAME
\where
  cards! = \{ n: NAME | n \in known; birthday(n) =   today? \}\\
\end{schema}
\begin{schema}{Remove}
  \Delta  BirthdayBook \\
  name?:  NAME
\where
  birthday' = \{ name?  \} \ndres birthday\\
\end{schema}
\end{zedgroup}

The schema $AddBirthday$ adds a mapping $name? \mapsto date?$ to
the birthday books state, the schema $FindBirthday$ lookups the
birthday of a particular $name?$, and the schema $Remind$ yields
the set of names whose birthday is $today?$. The schema $Remove$
deletes the entry for $name?$ from the birthday book. 

\section{Refinement for Execution}

A refinement of the section $BBSpec$ for execution is defined by the
section $BBExec$:

\begin{zdirectives}
  \zsection[BBSpec]{BBExec}
\end{zdirectives}

%%zap protected

The purpose of the refinement is to fix the given types $NAME$ and
$DATE$. The Z of \Zeta{} allows to refine given
types by free types. Below, the type of dates is described by the
schema ${}Date$:

\begin{zedgroup}
\begin{zed}
  NAME ::= Werner | Barbara | Martin\\
  DATE ::= date \ldata Date \rdata \\
\end{zed} \\

\begin{schema}{Date}
  day : \nat;
  month : \nat;
  year : \nat
\where
  day \in 1 \upto 31 \\
  month \in 1 \upto 12;
  month = 2 \implies day \leq 29;
  month \in \{4,6,7,9,11\} \implies day \leq 30 \\
  year \geq 0
\end{schema}
\end{zedgroup}

\begin{remark}
  One may ask why, in the schema $Date: 2000/07/20 06:44:07 $, we put constraints such as
  $day \in 1 \upto 31$ in the property part instead of in the
  declaration part, declaring $date: 1 \upto 31$, as it is Z praxis.
  The reason is that \ZAP{} ignores constraints imposed by
  declarations $n:E$. In many cases, such constraints are not
  executable. Consider e.g.~the declaration of a function to be total:
  $f: \num \fun \num$. If $f$ is non-finitely defined, this constraint
  can not be resolved.
\end{remark}

We can now evaluate expressions by applying the $date$ constructor
to bindings:

\begin{zexecexpr}
  date \lbind day == 1, month == 1, year == 1999 \rbind   
\yields
date <day == 1,month == 1,year == 1999>
\end{zexecexpr}

The next application of the $date$ constructor is undefined,
since the passed binding is not in its domain (the month $2$,
February, has no $30$th day):

\begin{zexecexpr}
  date \lbind day == 30, month == 2, year == 1999 \rbind 
\yields
WARNING[<ZAM>]: 
  undefined in toplevel goal {
    resolving ZAPSCRATCHUNIT GLOBALS
    unresolved constraints:
     LTX:bbook.tex(199.3-199.56) undefined: 
      mu-value undefined
  }
*unknown*
\end{zexecexpr}

Next we introduce some convenience functions for describing operations
on birthday books in the schema calculus.  These functions make use of
new features of the Standard regarding unification of schema
expressions and plain expressions: they yield a set of bindings, that
is a schema, which can be used in schema expressions:

\begin{axdef}
  add == 
   \lambda name?: NAME;  date?:  DATE @ [\Delta BirthdayBook | AddBirthday] \\
  find == 
    \lambda name?: NAME @ [\Delta BirthdayBook; date!: DATE | FindBirthday ] \\
  remind == 
    \lambda today?: DATE @ 
      [\Delta BirthdayBook; cards!: \power NAME  | Remind ] \\
  remove == 
    \lambda name?: NAME @ [\Delta BirthdayBook | Remove ]
\end{axdef}

We can now express computations on the birthday book, starting
from the initial state:

\begin{zexecexpr}
  InitBirthdayBook \\\t1
  \semi add(Werner, date \lbind   day == 7, month == 5, year == 1929 \rbind)
\yields
{<birthday == {},
  birthday' == {(Werner,date <day == 7,month == 5,year == 1929>)},
  known == {},known' == {Werner}>}
\end{zexecexpr}

\begin{zexecexpr}
 InitBirthdayBook   \\\t1
 \semi add(Werner, date \lbind day==7,month==5,year==1929 \rbind) \\\t1
 \semi add(Barbara,date \lbind day==18,month==11, year==1935 \rbind)\\\t1
 \semi add(Martin,date \lbind day==18, month==11, year==1935 \rbind)\\\t1
 \semi find(Werner)\\\t1
 \semi remind(date \lbind day==18, month==11, year==1935 \rbind)\\\t1
 \semi remove(Martin)
\yields
{<birthday == {},
  birthday' ==
   {(Barbara,date <day == 18,month == 11,year == 1935>),
    (Werner,date <day == 7,month == 5,year == 1929>)},
  cards! == {Barbara,Martin},
  date! == date <day == 7,month == 5,year == 1929>,known == {},
  known' == {Barbara,Werner}>}
\end{zexecexpr}

The last evaluation deserves some notes on the $\_ \semi \_$ schema
operation. $Op_1 \semi Op_2$ binds primed names $x'$ from the
signature of $Op_1$ to unprimed counterparts $x$ in the signature of
$Op_2$. Variables not bound this way (such as $cards!$ and $date!$) 
or kept as is. In the evaluation result, these variables represent the
result of applying the $find$ and $remind$ operations, respectively.

The application of a unique schema $InitBirthdaybook$ (given by a
singleton binding set) to deterministic operations yields a singleton
binding set. \ZAP{} is also capable of handling non-unique initial states
and indeterministic operations. Consider the following definitions:

\begin{zed}
  InitBirthdayBook2 == \\\t1
    \< InitBirthdayBook \lor \\
       [ BirthdayBook | \\\t1
           birthday = \{Werner \mapsto 
                         date \lbind day==7,month==5,year==1929 \rbind\} ] \>
\end{zed}

Evaluations yield:

\begin{zexecexpr}
 InitBirthdayBook2
\yields
{<birthday == {},known == {}>,
 <birthday == {(Werner,date <day == 7,month == 5,year == 1929>)},
  known == {Werner}>}
\end{zexecexpr}
\begin{zexecexpr}
 InitBirthdayBook2 \\\t1
 \semi (add(Barbara,date \lbind day==18,month==11, year==1935 \rbind)\\\t2
    \lor add(Martin,date \lbind day==18, month==11, year==1935 \rbind))
\yields
{<birthday == {},
  birthday' == {(Barbara,date <day == 18,month == 11,year == 1935>)},
  known == {},known' == {Barbara}>,
 <birthday == {},
  birthday' == {(Martin,date <day == 18,month == 11,year == 1935>)},
  known == {},known' == {Martin}>,
 <birthday == {(Werner,date <day == 7,month == 5,year == 1929>)},
  birthday' ==
   {(Barbara,date <day == 18,month == 11,year == 1935>),
    (Werner,date <day == 7,month == 5,year == 1929>)},known == {Werner},
  known' == {Barbara,Werner}>,
 <birthday == {(Werner,date <day == 7,month == 5,year == 1929>)},
  birthday' ==
   {(Martin,date <day == 18,month == 11,year == 1935>),
    (Werner,date <day == 7,month == 5,year == 1929>)},known == {Werner},
  known' == {Martin,Werner}>}
\end{zexecexpr}

Of course, such evaluations might easily lead to an explosion of
possibilities. 

\begin{remark}
  To understand the principles behind the above evaluations, recall
  that the schema calculus just works on \emph{sets} of bindings. The
  schema operation $S_1 \land S_2$ is just a shortcut for the set
  intersection $S_1\omega_1 \cap S_2\omega_2$, where $\omega_i$'s are
  according morphisms which map the binding sets $S_i$ to their
  smallest common signature. Similarly, $S_1 \lor S_2$ is
  $S_1\omega_1 \cup S_2\omega_2$. The schema operation $S_1 \semi S_2$
  is just a shortcut for $(S_1\omega_1 \land S_2\omega_2)\omega_3$, where
  $\omega_1$ and $\omega_2$ are morphisms which identify the bound variables,
  and $\omega_3$ is a morphism which forgets the bound variables. In fact,
  the execution model of \ZAP{} is based mostly on set algebra
  and morphisms on sets.
\end{remark}


\section{Refinement for Testing}

A refinement of the section $BBExec$ for the purpose of evaluation of
test-data with ZAP is given by the section $BBTest$:

\begin{zdirectives}
  \zsection[BBExec]{BBTest}
\end{zdirectives}

The test-data will be specified by sequences of input and
output behavior of the birthday book. The original specification
does not provide uniform input and output interfaces. Yet, we can
lift the original specification without modifying it to such uniform
interfaces.

First, we define free types for describing the input and output behavior:

\begin{zed}
 INPUT ::= \\\t1
  addI \ldata  NAME \cross DATE \rdata |\\\t1
  findI \ldata NAME \rdata |\\\t1
  remindI \ldata DATE \rdata |\\\t1
  removeI \ldata NAME \rdata\\
 OUTPUT ::= \\\t1
  okayO |\\\t1
  dateO \ldata DATE \rdata |\\\t1
  namesO \ldata \power NAME \rdata\\
\end{zed}

Next, we lift the operations of the original specification to ones with
uniform input/output interfaces:

\begin{zed}
 UniformAdd == \\\t1
  [ \Delta BirthdayBook; in: INPUT; out: OUTPUT | \\\t2
     \exists name?: NAME; date?: DATE @ \\\t3
       (name?, date?) \mapsto in \in addI 
       \land AddBirthday 
       \land out = okayO ]  \\
 UniformFind == \\\t1
  [ \Delta BirthdayBook; in: INPUT; out: OUTPUT | \\\t2
     \exists name?: NAME; date!: DATE @ \\\t3
       name? \mapsto in \in findI 
       \land FindBirthday 
       \land date! \mapsto out \in dateO ]\\
 UniformRemind == \\\t1
  [ \Delta BirthdayBook; in: INPUT; out: OUTPUT | \\\t2
     \exists today?: DATE; cards!: \power NAME @ \\\t3
        today? \mapsto in \in remindI 
        \land Remind 
        \land cards! \mapsto out \in namesO ]\\
 UniformRemove == \\\t1
  [ \Delta BirthdayBook; in: INPUT; out: OUTPUT | \\\t2
     \exists name?: NAME @  \\\t3
       name? \mapsto in \in removeI 
       \land Remove 
       \land out = okayO ]\\
\end{zed}

\begin{remark}
  Why do we write e.g.~$(name?,date?) \mapsto in \in addI$ instead of
  $in = addI(name?,date?)$? The reason is that $addI$ is partial
  function, and \ZAP{} is not allowed to perform backwards
  computation (that is computing the inputs from the outputs) 
  for such functions. Only if the domain of a constructor function
  is total, the equality notation is allowed.
\end{remark}

\ZAP{} has no problems with the execution of such morphed operations.
Instead of illustrating this we directly continue by defining a
framework which tests if sequences of $INPUTS$ and $OUTPUTS$ do match
the specification. The final evaluation of the test-data will proof
our claim.

On the level of the schema calculus, a ``state transition'' of the
birthday book is defined as the disjunction of all the 
operations with uniform I/O interface:

\begin{zed}
  UniformOps == UniformAdd \lor UniformFind \lor UniformRemind \lor UniformRemove
\end{zed}

Next we define a function which, given a pair of input and
output states, yields a relation between admissible pre and post states
of the birthday book w.r.t.~this I/O behavior:

\begin{axdef}
  step == \lambda in: INPUT; out: OUTPUT @ \\\t3
\{ \Delta BirthdayBook | UniformOps @ (\theta BirthdayBook, \theta BirthdayBook ') \}
\end{axdef}


Finally, a recursive ``test''-function is defined which computes the
post-states resulting from a sequence of input and output traces. The
function takes as a parameter a set of initial states -- the condition
to let it be executable is that this set is finitely enumeratable:

\begin{axdef}
  test : \power BirthdayBook \fun 
             \seq INPUT \cross \seq OUTPUT \fun  \power  BirthdayBook\\
\where
  test = \lambda  init: \power BirthdayBook @ \\\t2
          \lambda  is: \seq INPUT; os: \seq OUTPUT | \# is =  \# os @\\\t3
           \IF \# is = 0 \THEN init \\\t3
           \ELSE  step(last~is, last~os)\limg  test(init)( front~is, front~os) \rimg 
\end{axdef}


To test ``test'', some I/O traces representing test data are defined:

\begin{axdef}
 date1 == date \lbind   day == 7, month == 5, year == 1929 \rbind\\
 date2 == date \lbind   day == 18, month == 11, year == 1935 \rbind\\
\end{axdef}

\begin{axdef}
 in1  == \langle addI(Werner, date1) \rangle \\
 out1 == \langle okayO \rangle 
\end{axdef}

\begin{axdef}
 in2   ==  \langle addI(Werner, date1), findI(Werner) \rangle \\
 out2  ==  \langle okayO, dateO(date1) \rangle\\
 out2a ==  \langle okayO, dateO(date2) \rangle
\end{axdef}

\begin{axdef}
 in3  == \langle addI(Werner, date1), addI(Barbara, date2), 
                 addI(Martin, date2),\\\t3
                 findI(Werner), remindI(date2) \rangle \\
 in3a  ==  \langle addI(Werner, date1), addI(Werner, date2), 
                 addI(Barbara, date2),\\\t3
                 findI(Werner), remindI(date2) \rangle \\
 out3 ==  \langle okayO, okayO, okayO, dateO(date1), 
                  namesO(\{Barbara, Martin\}) \rangle \\
 out3a ==  \langle okayO, okayO, okayO, dateO(date1), 
                   namesO(\{Barbara\}) \rangle \\
\end{axdef}

Here are the evaluation results:

\begin{zexecexpr}
 test(InitBirthdayBook)(in1, out1)
\yields
{<birthday == {(Werner,date <day == 7,month == 5,year == 1929>)},
  known == {Werner}>}
\end{zexecexpr}

\begin{zexecexpr}
 test(InitBirthdayBook)(in2, out2)  
\yields
{<birthday == {(Werner,date <day == 7,month == 5,year == 1929>)},
  known == {Werner}>}
\end{zexecexpr}

\begin{zexecexpr}
 test(InitBirthdayBook)(in2, out2a) 
\yields
{}
\end{zexecexpr}

\begin{zexecexpr}
 test(InitBirthdayBook)(in3, out3) 
\yields
{<birthday ==
   {(Barbara,date <day == 18,month == 11,year == 1935>),
    (Martin,date <day == 18,month == 11,year == 1935>),
    (Werner,date <day == 7,month == 5,year == 1929>)},
  known == {Barbara,Martin,Werner}>}
\end{zexecexpr}

\begin{zexecexpr}
 test(InitBirthdayBook)(in3a, out3) 
\yields
{}
\end{zexecexpr}

\begin{zexecexpr}
 test(InitBirthdayBook)(in3, out3a) 
\yields
{}
\end{zexecexpr}

Failing tests are indicated in that they yield an empty set as the
post-state of an execution.


\end{document}

%%% Local Variables: 
%%% mode: latex
%%% TeX-master: t
%%% End: 
