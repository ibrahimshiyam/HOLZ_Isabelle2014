% $Id: Toolkit.tex 6805 2007-08-17 14:48:19Z brucker $
% ZETA Z toolkit 
% compatible with the ZRM
% based on work of Mike Spivey, Copyright (C) J.M. Spivey 1989, 1991 

\documentclass{article}

\parindent0pt
\parskip3pt
\usepackage{a4wide,zeta}

\begin{document}
\title{The \Zeta{} Toolkit}
\author{Wolfgang Grieskamp\thanks{Based on work of Mike Spivey, Copyright (C) J.M. Spivey 1989, 1991}}
\maketitle 

Note: this is \emph{not} a documentation of the toolkit.

%%%%%%%%%%%%%%%%%%%%%%%%%%%%%%%%%%%%%%%%%%%%%%%%%%%%%%%%%%%%%%%%%%%%%%%%%%%%%
\section{Parents}

This toolkit has no toolkit as implicit parent:

\begin{verbatim}
%%toolkit
\end{verbatim}


%%%%%%%%%%%%%%%%%%%%%%%%%%%%%%%%%%%%%%%%%%%%%%%%%%%%%%%%%%%%%%%%%%%%%%%%%%%%%
\section{Layout}

\begin{verbatim}
%%macro \also 0 \\
%%macro \Also 0 \\
%%macro \ALSO 0 \\
%%macro ~ 0
%%macro & 0
%%macro \t 1
%%macro \< 0
%%macro \> 0
%%macro \quad 0
%%macro \qquad 0
%%macro \; 0
%%macro \, 0
%%macro \! 0
\end{verbatim}

%%%%%%%%%%%%%%%%%%%%%%%%%%%%%%%%%%%%%%%%%%%%%%%%%%%%%%%%%%%%%%%%%%%%%%%%%%%%%
\section{Assumed Function}

\zgeneric{1 (\assumed \_)}

\begin{verbatim}
%%zap given (\assumed\_)
%%macro \assert 1 [|#1]
\end{verbatim}

\begin{axdef}[X]
  \assumed \_ : \power X
\end{axdef}


%%%%%%%%%%%%%%%%%%%%%%%%%%%%%%%%%%%%%%%%%%%%%%%%%%%%%%%%%%%%%%%%%%%%%%%%%%%%%
\section{Tracing}

\begin{verbatim}
%%zap native NATIVEtrace(?,?)
\end{verbatim}

\begin{axdef}[X,Y]
  NATIVEtrace : \assumed X \cross Y \fun Y
\end{axdef}

\begin{axdef}[X]
  \TRACE == \lambda x: \assumed X @ NATIVEtrace(x,x)
\end{axdef}

\begin{axdef}[X,Y]
  \PRINT == \lambda x: \assumed X; y: \assumed Y @ NATIVEtrace(x,y)
\end{axdef}


%%%%%%%%%%%%%%%%%%%%%%%%%%%%%%%%%%%%%%%%%%%%%%%%%%%%%%%%%%%%%%%%%%%%%%%%%%%%%
\section{Forcing}

\begin{verbatim}
%%zap native NATIVEdeepForce(!)
%%zap native NATIVEforce(!)
%%zap native NATIVEwaitfor(!,?)
\end{verbatim}

\begin{axdef}[X]
  NATIVEdeepForce : \assumed X \fun X
\end{axdef}

\begin{axdef}[X]
  NATIVEforce : \assumed X \fun X
\end{axdef}

\begin{axdef}[X,Y]
  NATIVEwaitfor : \assumed X \cross Y \fun Y
\end{axdef}



\begin{axdef}[X]
  \DEEPFORCE == \lambda x: \assumed X @ NATIVEdeepForce(x)
\end{axdef}

\begin{axdef}[X]
  \FORCE == \lambda x: \assumed X @ NATIVEforce(x)
\end{axdef}

\zfunction{2 (\_ \whenbound \_)}

\begin{axdef}[X,Y]
  \_ \whenbound \_ == \lambda x: \assumed X; y: \assumed Y @ NATIVEwaitfor(x,y)
\end{axdef}


%%%%%%%%%%%%%%%%%%%%%%%%%%%%%%%%%%%%%%%%%%%%%%%%%%%%%%%%%%%%%%%%%%%%%%%%%%%%%
\section{Memoization}

\begin{verbatim}
%%zap native NATIVEmemoize(!,!)
\end{verbatim}

\begin{axdef}[X]
  NATIVEmemoize : \assumed \num \cross \power X \fun \power X
\end{axdef}

\begin{axdef}[X]
  \MEMOIZE == \lambda x: \assumed \power X @ NATIVEmemoize(16, x)
\end{axdef}

%%%%%%%%%%%%%%%%%%%%%%%%%%%%%%%%%%%%%%%%%%%%%%%%%%%%%%%%%%%%%%%%%%%%%%%%%%%%%
\section{Sequentialization}

\begin{axdef}
  \BEGINSEQ, \ENDSEQ : \power []
\end{axdef}


%%%%%%%%%%%%%%%%%%%%%%%%%%%%%%%%%%%%%%%%%%%%%%%%%%%%%%%%%%%%%%%%%%%%%%%%%%%%%
\section{Wrapper of a Native Homomorphism}

\begin{zed}
  [NATIVEhomT] % pseudo type used for HOM constants
\end{zed}

% the wrapper function
\begin{axdef}[X,Y]
  NATIVEhom: \assumed NATIVEhomT \cross \power X \pfun Y
\end{axdef}


%%%%%%%%%%%%%%%%%%%%%%%%%%%%%%%%%%%%%%%%%%%%%%%%%%%%%%%%%%%%%%%%%%%%%%%%%%%%%
\section{Equality}

A definition for equality is provided for the case that equality is
used without args (with args it is transformed to a builtin): we use
in the definition below):

\begin{verbatim}
%%inrel = \neq
\end{verbatim}

\begin{axdef}[X]
  \_ = \_ : \assumed \power (X \cross X)
\where
  (\_ = \_) \in \{\{x1,x2:X|x1=x2\}\} 
\end{axdef}

\begin{axdef}[X]
  \_ \neq \_ == \{x1,x2:\assumed X|\lnot x1=x2\} 
\end{axdef}


%%%%%%%%%%%%%%%%%%%%%%%%%%%%%%%%%%%%%%%%%%%%%%%%%%%%%%%%%%%%%%%%%%%%%%%%%%%%%
\section{Booleans}

\begin{verbatim}
%%zap given \zbool
\end{verbatim}

\begin{zed}
  \zbool == \power []
\end{zed}

\begin{zed}
  \zfalse == \emptyset[[]] \\
  \ztrue == \{\lbind\rbind\}
\end{zed}



%%%%%%%%%%%%%%%%%%%%%%%%%%%%%%%%%%%%%%%%%%%%%%%%%%%%%%%%%%%%%%%%%%%%%%%%%%%%%
\section{Product}

\begin{verbatim}
%%inop \mapsto 1
\end{verbatim}

\begin{axdef}[X,Y]
   first == \lambda x:\assumed X;y:\assumed Y  @ x
\end{axdef}
\begin{axdef}[X,Y]
   second == \lambda x:\assumed X;y:\assumed Y @ y
\end{axdef}
\begin{axdef}[X,Y]
   \_ \mapsto \_ == \lambda x:\assumed X; y: \assumed Y @ (x,y)
\end{axdef}

%%%%%%%%%%%%%%%%%%%%%%%%%%%%%%%%%%%%%%%%%%%%%%%%%%%%%%%%%%%%%%%%%%%%%%%%%%%%%
\section{Sets}

% Set Constructors

\begin{axdef}
  NATIVEempty, NATIVEnotempty, NATIVEext,\\
  NATIVEpower, NATIVEbigcap, NATIVEcard : NATIVEhomT
\end{axdef}

\begin{verbatim}
%%pregen \power_1 \finset \finset_1 
\end{verbatim}

\zfunction{90 (\EXT \_)}

\zfunction{90 (\efinset \_)}


\begin{axdef}[X]
  \EXT \_ == \lambda S:\power X @ NATIVEhom[X,\power X](NATIVEext,S)
\end{axdef}

\begin{axdef}[X]
  \power_1 \_ == 
     \{S: \assumed \power X | 
          NATIVEhom[X,\zbool](NATIVEnotempty, S) = \ztrue\}
\end{axdef}

\begin{axdef}[X]
  \finset \_ == \power X
\end{axdef}

\begin{axdef}[X]
  \efinset \_ == \lambda S:\power X @ 
         NATIVEhom[X,\power (\power X)](NATIVEpower,S)
\end{axdef}


% Empty Set

\begin{axdef}[X]
  \emptyset: \power X
\end{axdef}


% Membership and Subset

\begin{verbatim}
%%inrel \in \notin \subseteq \subset
\end{verbatim}

\begin{axdef}[X]
   \_ \in \_ : \assumed \power(X \cross \power X)
\where
  (\_ \in \_) = \{x: \assumed X; S: \assumed \power X | x \in S\}
\end{axdef}
\begin{axdef}[X]
   \_ \notin \_ == \{x: \assumed X; S:\assumed \power X | \lnot x \in S\}
\end{axdef}
\begin{axdef}[X]
   \_ \subseteq \_ : \assumed \power(\power X \cross \power X)
\where
   (\_ \subseteq \_) = \{S,T: \assumed \power X | S \subseteq T\}
\end{axdef}
\begin{axdef}[X]
   \_ \subset \_ == \{S,T: \assumed \power X | S \subseteq T \land 
                         (\exists x: \assumed T | x \in T @ \lnot x \in S)\}
\end{axdef}


% Set Operations

%%inop \cup \setminus 3
%%inop \cap 4

\begin{axdef}[X]
   \_ \cup \_ : \assumed \power X \cross \power X \fun \power X
\where
   (\_  \cup \_)  = \lambda S1,S2: \assumed \power X @ S1 \cup S2 
\end{axdef}
\begin{axdef}[X]
   \_ \cap \_ : \assumed \power X \cross \power X \fun \power X
\where
   (\_  \cap \_)  = \lambda S1,S2: \assumed \power X @ S1 \cap S2 \\
\end{axdef}
\begin{axdef}[X]
   \_  \setminus \_  == \lambda S1,S2: \assumed \power X @ 
                           S1 \cap \{x: \assumed X|\lnot x \in S2\} \\
\end{axdef}
\begin{axdef}[X]
   \bigcup == \lambda S: \assumed \power(\power X) @ 
               \{s:\assumed \power X; x: \assumed X | s \in S; x \in s @ x\}
\end{axdef}
\begin{axdef}[X]
   \bigcap == \lambda S: \assumed \power(\power X) @ 
                       NATIVEhom[\power X,\power X](NATIVEbigcap, S)
\end{axdef}

% Set reduce (FIXME: put onto ZAM level)

\begin{axdef}[X,Y]
  setreduce == \lambda f: \assumed X \cross Y \pfun Y; y: \assumed Y;
                    S: \assumed \power X @ \\\t1
              AUXreduce(f,y,NATIVEhom[X,\power X](NATIVEext, S))
\end{axdef}



\begin{verbatim}
%%zap native NATIVEextIsEmpty(!)
%%zap native NATIVEextSplit(!)
\end{verbatim}

\begin{axdef}[X]
  NATIVEextIsEmpty: \assumed \power X
\end{axdef}

\begin{axdef}[X]
  NATIVEextSplit: \assumed \power X \fun X \cross \power X
\end{axdef}

\begin{axdef}[X,Y]
  AUXreduce : \assumed (X \cross Y \pfun Y) \cross Y \cross \power X
                    \pfun Y
\where
  AUXreduce = \lambda f: \assumed X \cross Y \pfun Y; y: \assumed Y;
                    S: \assumed \power X @ \\\t1
     \IF S \in NATIVEextIsEmpty \THEN y \\\t1
     \ELSE \mu x:\assumed X; S': \assumed \power X | 
                                         (x,S') = NATIVEextSplit(S) \\\t3
           @ AUXreduce(f, f(x,y), S')
\end{axdef}



%%%%%%%%%%%%%%%%%%%%%%%%%%%%%%%%%%%%%%%%%%%%%%%%%%%%%%%%%%%%%%%%%%%%%%%%%%%%%
\section{Relations and Functions}


% domain and range

\begin{axdef}[X,Y]
   \dom == \lambda R: \assumed X \rel Y @ 
               \{x:\assumed X; y:\assumed Y | (x,y) \in R @ x\}
\end{axdef}

\begin{axdef}[X,Y]
   \ran == \lambda R: \assumed (X \rel Y) @ 
               \{x:\assumed X; y:\assumed Y | (x,y) \in R @ y\}
\end{axdef}

% inversion

%%postop \inv

\begin{axdef}[X,Y]
   \_ \inv == \lambda R: \assumed (X \rel Y) @ 
               \{x:\assumed X; y:\assumed Y | (x,y) \in R @ (y,x) \}
\end{axdef}

% image

\zfunction{95 (\_ \limg \_ \rimg)}

\begin{axdef}[X,Y]
   \_ \limg \_ \rimg == 
      \lambda R: \assumed X \rel Y; S: \assumed \power X @ \\\t1
        \{y: \assumed Y |
               \exists x: \assumed X @ x \in S \land (x,y) \in R\}
\end{axdef}


% identity

%%pregen \id

\begin{axdef}[X]
   \id \_ == \{x: \assumed X | x \in X @ (x,x)\}
\end{axdef}


% composition

\begin{verbatim}
%%inop \circ \comp 4
\end{verbatim}

\begin{axdef}[X,Y,Z]
   \_ \comp \_ == 
     \lambda R1: \assumed (X \rel Y); R2: \assumed (Y \rel Z) @ \\\t1
       \{x: \assumed X; y: \assumed Y; z: \assumed Z |
            (x,y) \in R1; (y,z) \in R2 @ (x,z)\}
\end{axdef}

\begin{axdef}[X,Y,Z]
   \_ \circ \_ ==
     \lambda R1: \assumed (Y \rel Z); R2: \assumed (X \rel Y) @
       R2 \comp R1
\end{axdef}

% iteration

\begin{axdef}[X]
  iter : \assumed \nat \fun (X \rel X) \fun (X \rel X) 
\where
  iter = \lambda n: \assumed \nat | n \in \nat @
           \lambda R: \assumed X \rel X @ \\\t1
               \IF n = 0 \THEN \id (\dom R \cup \ran R) \\\t1
               \ELSE R \comp iter(n-1)(R)
\end{axdef}

\zfunction{95 (\_ \bsup \_ \esup)} 

\begin{axdef}[X]
  \_ \bsup \_ \esup == \lambda R: \assumed X \rel X; n: \assumed \nat @ 
        iter n R
\end{axdef}

\begin{verbatim}
%%postop \plus
\end{verbatim}

\begin{axdef}[X]
 \_\plus : \assumed (X \rel X) \fun (X \rel X)
\where
  (\_\plus) = \lambda R: \assumed X \rel X @ 
     R \cup
     \{ x,z: \assumed X | \exists y: \assumed X @
                            (x,y) \in R \land (y,z) \in R\plus \}
\end{axdef}

\begin{verbatim}
%%postop \star
\end{verbatim}

\begin{axdef}[X]
 \_\star == 
   \lambda R: \assumed X \rel X @ \id (\dom R \cup \ran R) \cup R\plus 
\end{axdef}


% arrows

\begin{verbatim}
%%ingen \rel \pfun \fun \pinj \inj \psurj \surj \bij \ffun \finj
\end{verbatim}

\begin{axdef}[X,Y]
  \_ \rel \_ == \power \{x: X; y: Y\}
\end{axdef}

\begin{axdef}[X,Y]
  \_ \pfun \_ ==  \\\t1
    \{R: \assumed X \rel Y | 
         \forall x:X | x \in \dom R @ \exists_1 y:Y @ (x,y) \in R\}
\end{axdef}

\begin{axdef}[X,Y]
  \_ \fun \_ == \\\t1
    \{R: \assumed X \rel Y | 
         \forall x: \assumed X | x \in X @ \exists_1 y: \assumed Y | y \in Y
          @ (x,y) \in R\}
\end{axdef}

\begin{axdef}[X,Y]
  \_ \pinj \_ == \\\t1
    \{R: \assumed X \rel Y | 
         \forall y: \assumed Y | y \in \ran R @ 
                \exists_1 x: \assumed X | x \in X @ (x,y) \in R\}
\end{axdef}

\begin{axdef}[X,Y]
  \_ \psurj \_ == \\\t1
    \{R: \assumed X \rel Y | 
         \forall y: \assumed Y | y \in Y @ 
                \exists_1 x: \assumed X | x \in X @ (x,y) \in R\}
\end{axdef}

\begin{zed}
  X \surj Y == (X \psurj Y) \cap (X \fun Y) \\
  X \inj  Y == (X \pinj Y) \cap (X \fun Y) \\
  X \bij  Y == (X \inj Y) \cap (X \surj Y) \\
  X \ffun Y == (X \pfun Y) \cap \finset(X \cross Y) \\
  X \finj Y == (X \pinj Y) \cap (X \ffun Y)
\end{zed}


% Operations

\begin{verbatim}
%%inop \oplus 5
%%inop \dres \rres \ndres \nrres \wplus 6
%%prerel \disjoint 
%%inrel \partition 
\end{verbatim}

\begin{axdef}[X,Y]
  \_ \dres \_ == 
      \lambda S: \assumed \power X; R: \assumed (X \rel Y) @ \\\t1
        \{x: \assumed X; y: \assumed Y| x \in S \land (x,y) \in R\}
\end{axdef}

\begin{axdef}[X,Y]
  \_ \ndres \_ == 
      \lambda S: \assumed \power X; R: \assumed (X \rel Y) @ \\\t1
        \{x: \assumed X; y: \assumed Y| \lnot x \in S; (x,y) \in R\}
\end{axdef}

\begin{axdef}[X,Y]
  \_ \rres \_ == 
      \lambda R: \assumed (X \rel Y); S: \assumed \power Y @ \\\t1
        \{x: \assumed X; y: \assumed Y | y \in S \land (x,y) \in R\}
\end{axdef}

\begin{axdef}[X,Y]
  \_ \nrres \_ == 
      \lambda R: \assumed (X \rel Y); S: \assumed \power Y @ \\\t1
        \{x: \assumed X; y: \assumed Y | (x,y) \in R; \lnot y \in S\}
\end{axdef}

\begin{axdef}[X,Y]
  \_ \oplus \_ == 
      \lambda R1: \assumed X \rel Y; R2: \assumed X \rel Y @ \\\t1
        (\dom R2 \ndres R1) \cup R2
\end{axdef}

\begin{axdef}[X,Y]
   \disjoint \_: \power (X \pfun \power Y) 
\where
   (\disjoint \_) =
     \{F: \assumed X \pfun \power Y |
        \forall x1,x2: X | x1 \in \dom F; x2 \in \dom F; x1 \neq x2 @ \\\t9
           F~x1 \cap F~x2 = \emptyset\}
\end{axdef}
           
\begin{axdef}[X,Y]
   \_ \partition \_: (X \pfun \power Y) \rel \power Y
\where
   (\_ \partition \_) =
     \{F: \assumed X \pfun \power Y; S: \power Y |
        \disjoint F; \bigcup (\ran F) = S\}
\end{axdef}
           




%%%%%%%%%%%%%%%%%%%%%%%%%%%%%%%%%%%%%%%%%%%%%%%%%%%%%%%%%%%%%%%%%%%%%%%%%%%%%
\section{Numbers}

\begin{verbatim}
%%inop \upto 2
%%inop + -   3
%%inop * \div \mod  4
%%inrel < \leq \geq > 
\end{verbatim}

\zfunction{90 (- \_)}


\begin{verbatim}
%%zap native NATIVEmakeNumber(!) 
%%zap native NATIVEnumAdd(!,!) 
%%zap native NATIVEnumSub(!,!) 
%%zap native NATIVEnumMul(!,!) 
%%zap native NATIVEnumDiv(!,!) 
%%zap native NATIVEnumMod(!,!) 
%%zap native NATIVEnumLess(!,!)
%%zap native NATIVEnumLessEq(!,!)
\end{verbatim}

%%zap given \num \baseNum 
\begin{zed}
  [\baseNum] \\
  \num == \baseNum
\end{zed}

% declaration of natives
\begin{axdef}
  NATIVEnumAdd,NATIVEnumMul, NATIVEnumSub, \\
  NATIVEnumDiv, NATIVEnumMod
                    : \assumed \power((\num \cross \num) \cross \num) \\
  NATIVEnumLess,NATIVEnumLessEq : \assumed \power(\num \cross \num)
\end{axdef}

\begin{axdef}
   \nat == \{ x:\num | (0,x) \in NATIVEnumLessEq\}
\end{axdef}

\begin{axdef}
   \nat_1 == \{ x:\num | (0,x) \in NATIVEnumLess\}
\end{axdef}

\begin{axdef}
  - \_ == \lambda x:\num @ NATIVEnumSub(0,x)
\end{axdef}

\begin{axdef}
  \_ + \_ == \lambda x,y: \num @ NATIVEnumAdd(x,y)
\end{axdef}

\begin{axdef}
  \_ - \_ == \lambda x,y: \num @ NATIVEnumSub(x,y)
\end{axdef}

\begin{axdef}
  \_ * \_ == \lambda x,y: \num @ NATIVEnumMul(x,y) 
\end{axdef}

\begin{axdef}
  \_ \div \_ == \lambda x,y: \num @ NATIVEnumDiv(x,y) 
\end{axdef}

\begin{axdef}
  \_ \mod \_ == \lambda x,y: \num @ NATIVEnumMod(x,y) 
\end{axdef}

\begin{axdef}
  \_ < \_ == \{x,y: \num | (x,y) \in NATIVEnumLess \}
\end{axdef}

\begin{axdef}
  \_ \leq \_ == \{x,y: \num | (x,y) \in NATIVEnumLessEq \}
\end{axdef}

\begin{axdef}
  \_ > \_ == \{x,y: \num | (y,x) \in NATIVEnumLess \}
\end{axdef}

\begin{axdef}
  \_ \geq \_ == \{x,y: \num | (y,x) \in NATIVEnumLessEq \}
\end{axdef}

\begin{axdef}
  succ == \lambda x:\num @ x+1
\end{axdef}

\begin{axdef}
  \_ \upto \_ : \assumed \num \cross \num \fun \power \num
\where
  (\_ \upto \_) = \lambda l,u: \num @ 
     \IF l \leq u \THEN \{l\} \cup ( (l+1) \upto u )
                  \ELSE \emptyset
\end{axdef}

\begin{axdef}
  min == \lambda S:\power \num @ 
             \LET S' == \EXT S @
             (\mu x:\num | x \in S' \land
                           (\forall x':\num | x' \in S' @ x \leq x'))
\end{axdef}

\begin{axdef}
  max == \lambda S:\power \num @ 
             \LET S' == \EXT S @
              (\mu x:\num | x \in S' \land
                            (\forall x':\num | x' \in S' @ x \geq x'))
\end{axdef}

\begin{axdef}
  \abs == \lambda x:\num @ \IF x < 0 \THEN -x \ELSE x 
\end{axdef}



%%%%%%%%%%%%%%%%%%%%%%%%%%%%%%%%%%%%%%%%%%%%%%%%%%%%%%%%%%%%%%%%%%%%%%%%%%%%%%
\section{Sequences}

\begin{verbatim}
%%pregen \seq \seq_1 \iseq
%%inop \zscons 3
%%inop \cat 3
%%inop \filter \extract 4
%%inrel \prefix \suffix
\end{verbatim}

\zfunction{9999 (\langle ,, \rangle)}


% homomorphisms

\begin{axdef}
  NATIVEisSeq,NATIVEisNonEmptySeq,NATIVEasSeq,NATIVEsquash : NATIVEhomT
\end{axdef}

% primivites

\begin{verbatim}
%%zap native NATIVEisSeqRepr(?)
%%zap native NATIVEcons(?,?)
%%zap native NATIVEhead(!)
%%zap native NATIVEtail(!)
%%zap native NATIVEseqCard(?)
\end{verbatim}

\begin{axdef}[X]
  NATIVEisSeqRepr : \assumed \power X \\
  NATIVEcons : \assumed X \cross \seq X \fun \seq X \\
  NATIVEhead : \assumed \seq X \pfun X \\
  NATIVEtail : \assumed \seq X \pfun \seq X \\
  NATIVEseqCard : \assumed \power X \pfun \nat
\end{axdef}


% auxiliary to ensure a sequence representation

\begin{axdef}[X]
  AUXasSeq == 
    \lambda S: \assumed \seq X @ \\\t1
      \IF S \in NATIVEisSeqRepr \THEN S
      \ELSE NATIVEhom[\nat \cross X,\power(\nat \cross X)](NATIVEasSeq, S)
\end{axdef}

% types

\begin{axdef}[X]
  \seq \_ : \power (\num \fun X)
\where
  (\seq \_) = 
    \{S: \assumed(\nat \ffun X) | \\\t3
       S \in NATIVEisSeqRepr \lor
                NATIVEhom[\num \cross X,\zbool](NATIVEisSeq, S) = \ztrue \}
\end{axdef}

\begin{zed}
  \seq_1 X == 
    \{S: \assumed \seq X |  \\\t3
        NATIVEhom[\num \cross X,\zbool](NATIVEisNonEmptySeq, S) = \ztrue \}
\end{zed}

\begin{zed}
  \iseq X == \{s: \seq X| s \in \seq X; \# (\ran s) = \# s \}
\end{zed}


% denotation

\begin{axdef}[X]
  \langle ,, \rangle == AUXasSeq[X]
\end{axdef}

% concatentation


\begin{axdef}[X]
  \_ \zscons \_ == 
    \lambda x: \assumed X; xs: \assumed \seq X @ NATIVEcons(x,xs)
\end{axdef}


\begin{axdef}[X]
   \_ \cat \_ : \assumed \seq X \cross \seq X \fun \seq X
\where
   (\_\cat\_) = \\\t1
     \{ ys,ys': \assumed \seq X | ys' = AUXasSeq(ys) 
                                @ ((\emptyset, ys'),ys') \} \cup \\\t1
     \{ 
        x: \assumed X; xs,xs',ys,zs,zs': \assumed \seq X | \\\t1 ~ ~
        xs = NATIVEcons(x,xs'); ((xs',ys),zs) \in (\_\cat\_) \\
        zs' = NATIVEcons(x,zs) @
        ((xs,ys),zs')
     \}
\end{axdef}


% squash

\begin{axdef}[X]
  squash == 
    \lambda F: \assumed \nat \ffun X @ \\\t1
      \IF F \in NATIVEisSeqRepr \THEN F
      \ELSE NATIVEhom[\nat \cross X,\power(\nat \cross X)](NATIVEsquash, F)
\end{axdef}
      
% head, tail, last, front

\begin{axdef}[X]
  head == \{S,S': \assumed \seq X; x: \assumed X |
              S = NATIVEcons(x, S') @ (S,x)\}
\end{axdef}

\begin{axdef}[X]
  tail == \{S,S': \assumed \seq X; x: \assumed X |
              S = NATIVEcons(x, S') @ (S,S')\}
\end{axdef}

\begin{axdef}[X]
  rev : \assumed \seq X \fun \seq X
\where
  rev = \lambda S: \assumed \seq X @
             \IF S = \emptyset \THEN \emptyset
             \ELSE rev(tail(S)) \cat \langle head S \rangle
\end{axdef}

   % FIXME: efficiency
\begin{axdef}[X]
  last == \lambda S: \assumed \seq X @ head(rev S)
\end{axdef}

   % FIXME: efficiency
\begin{axdef}[X]
  front == \lambda S: \assumed \seq X @ rev(tail(rev S))
\end{axdef}

% fold

\begin{axdef}[X,Y]
  foldr : \assumed (X \cross Y \pfun Y) \cross Y  \fun \seq X \pfun Y
\where
  foldr = \lambda F: \assumed X \cross Y \pfun Y; y: Y @ \\\t1
           \lambda xs: \seq X @ INTERNfoldr(F, y, AUXasSeq(\DEEPFORCE(xs)))
\end{axdef}

\begin{axdef}[X,Y]
  foldl : \assumed Y \cross (Y \cross X \pfun Y)  \fun \seq X \pfun Y
\where
  foldl = \lambda y: Y; F: \assumed Y \cross X \pfun Y @ \\\t1
           \lambda xs: \seq X @ INTERNfoldl(y, F, AUXasSeq(\DEEPFORCE~xs))
\end{axdef}

\begin{axdef}[X,Y]
  INTERNfoldr : \assumed (X \cross Y \pfun Y) \cross Y \cross \seq X \pfun Y
\where
  INTERNfoldr =
    \lambda F: \assumed X \cross Y \pfun Y; y: Y; xs: \assumed \seq X @ \\\t1
      \IF xs = \emptyset \THEN y
                         \ELSE F(head~xs,INTERNfoldr(F,y,tail~xs))
\end{axdef}

\begin{axdef}[X,Y]
  INTERNfoldl : \assumed Y \cross (Y \cross X \pfun Y) \cross \seq X \pfun Y
\where
  INTERNfoldl =
    \lambda y: Y; F: \assumed Y \cross X \pfun Y; xs: \assumed \seq X @ \\\t1
      \IF xs = \emptyset \THEN y
                         \ELSE INTERNfoldl(F(y,head~xs),F,tail~xs)
\end{axdef}

\begin{axdef}[X]
  fold : \assumed (X \cross X \pfun X) \fun \seq X \pfun X
\where
  fold =
    \lambda F: \assumed X \cross X \pfun X @
    \lambda xs: \assumed \seq X @ \\\t1
      INTERNfoldl(head~xs,F,tail~xs)
\end{axdef}


% filter and extract

\begin{axdef}[X]
   \_ \filter \_ : \assumed \seq X \cross \power X \fun \seq X 
\where
   (\_ \filter \_) = \lambda S: \assumed \seq X; R: \assumed \power X @
     squash (S \rres R)
\end{axdef}
   
\begin{axdef}[X]
   \_ \extract \_ == 
      \lambda D: \assumed \power \nat; S: \assumed \seq X @
        squash (D \dres S)
\end{axdef}

\begin{axdef}[X]
   \_ \prefix \_ : \assumed \seq X \rel \seq X
\where
  (\_ \prefix \_) = \\\t1
     \{ S2: \assumed \seq X @ (\emptyset, S2) \} \cup \\\t1
     \{ S1,S2: \assumed \seq X; x: \assumed X | \\\t2
         S1 \prefix S2 @ (NATIVEcons(x,S1),NATIVEcons(x, S2)) \}
\end{axdef}

\begin{axdef}[X]
   \_ \suffix \_ : \assumed \seq X \rel \seq X
\where
  (\_ \suffix \_) = \\\t1
     \{ S2: \assumed \seq X @ (\emptyset, S2) \} \cup \\\t1
     \{ S1,S1',S2,S2': \assumed \seq X; x: \assumed X | \\\t2
         ((S1',\langle x \rangle),S1) \in (\_ \cat \_); 
         ((S2',\langle x \rangle),S2) \in (\_ \cat \_); 
         S1' \suffix S2' @
        (S1, S2) \}
\end{axdef}

\begin{axdef}[X]
   \dcat: \assumed \seq (\seq X) \fun \seq X 
\where
   \dcat = \lambda SS: \assumed \seq(\seq X) @
            foldr((\_\cat\_),\emptyset)(SS)
\end{axdef}


%%%%%%%%%%%%%%%%%%%%%%%%%%%%%%%%%%%%%%%%%%%%%%%%%%%%%%%%%%%%%%%%%%%%%%%%%%%%%%
\section{Cardinality (optimized for sequences)}

\begin{axdef}[X]
   \# == \lambda S: \assumed \finset X @ \\\t1
             \IF S \in NATIVEisSeqRepr \THEN NATIVEseqCard S \\\t1
             \ELSE NATIVEhom[X,\baseNum](NATIVEcard, S)
\end{axdef}




%%%%%%%%%%%%%%%%%%%%%%%%%%%%%%%%%%%%%%%%%%%%%%%%%%%%%%%%%%%%%%%%%%%%%%%%%%%%%%
\section{Denotations}

\begin{verbatim}
%%zap native NATIVEencode(!) 
%%zap native NATIVEdecode(!) 
%%zap given \denotation
\end{verbatim}

\begin{zed}
  [\denotation]
\end{zed}

% declaration of natives

\begin{axdef}
  NATIVEencode: \assumed \seq \nat \pfun \denotation \\
  NATIVEdecode: \assumed \denotation \fun \seq \nat
\end{axdef}

% decode/encode

\begin{axdef}
  \ddec == \lambda x: \assumed \denotation @ NATIVEdecode(x)
\end{axdef}

\begin{axdef}
  \denc == 
    \lambda x: \assumed \seq \nat @ NATIVEencode(AUXasSeq~x)
\end{axdef}


%%%%%%%%%%%%%%%%%%%%%%%%%%%%%%%%%%%%%%%%%%%%%%%%%%%%%%%%%%%%%%%%%%%%%%%%%%%%%
\section{Reflection}

\begin{zed}
  TYPEINFO[A] ::= TypeInfo \ldata \denotation \rdata \\
  GIVENTYPEINFO[A] ::= GivenTypeInfo \ldata \denotation \rdata
\end{zed}

\zgeneric{10 (\typeinfo \_)}
\zgeneric{10 (\giventypeinfo \_)}

\begin{axdef}[A]
  \typeinfo \_ : TYPEINFO[A] \\
\end{axdef}

\begin{axdef}[A]
  \giventypeinfo \_ : GIVENTYPEINFO[A]
\end{axdef}

\begin{verbatim}
%%zap native NATIVEgiventypeinfo(!,!) 
\end{verbatim}

\begin{axdef}
  NATIVEgiventypeinfo: \assumed \denotation \cross \denotation \fun \denotation
\end{axdef}


%%%%%%%%%%%%%%%%%%%%%%%%%%%%%%%%%%%%%%%%%%%%%%%%%%%%%%%%%%%%%%%%%%%%%%%%%%%%%
\section{Data Input}


\begin{verbatim}
%%zap native NATIVEfromFile(!,!) 
%%zap native NATIVEfromDeno(!,!) 
\end{verbatim}

\begin{axdef}[A]
  NATIVEfromFile: \assumed TYPEINFO[A] \cross \denotation \fun \seq A
\end{axdef}

\begin{axdef}[A]
  NATIVEfromDeno: \assumed TYPEINFO[A] \cross \denotation \fun \seq A
\end{axdef}

\zfunction{9 (\_ \fromfile \_)}
\zfunction{9 (\_ \fromdeno \_)}

\begin{axdef}[A]
  \_ \fromfile \_ == \lambda i: TYPEINFO[A]; f: \denotation @ NATIVEfromFile(i,f)
\end{axdef}

\begin{axdef}[A]
  \_ \fromdeno \_ == \lambda i: TYPEINFO[A]; f: \denotation @ NATIVEfromDeno(i,f)
\end{axdef}


\end{document}

%%% Local Variables: 
%%% mode: latex 
%%% TeX-master: t
%%% End: 
