
%% SECTION
\section{Introduction} \label{sec:intro}

This document is based on the OMG CORBA\index{CORBA} Specification version
2.4~\cite{omg:CORBA:2.4:2000} and the OMG Security Service Secification, version
1.5~\cite{omg:Security:1.5:2000}.  Although there are
newer specifications available by the OMG, we stay with the ones mentioned above
because:
\begin{itemize}
\item we already formalized great parts of it, 
\item we have it printed out, and
\item most current open source implementations (JacORB, MICO) are based on the
  older ones (Tao uses the new one, but the security service implementation is
  very partial\dots).
\end{itemize}

In the following, we will use the term ``\corbasec'' to refer to the OMG
specification of the CORBA Security Service, v1.5~\cite{omg:Security:1.5:2000}.


\subsection{Access to Security Service objects}

The following two quotes from the CORBA-specifcation and the Security Service
specification should motivate that other CORBA-objects (i.e. in another thread
of execution) can only access very limited information and are especially unable
to alter directly data of the current thread.

\begin{quote}
  Operations on interfaces derived from \texttt{Current} access state associated
  with the thread in which they are invoked, not state associated with the
  thread from which the Current was obtained. This prevents one thread from
  manipulating another thread's state, and avoids the need to obtain and narrow
  a new Current in each method's thread context.  Current objects must not be
  exported to other processes, or externalized with
  \texttt{ORB::object\_to\_string}. If any attempt is made to do so, the
  offending operation will raise a \texttt{MARSHAL} system exception.  Currents
  are per-process singleton objects, so no destroy operation is needed.
  \hspace{3ex} \emph{(4.7 Current Object, p.4--31, Common Object Request Broker
    Architecture (CORBA)~\cite{omg:CORBA:2.4:2000})}
\end{quote}

\begin{quote}
  Certain interfaces are identified as ``Locality Constrained''. These
  interfaces are intended to be accessible only from within the context (e.g.,
  process or RM-ODP capsule) in which they are instantiated (i.e., from within
  the protection boundary around that context). No object reference to these
  interfaces can therefore be passed meaningfully outside of that context. The
  exact details of how this protection boundary is implemented is an
  implementation detail that the implementor of the service will need to provide
  in order to establish that the implementation is secure. Locality Constrained
  objects may not be accessible through the DII/DSI facilities, and they may not
  appear in the Interface Repository. Any attempt to pass a reference to a
  locality constrained object outside its locality, or any attempt to
  externalize it using \texttt{ORB::object\_to\_string} will result in the
  raising of the \texttt{CORBA::NO\_MARSHAL} exception.  \hspace{3ex}
  \emph{(2.2.1.5 Object System Implementor's View, p.2--31, Security
    Service~\cite{omg:Security:1.5:2000})}
\end{quote}


\subsection{Items not included in the OMG Specification}
\label{sec:not-included}

The administration of principals (i.e. the creation and destruction of
principals and the mapping of attributes to principals) and the management of
policy domains and moving objects between domains is out of scope of the
specification,


\subsection{Static features}

The following features are fixed at runtime: Rights, security attribute types,
interfaces (i.e. we do not use the DII).

\note{The RequiredRights object can be changed at runtime by principals that
  have the right permissions.}


\subsection{Relevant interfaces}

Question: which interfaces of the security service are subject to access
control? \\
Only those that can be accessed by objects, and not the ones that are internal
to the security service.  An object gets access to interfaces of the security
service through a call to
\texttt{resolve\_initial\_references(``SecurityCurrent'')} and all interfaces
that are accessible through this, and so on.  In short, these are all the
interfaces of the SecurityLevel2 and Administration modules.  Although, it does
not make a lot sense to restrict access to these operations.  The security
depends on the parameters of these operations.  This question is only important
for the interfaces that give access to the internal security data, like
\texttt{RequiredRights} or the domain policy objects or the underlying security
technologies, i.e. the administration of principals.

\begin{itemize}
\item Access control needed:
  \begin{itemize}
  \item RequiredRights
  \item all kinds of Policies
  \item Plus administrational interfaces not defined by the OMG specification.
  \end{itemize}
\item No access control needed:
  \begin{itemize}
  \item Vault
  \item SecurityContext
  \end{itemize}
\item Check of parameters needed:
  \begin{itemize}
  \item PrincipalAuthenticator
  \item AccessDecision
  \item Rights
  \item Current
  \item Credentials
  \end{itemize}
\end{itemize}



%%% Local Variables: 
%%% mode: latex
%%% TeX-master: "model"
%%% End: 
