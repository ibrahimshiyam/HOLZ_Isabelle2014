
%% SECTION
\section{Transformation From IDL To Z} \label{sec:oo}

In this section we will describe our approach to model the object oriented
features of the OMG \corbasec specification in Z.

\begin{itemize}
\item Naming convention: try to make most names unique (using prefixes) to
  eliminate the problem of name clashes.
\item sometimes we introduce object ids, sometimes not...
\item 
\end{itemize}

When an interface is only used to organize the namespace and don't represent
data ($PrinciplaAuthenticator$ for example, see
section~\ref{sec:PrincipalAuthenticator}) then we don't model this interface but
only the operations it provides.  If at any time only one instance of an
interface may be present in the system then we model this interface as a scheme
of the same name and add this schema to the system state.  The operations are
defined directly on this schema ($RequiredRights$ for example, see
section~\ref{sec:RequiredRights}).

Other methods to model object-orientation in Z are presented
in~\cite{stepney:object:1990,smith:object:1999}.


\subsection{Object interfaces}

\zsection{oo-in-z}

For some interfaces we specify some kind of object system.  This is done when
many objects of this interface might be created within the system.  Examples are
Credentials or Current.  In the following we will sketch the process to specify
such an object model for current objects:

We introduce a basic (or given) type $CURRENT$ which denotes the object ids:
\begin{zed}
  [CURRENT]
\end{zed}

Then we define a schema $Current$ that defines the internal data of objects of
this type and a schema $CurrentS$ which defines the state of all objects.
\begin{zedgroup}
  \begin{schema}{Current}
    \dots \\
    \where
    \dots \\
  \end{schema}
  \begin{schema}{CurrentS}
    currentId: CURRENT \pfun Current \\
    \dots \\
    \where
    \dots \\
  \end{schema}
\end{zedgroup}
The partial function $currentId$ assignes ids to objects and is always used to
access objects.

Operations on current objects are modeled as operation schemas that operate on
the $CurrentS$ state schema and always have an additional component $id?:
CURRENT$.  The following example is for an operation \texttt{set\_credentials}.
Note that the names are prefixed to prevent name clashes:
\begin{schema}{CUsetCredentials}
  \Delta CurrentS \\
  id?: CURRENT \\
  \dots \\
  \where
  \dots \\
\end{schema}
\begin{axdef}
  CU\_setCredentials: CREDS \cross Current \fun Current
\end{axdef}
The final operation will be build from different operation schemas handling
success, failure and system exceptions and will be named like the IDL operation
prefixed with the initials of the interface name:
\begin{zed}
  CU\_set\_credentials == (CUsetCredentials \land NoEx) \lor SysEx \\
  \t6 \lor (CUsomeError \land someEx) \lor \dots
\end{zed}

We do not model a $this$ or $self$ component but take care when we define
operations that an id always refers to the same object.


\subsection{Inheritance}

Inheritance is important for the definition of policies in our model.  The basic
scheme is the same as for the Current interface, but we introduce a function
$Ppolicy$ which assigns a policy type to each policy id and instead of onw
function mapping ids to objects we define one for each policy type:
\begin{zed}
  [POLICY]
\end{zed}
\begin{zedgroup}
  \begin{schema}{DelegationPolicy}
    \dots \\
  \end{schema}
  \begin{schema}{PolicyS}
    Ppolicy: POLICY \pfun PolicyType \\
    \dots \\
    PdelegationPolicy: POLICY \pfun DelegationPolicy \\
    \dots \\
  \end{schema}
\end{zedgroup}

Operations on policies have the additional component $id?: POLICY$ as for
operations on Currents above, which can also be used to model some kind of
polymorphism: The predicate part of the operation schema must define predicates
for each policy type and check the type:
\begin{schema}{Pcopy}
  \Delta PolicyS \\
  id?: POLICY \\
  \dots \\
  \where
  \dots \\
  id? \in \dom PdelegationPolicy \implies \dots \\
  \dots \\
\end{schema}

The total operation is assembled like above with error cases and exceptions.


