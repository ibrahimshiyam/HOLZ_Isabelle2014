
%%%%%%%%%%%%%%%%%%%%%%%%%%%%%%%%%%%%%%%%%%%%%%%%%%%%%%%%%%%%%%%%%%%%%%%%%%%%%%%%%
%% SECTION
%%%%%%%%%%%%%%%%%%%%%%%%%%%%%%%%%%%%%%%%%%%%%%%%%%%%%%%%%%%%%%%%%%%%%%%%%%%%%%%%%
\section{The Complete ORB Data Model} \label{sec:ORB-model}

\zsection[ZSrbacAdmin]{ZSspecification}\vspace{2ex}

\subsection{The system state}
In this section, we define the parts that connect the whole specification.
\begin{schema}{ORB}
  InterfaceRepository \\
  ImplementationRepository \\
  PolicyS \\
  Users \\
  Domains \\
  ObjectS \\
  CredentialsS \\
  CurrentS \\
  RequiredRightsS \\
  \where
  \ran IRimplements = \dom IRinterface \\
  \bigcup (\ran Ddomain) = \dom OobjectId \\
  \forall o: \ran OobjectId; p,q: POLICY | p \in o.OoverridePolicies \land \\
  \t1 q \in o.OoverridePolicies @ \\
  \t2 Ppolicy(p) \in overridesAllowed \land (Ppolicy(p) = Ppolicy(q) \implies p
  = q) \\

  \forall d: \ran DmanagerId @ \ran d.DMpolicy \subseteq \dom Ppolicy \\
\end{schema}
The meaning of the predicates:
\begin{enumerate}
\item All objects in the implementation repositroy must be legal objects, i.e.\
  must be registered with the object system.
\item Every object is at least member of one domain.
\item The policies in $Object::OoverridePolicies$ must be of different types.
\end{enumerate}


\subsection{Policies}
Now, we define the access policies for the internal operations that should be
orthogonal to any application-level access control policies.  For this, we
introduce two groups: ``GroupId:user'' and ``GroupId:admin''.  Every normal user
is in group $user$ and administrators are in the group $admin$.  Note, that a
``normal'' user is a user that is registered and authenticated.  An
unauthenticated user does not get the same privileges (only $Public$).  It is
the administrators responsibility to set the attributes of new users
appropriately, there is no automatic mechanism.  Additionally, we define an
access domain $Internal$ for all internal objects, like $RequiredRights$.

In the last section, we defined a \rbaci{} model for \corbasec.  The
administration operations of this model will be collected in the $RbacAdmin$
interface. 
\begin{table}[ht]
  \begin{center}
    \begin{tabular}{l|l|l}
      Interface & Operation & Rights \\ \hline
      RequiredRights & get\_required\_rights & g \\
      & set\_required\_rights & sm \\ \hline
      UserAdmin & createPrincipal & sm \\
      & deletePrincipal & m \\
      & changeAuthData & sm \\
      & setAttributes & sm \\ \hline
      RbacAdmin & assignUser & sm \\
      & deassignUser & m \\
      & grantPermission & sm \\
      & revokePermission & sm \\ \hline
      DomainAdmin & createDomain & sm \\
      & deleteDomain & m \\
      & insertObject & sm \\
      & removeObject & m \\
    \end{tabular}
    \caption{Required rights for internal operations}
  \end{center}
\end{table}
\begin{table}[ht]
  \begin{center}
    \begin{tabular}{l|l}
      Attribute & Rights \\ \hline
      GroupId:user & g \\ \hline
      GroupId:admin & gsm \\
    \end{tabular}
    \caption{Granted rights in domain $Internal$}
  \end{center}
\end{table}
The domain access policy of this domain must return these granted rights.
\begin{axdef}
  group\_user, group\_admin: SecAttribute \\
  internalDomainPolicy: DomainAccessPolicy \\
  internalDomainPOLICY: POLICY \\
  \where
  internalDomainPolicy.Ppolicy\_type \in \{ SecClientInvocationAccess, \\
  \t1 SecTargetInvocationAccess \} \\
  internalDomainPolicy.PgrantedRights = \\
  \t1 \{ (group\_user, SecInitiator) \mapsto \{ get \}, \\
  \t1 (group\_admin, SecInitiator) \mapsto \{ get,set,manage \} \} \\
\end{axdef}
To do: connect this policy with the domain $Internal$ and verify that a call to
granted\_rights really displays these rights.  Define the required rights
object.  Axiomatically or with the proper operations?


\subsection{Initialization of data} \label{sec:init-data}
The most crucial part here is to develop mechanisms that enable one to specify
the internal data independend of and extendible with external application
interfaces and data.


\subsubsection{Security attributes}
The actual security attributes must be defined here, e.g. setting their family
etc.

\subsubsection{State schemas}
In order to initialize the $InterfaceRepository$ and $ImplementationRepository$
schemas, we must identify the interfaces that are subject to access control.

As stated earlier in section~\ref{sec:intro}, all interfaces that are locality
constrained cannot be exported from the current thread of execution and should
also always be allowed to the principal of the current thread.  The only
security checks must be on the parameters of the operations not on the
operations themselves.  This must be handled within the operations.

We must also define the rights and policies for these operations.

The following interfaces are not locality constrained: RequiredRights,
AccessPolicy, DomainAccessPolicy, AuditPolicy, SecureInvocationPolicy,
DelegationPolicy.

In the following, we will focus on the interfaces that are part of the access
control scheme, namely RequiredRights and DomainAccessPolicy. 

\begin{axdef}
  \_RR\_get\_required\_rights: Identifier \\
  \_RR\_set\_required\_rights: Identifier \\
  \_U\_createPrincipal: Identifier \\
  \_U\_deletePrincipal: Identifier \\
  \_U\_changeAuthData: Identifier \\
  \_U\_setAttributes: Identifier \\
  \_D\_createDomain: Identifier \\
  \_D\_deleteDomain: Identifier \\
  \_D\_insertObject: Identifier \\
  \_D\_removeObject: Identifier \\
%  : Identifier \\
  \_RequiredRights: RepositoryId \\
  \_UserAdmin: RepositoryId \\
  \_DomainAdmin: RepositoryId \\
%  : RepositoryId \\
  \_RequiredRights\_Impl: IMPLEMENTATION \\
  \_UserAdmin\_Impl: IMPLEMENTATION \\
  \_DomainAdmin\_Impl: IMPLEMENTATION \\
  \_RequiredRights\_CURRENT: CURRENT \\
  \_UserAdmin\_CURRENT: CURRENT \\
  \_DomainAdmin\_CURRENT: CURRENT \\
  \_RequiredRights\_Current: Current \\
  \_UserAdmin\_Current: Current \\
  \_DomainAdmin\_Current: Current \\
  admin: SecurityName \\
  empty: Opaque \\
  internalDomain: DomainManager \\
  internalDOMAIN: DOMAIN\_MANAGER \\
\end{axdef}

We initialize the interface repository with the interfaces of our security
infrastructure that are visible outside the ORB and are accessible from
every client by every user.
\begin{schema}{InitInterfaceRepository}
  InterfaceRepository \\
  \where
  IRinterface = \{ \< \_RequiredRights \mapsto \{\_RR\_get\_required\_rights, \\
  \t1 \_RR\_set\_required\_rights \}, \\ 
  \_UserAdmin \mapsto \{ \_U\_createPrincipal, \_U\_deletePrincipal, \\
  \t1 \_U\_changeAuthData, \_U\_setAttributes \}, \\
  \_DomainAdmin \mapsto \{ \_D\_createDomain, \_D\_deleteDomain, \\
  \t1 \_D\_insertObject, \_D\_removeObject \} \} \> \\ 
  % IRname
  IRparents = \{ \< \_RequiredRights \mapsto \langle \_RequiredRights \rangle,\\
   \_UserAdmin \mapsto \langle \_UserAdmin \rangle, \\
   \_DomainAdmin \mapsto \langle \_DomainAdmin \rangle \} \>
\end{schema}
We define an implementation for these three internal objects and we define
current objects for them.  Note, that yet no reference was created to them, and
hence the domains and the object schema are still empty.
\begin{schema}{InitImplementationRepository}
  ImplementationRepository \\
  \where
  IRimplements = \{ \< \_RequiredRights\_CURRENT \mapsto \_RequiredRights, \\
  \_UserAdmin\_CURRENT \mapsto \_UserAdmin, \\
  \_DomainAdmin\_CURRENT \mapsto \_DomainAdmin \} \> \\
  IRinstance = \{ \< \_RequiredRights\_CURRENT \mapsto \_RequiredRights\_Impl, \\
  \_UserAdmin\_CURRENT \mapsto \_UserAdmin\_Impl, \\
  \_DomainAdmin\_CURRENT \mapsto \_DomainAdmin\_Impl \} \>
\end{schema}
We create an internal policy domain in which the references for the internal
objects will be inserted.
\begin{schema}{InitPolicyS}
  PolicyS \\
  \where
  Ppolicy = \{ internalDomainPOLICY \mapsto SecTargetInvocationAccess \} \\
  PdomainAccessPolicy = \{ internalDomainPOLICY \mapsto internalDomainPolicy \} \\
  PinvocationCredentialsPolicy = \emptyset \\
\end{schema}
Initially only the security administrator exists, but no other users.
\begin{schema}{InitUsers}
  Users \\
  \where
  Uprincipals = \{ admin \} \\
  Uauthenticates = \{ admin \mapsto empty \} \\
  Uattributes = \{ admin \mapsto \{ group\_admin \} \} \\
%  Uprivileges = \{ \} \\
\end{schema}
\begin{schema}{InitDomains}
  Domains \\
  \where
  DmanagerId = \{ internalDOMAIN \mapsto internalDomain \} \\
  Ddomain = \{ internalDOMAIN \mapsto \emptyset \} \\
\end{schema}
\begin{schema}{InitObjectS}
  ObjectS \\
  \where
  OobjectId = \emptyset \\
  Otarget = \emptyset \\
\end{schema}
\begin{schema}{InitCredentialsS}
  CredentialsS \\
  \where
  CcredentialsId = \emptyset \\
  CreceivedCredentialsId = \emptyset \\
  CisValid = \emptyset \\
  CcredentialsType = \emptyset \\
\end{schema}
\begin{schema}{InitCurrentS}
  CurrentS \\
  \where
  CUcurrentId = \{ \< \_RequiredRights\_CURRENT \mapsto \_RequiredRights\_Current, \\
  \_UserAdmin\_CURRENT \mapsto \_UserAdmin\_Current, \\
  \_DomainAdmin\_CURRENT \mapsto \_DomainAdmin\_Current \} \> \\
  CUcurrentRole = \{ \< \_RequiredRights\_CURRENT \mapsto TargetCurrent, \\
  \_UserAdmin\_CURRENT \mapsto TargetCurrent, \\
  \_DomainAdmin\_CURRENT \mapsto TargetCurrent \} \>
\end{schema}
The required rights for the three internal interfaces must be defined:
\begin{schema}{InitRequiredRights}
  RequiredRightsS \\
  \where
  requiredRights =  \\
  \t1 \{ \< (\_RR\_get\_required\_rights,\_RequiredRights) \mapsto (\langle get
  \rangle, SecAllRights), \\ 
  (\_RR\_set\_required\_rights, \_RequiredRights) \mapsto (\langle set, manage
  \rangle, SecAllRights), \\ 
  (\_U\_createPrincipal, \_UserAdmin) \mapsto (\langle set, manage \rangle,
  SecAllRights), \\ 
  (\_U\_deletePrincipal, \_UserAdmin) \mapsto (\langle manage \rangle,
  SecAllRights), \\ 
  (\_U\_changeAuthData, \_UserAdmin) \mapsto (\langle set, manage \rangle,
  SecAllRights), \\ 
  (\_U\_setAttributes, \_UserAdmin) \mapsto (\langle set, manage \rangle,
  SecAllRights), \\ 
  (\_D\_createDomain, \_DomainAdmin) \mapsto (\langle set, manage \rangle,
  SecAllRights), \\ 
  (\_D\_deleteDomain, \_DomainAdmin) \mapsto (\langle manage \rangle,
  SecAllRights), \\ 
  (\_D\_insertObject, \_DomainAdmin) \mapsto (\langle set, manage \rangle,
  SecAllRights), \\ 
  (\_D\_removeObject, \_DomainAdmin) \mapsto (\langle manage \rangle,
  SecAllRights) \} \> 
\end{schema}


\subsubsection{Operations}
Now, the operations of the internal interfaces must be defined in terms of an
operation invocation of CORBA and enhanced with access control.

\paragraph{RequiredRights}
\begin{zedgroup}
\begin{schema}{I\_RR\_get\_required\_rights}
  RR\_get\_required\_rights \\
  operation?: Identifier \\
  \where
  operation? = \_RR\_get\_required\_rights \\
\end{schema}
\begin{schema}{I\_RR\_set\_required\_rights}
  RR\_set\_required\_rights \\
  operation?: Identifier \\
  \where
  operation? = \_RR\_set\_required\_rights \\
\end{schema}
\end{zedgroup}


\paragraph{DomainAccessPolicy}


\paragraph{User administration}
\begin{zedgroup}
\begin{schema}{I\_U\_createPrincipal}
  createPrincipal \\
  operation?: Identifier \\
  \where
  operation? = \_U\_createPrincipal \\
\end{schema}
\begin{schema}{I\_U\_deletePrincipal}
  deletePrincipal \\
  operation?: Identifier \\
  \where
  operation? = \_U\_deletePrincipal \\
\end{schema}
\begin{schema}{I\_U\_changeAuthData}
  changeAuthData \\
  operation?: Identifier \\
  \where
  operation? = \_U\_changeAuthData \\
\end{schema}
\begin{schema}{I\_U\_setAttributes}
  setAttributes \\
  operation?: Identifier \\
  \where
  operation? = \_U\_setAttributes \\
\end{schema}
\end{zedgroup}


\paragraph{Domain administration}
\begin{zedgroup}
\begin{schema}{I\_D\_createDomain}
  createDomain \\
  operation?: Identifier \\
  \where
  operation? = \_D\_createDomain \\
\end{schema}
\begin{schema}{I\_D\_deleteDomain}
  deleteDomain \\
  operation?: Identifier \\
  \where
  operation? = \_D\_deleteDomain \\
\end{schema}
\begin{schema}{I\_D\_insertObject}
  insertObject \\
  operation?: Identifier \\
  \where
  operation? = \_D\_insertObject \\
\end{schema}
\begin{schema}{I\_D\_removeObject}
  removeObject \\
  operation?: Identifier \\
  \where
  operation? = \_D\_removeObject \\
\end{schema}
\end{zedgroup}


\paragraph{Internal operations}
The internal operations are all operations together with access control:
\begin{zed}
  internalOps == AccessDecision \land \\
  \t1 ( \< I\_RR\_get\_required\_rights \lor \\
  I\_RR\_set\_required\_rights \lor \\
  I\_U\_createPrincipal \lor \\
  I\_U\_deletePrincipal \lor \\
  I\_U\_changeAuthData \lor \\
  I\_U\_setAttributes \lor \\
  I\_D\_createDomain \lor \\
  I\_D\_deleteDomain \lor \\
  I\_D\_insertObject \lor \\
  I\_D\_removeObject ) \>
\end{zed}


%%% Local Variables: 
%%% mode: latex
%%% TeX-master: "model"
%%% End: 
