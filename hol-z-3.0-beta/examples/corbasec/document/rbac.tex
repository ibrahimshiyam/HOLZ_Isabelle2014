
%% SECTION
\section{Introducing RBAC into \corbasec} \label{sec:rbac}

%\zsection[LTX:ZSaccessDecision#specification]{ZSrbac}\vspace{2ex}
\zsection[ZSaccessDecision]{ZSrbac}\vspace{2ex}

\noindent
In this section, we introduce role-based access control into \corbasec.


\subsection{Mapping \rbaci{} to \corbasec}

For this mapping, we follow closely the approach
of~\cite{beznosov.ea:framework:1999}.

\subsubsection{The \rbaci{} Model}
As a reference for \rbaci, we use the definitions given
in~\cite{sandhu.ea:RBAC:1996}:
\begin{itemize}
\item Users $U$; roles $R$; permissions $P$; and sessions $S$.
\item $PA \subseteq P \cross R$: Permission to role assignment.
\item $UA \subseteq U \cross R$: User to role assignment.
\item $RH \subseteq R \cross R$: Partial order on roles, called the role
  hierarchy.
\item $user: S \fun U$: A function, mapping a session to a single user.
\item $roles: S \fun \power R$: A function, mapping each session to a set of
  roles (with some constraints).
\end{itemize}
In order to define a refinement relation, we give a definition of \rbaci{} as a
Z schema:
\begin{zed}
  [ USER, ROLE, PERMISSION, SESSION ]
\end{zed}
\begin{schema}{RBAC1}
  U: \power USER; R: \power ROLE; P: \power PERMISSION; S: \power SESSION \\
  UA: USER \rel ROLE; RH : ROLE \rel ROLE \\
  PA: PERMISSION \rel ROLE \\
  user: SESSION \pfun USER; roles: SESSION \pfun \power ROLE \\
  \where
  user \in S \fun U \land roles \in S \fun \power R \land UA \in U \rel R \land
  PA \in P \rel R \\ 
  RH \in R \rel R \land \id R \subseteq RH \land RH \comp RH \subseteq RH \land
  RH \cap RH\inv \subseteq \id R \\ 
\end{schema}

The access control for \rbaci{} can then be defined as a schema
$checkPermission$, which checks whether a session has certain permissions:
\begin{schema}{checkPermission}
  RBAC1 \\
  session?: SESSION \\
  permission?: PERMISSION \\
  \where
  session? \in S \\
  permission? \in ( RH \comp PA\inv ) \limg roles(session?) \rimg \\
\end{schema}
%In the context of \corbasec, roles are defined as security attributes of type
%$Role$ and permissions are defined through rights (for now, the standard rights
%of \corbasec{}).  Sessions can be identied with principals, although this is not
%so straight forward, as will be explained in an instance.  The relations and
%functions are then defined accordingly as follows:
%\begin{itemize}
%\item Users $U$; attributes of type roles $A$; rights $R$; and principals $P$.
%\item $PA \subseteq R \cross A$: Granted rights to security attributes of type
%  role.
%\item $UA \subseteq U \cross A$: User to security attributes of type role.
%\item $RH \subseteq A \cross A$: Partial order on $A$, called the role hierarchy.
%\item $user: P \fun U$: Mapping a principal to a single user.
%\item $roles: P \fun \power A$: A function mapping each principal to a set of
%  privilege attributes.
%\end{itemize}

\subsection{Refining \rbaci{} to \corbasec}

In our concrete mapping, we deviate from the mapping, which was given
in~\cite{beznosov.ea:framework:1999} in two aspects:
\begin{itemize}
\item Beznosov makes no distinction between interface instances and references
  to such interfaces.  Actually, there could be several references to the same
  instance and these references could be in different policy domains.  Initially
  all references to an object will be in the same domain, but given operations
  for administration of domains, these references could be moved into different
  domains.  We set rules for object references, not only for interfaces.
\item We do not identify sessions with principals but with the current object
  that was created for a specific client, and resembles the runtime environment
  of that client.
\end{itemize}
The following enumeration defines our mapping of the components of \rbaci{} to
components in \corbasec:
\begin{itemize}
\item Users $U$ are users (i.e. security names) in \corbasec.  They are
  represented by the set $Users.Uprincipals$.  We define a function $urbac$ that
  maps each possible user of \rbaci{} to a user name in \corbasec.
  \begin{axdef}
    urbac: USER \bij SecurityName \\
  \end{axdef}
\item Roles $R$ are security attributes of type $Role$.  The function $rrbac$
  assigns a security attribute of type $Role$ to each role of \rbaci:
  \begin{axdef}
    rrbac: ROLE \bij SecAttribute \\    
  \end{axdef}
\item Permissions could be simply represented by rights if we restrict our
  \corba -model to only one domain.  Then the relation $PA$ would correspond to
  the $GrantedRights$-table of this domain.  Since we want to be more general
  and allow for different domains, we must encode permissions depending on
  domains.
  
  Permissions $P$ must be encoded such that a permission incorporates whether an
  operation of some interface may be invoked in some domain.  The functions
  $getOp$ and $getPolicy$ are only accessor functions for the triple $Perm$.
  \begin{zed}
    Perm == Identifier \cross RepositoryId \cross DomainAccessPolicy \\
  \end{zed}
  \begin{axdef}
    prbac: PERMISSION \fun Perm \\
    getOp: Perm \fun (Identifier \cross RepositoryId) \\
    getPolicy: Perm \fun DomainAccessPolicy \\
  \end{axdef}
  
\item Sessions $S$ are represented by the Current object that is created for a
  client or a target object.  This resembles the runtime environment, or thread
  in OMG terms and seems similar to the session concept of \rbaci.  The function
  $srbac$ defines a bijection between sessions and Current objects.
  \begin{axdef}
    srbac: SESSION \bij Current \\    
  \end{axdef}
\item $PA: P \cross R$ is the assignment of permissions to security attributes.
  This is represented by different constructs of \corbasec, e.g.\ RequiredRights
  and GrantedRights \dots
  
  This is represented by the domain access policies ($DomainAccessPolicy$) which
  define the rights that are granted to certain attributes.  These policy
  objects provide operations for granting, revoking and getting rights for given
  attributes.

\item $UA: U \cross R$ is represented by the function $Users.Uprivileges$ which
  defines all privilege attributes (not just of type $Role$) for every user.
\item $user: S \fun U$ is represented by the component $Current.CUinitiator$
  which defines the user which initiated the thread of this Current object.
\item $roles: S \fun \power R$ determines the privilege attributes of type
  $Role$ of the given Current object.  This is a subset of the attributes of
  $Current.CUown\_credentials$.
\item $RH: H \cross H$ is represented by a relation $RH_c$ which must be added
  to our \corbasec{} model.  Additionally, we define the define the set $R_c$ of
  all roles that are currently known, and an auxiliary function $getRoles$ for
  retrieving all role attributes from a set of Credentials.
  \begin{schema}{RoleHierarchy}
    R_{c} : \power SecAttribute \\
    RH_{c}: SecAttribute \rel SecAttribute \\
    getRoles: \power Credentials \pfun \power SecAttribute \\
    \where \forall a: R_{c} @ Role = \\
    \t1 AttributeType\_attribute\_type(SecAttribute\_attribute\_type(a)) \\
    RH_{c} \in  R_{c} \rel R_{c} \land \id R_{c} \subseteq RH_{c} \\
    RH_{c} \comp RH_{c} \subseteq RH_{c} \land RH_{c} \cap RH_{c}\inv \subseteq
    \id R_{c} \\

    \forall cl: \power Credentials @ getRoles(cl) =  \{ a: R_{c} | (\exists c: cl
    @ a \in c.Cprivileges) \} \\
 \end{schema}
\end{itemize}
Additionally we must define administration operations for roles and users.


\subsubsection{Refinement Relation $RBACtoCORBA$}
This refinement schema relates the definitions of the \rbaci{} schema and the
\corbasec{} specification.  Please note that, to get a shorter specification, we
only consider the $SecAllRights$ rights combinator whenever the combinator is of
concern.
\begin{schema}{RBACtoCORBA}
  RBAC1 \\
  Users \\
  RequiredRightsS \\
  CurrentS \\
  CredentialsS \\
  RoleHierarchy \\
  InterfaceRepository \\
  PolicyS \\
  \where
  urbac \limg U \rimg = Uprincipals \\
  rrbac \limg R \rimg =  R_{c} \\
  srbac \limg S \rimg = \ran CUcurrentId \\

  \forall i: Identifier; rid: RepositoryId; dap: DomainAccessPolicy; p: prbac
  \limg P \rimg | \\
  \t1 p = (i, rid, dap) @ rid \in \dom IRinterface \land i \in IRinterface(rid) \\
  \t2 \land dap \in \ran PdomainAccessPolicy \\

  \forall u: U @ rrbac \limg ( UA \limg \{ u \} \rimg )  \rimg =
  Uprivileges(urbac~u) \\
  
  \forall r_1, r_2: R @ (r_1, r_2) \in RH \iff (rrbac(r_1), rrbac(r_1)) \in
  RH_{c} \\ 
  
  \forall s: S; u: U @ (s, u) \in user \iff urbac(u) = (srbac~s).CUinitiator \\

  \forall s: S; rs: \power R @ (s, rs) \in roles \iff rrbac \limg rs \rimg = \\
  \t1 getRoles(CcredentialsId \limg \ran (srbac~s).CUown\_credentials \rimg ) \\

  \forall p: P; r: R @ \exists ds: DelegationState @ \\
  \t1 (p,r) \in PA \iff \ran (first(requiredRights(getOp(prbac~p))))
  \subseteq \\
  \t2 (getPolicy(prbac~p)).PgrantedRights(rrbac(r), ds)
  
%  \forall p: PERMISSION; r: ROLE; d: DomainAccessPolicy @ \exists ds:
%  DelegationState @ \\
%  \t1 ((second(prbac~p) = SecAllRights) \implies ((p,r) \in PA(d) \\
%  \t2 \iff (\ran (first(prbac~p)) = d.grantedRights(rrbac(r), ds)))) \\
%  \t1 \land ((second(prbac~p) = SecAnyRight) \implies ((p,r) \in PA(d) \\
%  \t2 \iff (\exists s: \ran (first(prbac~p)) @ s \in
%  d.grantedRights(rrbac(r), ds)))) \\
\end{schema}

%\zsection{ref}
\paragraph{Proof obligation:} The proof obligation to show that our
configuration of \corbasec{} indeed realizes \rbaci{} is that \corbasec{} allows
access iff \rbaci{} allows access:
\zcomment{
\begin{zed}
  \forall PolicyS; RBAC1; Domains; RequiredRightsS; CredentialsS; \\
  \t2 AccessParams; RBACtoCORBA; session?: SESSION; \\
  \t3 cuid?: CURRENT; permission?: PERMISSION; \\
  \t4 operation?: Identifier | \\
  \t5 first(getOp(prbac(permission?))) = operation? \\
  \t5 \land srbac(session?) = CUcurrentId(cuid?) @ \\
  \t6 checkPermission \iff AccessAllowed \\
\end{zed}
}

\subsubsection{Administration Operations for RBAC}

\zsection[ZSrbac]{ZSrbacAdmin}

\begin{zedgroup}
  \begin{schema}{createSession}
    \Delta RBAC1\\
    user? : USER\\
    roles? : \power ROLE\\
    session! : SESSION\\
    \where
    session! \notin S\\
    S'  = S \cup \{session!\}\\
    U'  = U\\
    R'  = R\\
    P'  = P\\
    PA' = PA\\
    UA' = UA\\
    RH' = RH\\
    user? \in U\\
    user' = user \oplus \{ session! \mapsto user? \}\\
    roles? \subseteq  (UA \comp RH) \limg \{ user? \}  \rimg\\
    roles' = roles \oplus \{ session! \mapsto roles? \}\\
  \end{schema}
  \begin{schema}{deleteSession}
    \Delta RBAC1\\
    session? : SESSION\\
    \where
    session? \notin S\\
    S' = S \setminus \{ session? \}\\
    U' = U\\
    R' = R\\
    P' = P\\
    PA' = PA\\
    UA' = UA\\
    RH' = RH\\
    user' = \{ session? \} \ndres user\\
    roles' = \{ session? \} \ndres roles\\
  \end{schema}
\end{zedgroup}
We need operations to administrate the RBAC. For instance, we need operations to
add and delete associations of users to roles.

\begin{zedgroup}
  \begin{schema}{addUser}
    \Delta RBAC1\\
    usr? : USER\\
    \where
    S' = S\\
    usr? \notin U\\
    U' = U \cup \{ usr? \}\\
    R' = R\\
    P' = P\\
    PA' = PA\\
    UA' = UA\\
    RH' = RH\\
    user' = user\\
    roles' = roles\\
  \end{schema}
  \begin{schema}{deleteUser}
    \Delta RBAC1\\
    usr? : USER\\
    \where
    S' = S\\
    usr? \in U\\
    usr? \notin user \limg S \rimg\\
    U' = U \setminus \{ usr? \}\\
    R' = R\\
    P' = P\\
    PA' = PA\\
    UA' = \{ usr? \} \ndres UA\\
    RH' = RH\\
    user' = user\\
    roles' = roles\\
  \end{schema}
\end{zedgroup}  

\begin{zedgroup}
  \begin{schema}{addRole}
    \Delta RBAC1\\
    role? : ROLE\\
    \where
    S' = S\\
    U' = U\\
    role? \notin R\\
    R' = R \cup \{ role? \}\\
    P' = P\\
    PA' = PA\\
    UA' = UA\\
    RH' = RH\\
    user' = user\\
    roles' = roles\\
  \end{schema}
  \begin{schema}{deleteRole}
    \Delta RBAC1\\
    role? : ROLE\\
    \where
    S' = S\\
    U' = U\\
    role? \in R\\
    role? \notin RH \limg \bigcup(roles \limg S \rimg) \rimg\\
    R' = R \setminus \{ role? \}\\
    P' = P\\
    PA' = PA \nrres \{ role? \}\\
    UA' = UA \nrres \{ role? \}\\
    RH' = RH\\
    user' = user\\
    roles' = roles\\
  \end{schema}
\end{zedgroup}  

\begin{zedgroup}
  \begin{schema}{addPermission}
    \Delta RBAC1\\
    permission? : PERMISSION\\
    \where
    S' = S\\
    U' = U\\
    R' = R\\
    permission? \notin P\\
    P' = P \cup \{ permission? \}\\
    PA' = PA\\
    UA' = UA\\
    RH' = RH\\
    user' = user\\
    roles' = roles\\
  \end{schema}
  \begin{schema}{deletePermission}
    \Delta RBAC1\\
    permission? : PERMISSION\\
    \where
    S' = S\\
    U' = U\\
    R' = R\\
    permission? \in P\\
    permission? \notin (RH \comp PA\inv) \limg \bigcup(roles \limg S \rimg) \rimg\\
    P' = P  \setminus \{ permission? \}\\
    PA' = \{ permission? \} \ndres PA\\
    UA' = UA\\
    RH' = RH\\
    user' = user\\
    roles' = roles\\
  \end{schema}
\end{zedgroup}  

\begin{zedgroup}
  \begin{schema}{assignUser}
    \Delta RBAC1\\
    user? : USER\\
    role? : ROLE\\
    \where
    user? \in U\\
    user? \notin user \limg S \rimg\\
    role? \in R\\
    S' = S\\
    U' = U\\
    R' = R\\
    P' = P\\
    PA' = PA\\
    UA' = UA \oplus \{ user? \mapsto role? \}\\
    RH' = RH\\
    user' = user\\
    roles' = roles\\
  \end{schema}
  \begin{schema}{deassignUser}
    \Delta RBAC1\\
    user? : USER\\
    role? : ROLE\\
    \where
    user? \in U\\
    user? \notin user \limg S \rimg\\
    role? \in R\\
    S' = S\\
    U' = U\\
    R' = R\\
    P' = P\\
    PA' = PA\\
    (user? \mapsto role?) \in UA\\
    UA' = UA \setminus \{ user? \mapsto role? \}\\
    RH' = RH\\
    user' = user\\
    roles' = roles\\
  \end{schema}
\end{zedgroup}

\begin{zedgroup}
  \begin{schema}{grantPermission}
    \Delta RBAC1\\
    role? : ROLE\\
    permission? : PERMISSION\\
    \where
    S' = S\\
    U' = U\\
    role? \in R\\
    R' = R\\
    permission? \in P\\
    P' = P\\
    (permission? \mapsto role?) \notin PA\\
    PA' = PA \oplus \{ permission? \mapsto role? \}\\
    UA' = UA\\
    RH' = RH\\
    user' = user\\
    roles' = roles\\
  \end{schema}
  \begin{schema}{revokePermission}
    \Delta RBAC1\\
    role? : ROLE\\
    permission? : PERMISSION\\
    \where
    S' = S\\
    U' = U\\
    role? \in R\\
    role? \notin RH \limg \bigcup(roles \limg S \rimg) \rimg\\
    R' = R\\
    permission? \in P\\
    permission? \notin (RH \comp PA\inv) \limg \bigcup(roles \limg S \rimg) \rimg\\
    P' = P\\
    (permission? \mapsto role?) \in PA\\
    PA' = PA \setminus \{ permission? \mapsto role? \}\\
    UA' = UA\\
    RH' = RH\\
    user' = user\\
    roles' = roles\\
  \end{schema}
\end{zedgroup}


\subsubsection{Administration Operations for CORBAsec}

In this section, we map the administration operations that were defined in the
last section to administration operations for \corbasec{}.  In \corba{} we
define an \rbaci{} administration interface similar to UserAdmin (actually these
operations could be added to UserAdmin).

\index{CcreateSession}
\begin{zed}
  CcreateSession == StartClient \land PA\_authenticate \\
%  CdeleteSession ==
\end{zed}

We only change user data, if the corresponding user is not looged in into the
system and has open sessions (this precondition is copied from the pure \rbaci{}
model).
\begin{schema}{CassignUser}
  \Xi CurrentS \\
  \Xi RoleHierarchy \\
  \Delta Users \\
  s?: SecurityName \\
  r?: SecAttribute \\
  \where
  r? \in R_{c} \\
  s? \in Uprincipals \\

  \forall c: \ran CUcurrentId @ c.CUinitiator \neq s? \\

  Uprivileges' = Uprivileges \oplus \{ s? \mapsto ( Uprivileges(s?) \cup \{ r?
  \} ) \} \\
  Uattributes' = Uattributes \oplus \{ s? \mapsto ( Uattributes(s?) \cup \{ r?
  \} ) \} \\
  Uprincipals' = Uprincipals \\
  Uauthenticates' = Uauthenticates \\
\end{schema}

It is not clear how to handle this operation:  One could imagine to quit all
sessions of this user, or to simply remove the roles and use all issued
Credentials as long as they are valid, and don't revalidate them.
%\begin{schema}{CdeassignUser}
  
%\end{schema}

\begin{schema}{CgrantPermission}
  \Xi InterfaceRepository \\
  \Xi RequiredRightsS \\
  DAP\_grant\_rights \\
  op?: Identifier \\
  iface?: RepositoryId \\
  \where
  iface? \in \dom IRinterface \\
  op? \in IRinterface( iface? ) \\

  \ran ( first( requiredRights( op?, iface? ))) \subseteq \ran rights? \\
\end{schema}
We are assuming that we always have a RightsCombinator of ``ALL''.  This is
justifiable since it is more rigorous, i.e. more secure, to always require all
rights to be met.
\begin{schema}{CrevokePermission}
  \Xi InterfaceRepository \\
  \Xi RequiredRightsS \\
  DAP\_revoke\_rights \\
  op?: Identifier \\
  iface?: RepositoryId \\
  \where
  iface? \in \dom IRinterface \\
  op? \in IRinterface( iface? )
\end{schema}


\subsubsection{Refinement Proofs}

\zsection{proofs}

\paragraph{Weakening the precondition / Termination}
\begin{zed}
  \forall CredentialsS; InterfaceRepository; PolicyS; RequiredRightsS; \\
  \t1 RBAC1; CurrentS; RoleHierarchy; Users; s?: SecurityName; \\
  \t1 r?: SecAttribute; user?: USER; role?: ROLE @ \\
  \t2 \pre assignUser \land RBACtoCORBA \implies \pre CassignUser
\end{zed}

\paragraph{Consistency}
\begin{zed}
  \forall CredentialsS; InterfaceRepository; PolicyS; RequiredRightsS; \\
  \t1 RBAC1; CurrentS; RoleHierarchy; Users; s?: SecurityName; \\
  \t1 r?: SecAttribute; user?: USER; role?: ROLE; \\
  \t1 CurrentS '; RoleHierarchy '; Users '; \\
  \t1 CredentialsS '; InterfaceRepository '; PolicyS '; RequiredRightsS ' @ \\
  \t2 \pre assignUser \land RBACtoCORBA \land CassignUser \implies \\
  \t3 (\exists RBAC1 ' @ assignUser \land RBACtoCORBA ' )
\end{zed}


%%% Local Variables: 
%%% mode: latex
%%% TeX-master: "model"
%%% End: 
