s\rcsInfo $Id: sys-arch.tex,v 1.37 2004/01/29 10:36:23 hiss Exp $

\chapter{A Formal Model of the System Security Architecture}\label{cha:sys-arch}

In this section, we will describe the more abstract view of our system. In order
to avoid unnecessary complexity, we will reduce our architecture to exactly one
client communicating with the server.  This is realistic since the CVS server
processes all requests sequentially and different clients could be modeled by
one client logging in and out with different user IDs and roles.  We will model
the state of a client and the state of the CVS repository and the operations the
combined system may perform.  These operations include read and write access of
clients to the repository and allow for a fairly compact description of the
access control policy the CVS-Server may enforce.

This access control policy is well known as \emph{role-based access
  control}\index{RBAC}\index{role-based access control}. In
\cite{sandhu.ea:role-based:1996}, a formal framework of role-based
access control is described that introduces the \rbac-model for role
hierarchies that is closely related to the security policy enforced by
our system architecture.  However, we differ from \rbac{} in a number of
minor issues in order to get a model that can be refined to the
implementation architecture.  For instance, we will \emph{not} model
\emph{sessions} of \rbac directly.  Moreover, since we assume a
one-to-one correspondence between roles and permissions (i.e.\ the
\emph{permission assignment $(PA)$}\index{permission assignment} is a
bijective function), we directly define the role hierarchy $(RH)$ on
permissions, which is more convenient for technical reasons.  Further,
since the administrator is part of the model (the authorization for a
user may thus be dynamically withdrawn), the pair consisting of a
\emph{user ID}\index{user ID} and its authentication (a
\emph{password}\index{password}) representing a \emph{permission} has
to be kept in the user state and can not be evaluated to a permission
for the whole session during login-time; rather, for any access to the
repository, the system checks (as the real CVS-Server) if the pair
still represents a valid permission at access-time.

Besides the access operations, we introduce a
\emph{login}\index{login} operation that marks the entry of a session
(if we neglect the dynamic aspect of authentication).  During the
login operation, a user authenticates for a role; this operation
either results in a change of the permissions of the client state
corresponding to that user or in a failure.


\section{Fundamental Entities of the Model}\label{sec:basics}

%%macro \zcomment 1

\zsection{Basics}\vspace{1ex}\noindent

Common to both architectural models is an adopted \rbac-model.  Therefore, we
introduce basic data types for users, permissions (one particularity of our
model is the isomorphism of roles and permissions), and passwords for
authenticating.
\begin{zed}
  [ Cvs\_Uid, Cvs\_Perm, Cvs\_Passwd ]
\end{zed}

The role hierarchy\index{role!hierarchy} is captured by a reflexive,
transitive relation with the administrative permissions ($cvs\_admin$)
as greatest and the permissions for public access ($cvs\_public$) as
least element.  Since roles and permissions are isomorphic in our
model, for convenience, we define the order directly on permissions.
\begin{doc}{4}
  \begin{axdef}
    cvs\_admin, cvs\_public : Cvs\_Perm \\
    cvs\_perm\_order        : Cvs\_Perm \rel Cvs\_Perm \\
    \where
    cvs\_perm\_order = cvs\_perm\_order \star \\
    \forall x: Cvs\_Perm @ (x,cvs\_admin) \in cvs\_perm\_order \\
    \forall x: Cvs\_Perm @ (cvs\_public,x) \in cvs\_perm\_order \\
    \forall x: Cvs\_Perm @ (x \neq cvs\_admin)\implies (cvs\_admin,x) \notin cvs\_perm\_order \\
    \forall x: Cvs\_Perm @ (x \neq cvs\_public)\implies (x, cvs\_public) \notin cvs\_perm\_order \\
    \forall x: Cvs\_Perm @ \exists y: Cvs\_Perm @ (x,y) \in cvs\_perm\_order\\
  \end{axdef}
\end{doc}

For authentication\index{authentication} and
authorization\index{authorization} of the client upon him accessing
the repository or for logging in, we need a fixed preconceived table
that associates permissions to an ID, a client has authenticated for
with his password.  Additionally, we define a table that associates to
CVS user IDs the corresponding passwords.  The client state of our
model will manage such a table.
\begin{zed}
  AUTH\_TAB == Cvs\_Uid \cross Cvs\_Passwd \pfun Cvs\_Perm \\
  PASSWD\_TAB == Cvs\_Uid \rel Cvs\_Passwd \\
\end{zed}


\section{The State of the System Architecture}

\zsection[Basics]{AbstractState}\vspace{1ex}\noindent

As a basis to define the system state and its operations in the Z-sections
\texttt{AbsState} and \texttt{AbsOperations} we have to define Z data types that
represent files (a mapping from names to data), and data types for modeling
access control properties, e.g., necessary permissions to access data.
\begin{zed}
  [Abs\_Name, Abs\_Data] \\
  ABS\_DATATAB == Abs\_Name \pfun Abs\_Data \\
  ABS\_UIDTAB == Abs\_Name \pfun Cvs\_Uid \\
  ABS\_PERMTAB == Abs\_Name \pfun Cvs\_Perm \\
\end{zed}
CVS server store their authentication table inside the repository --- thus, the
table can be accessed and modified using the operations of CVS\@. We will
capture this fact in our model; as a prerequisite, we define a name
$abs\_cvsauth$ to which data is associated, that is isomorphic to an
authentication table. This isomorphism is established by a pair of functions ---
that can be viewed as \emph{parser} and \emph{pretty\_printer} --- and the usual
axioms.
\begin{axdef}
  abs\_cvsauth: Abs\_Name \\
  abs\_auth\_of: Abs\_Data \pfun AUTH\_TAB \\
  abs\_data\_of: AUTH\_TAB \fun Abs\_Data \\
  authtab   : ABS\_DATATAB \pfun AUTH\_TAB \\
  \where
  \ran(abs\_data\_of) \subseteq \dom(abs\_auth\_of) \\
  \forall x : \dom abs\_auth\_of @ abs\_data\_of(abs\_auth\_of~x) = x \\
  \forall x : AUTH\_TAB @ abs\_auth\_of(abs\_data\_of~x) = x \\
  \forall r : ABS\_DATATAB @   abs\_cvsauth \in \dom(r) \implies \\
  \t2 authtab(r) = abs\_auth\_of(r~abs\_cvsauth) \\
\end{axdef}

A vital predicate for the access control is the predicate $has\_access$.  It
checks according to the underlying data-structures of the server, if the user
has for a particular $Cvs\_Uid$ (i.e.\ a ``role'') and a particular $file$ access
in the repository. This requires:
\begin{enumerate}
\item that role and actual password are defined in the authentification
      table in the repository,
\item and that the resulting permission from the authentification
      is sufficient with respect to the permission ordering.
\end{enumerate}
Formally, this predicate looks as follows:
%%prerel has\_access
\begin{axdef}
  has\_access \_ :\power(ABS\_PERMTAB\cross ABS\_DATATAB\cross PASSWD\_TAB\cross Abs\_Name\cross Cvs\_Uid)
  \where
  \forall rep\_pt : ABS\_PERMTAB; rep:ABS\_DATATAB; pwtb : PASSWD\_TAB; \\
    \t1 file:Abs\_Name; role: Cvs\_Uid @ \\
    \t2 has\_access(rep\_pt,rep,pwtb,file,role)  \iff \\
    \t3 (role, pwtb(role)) \in \dom(authtab~rep) \\
    \t3 \land (rep\_pt(file),authtab(rep)(role, pwtb(role)))\in cvs\_perm\_order
\end{axdef}


Note that the parser $abs\_auth\_of$ is a partial function; this models the fact
that not all file contents do actually represent input that can be interpreted.
Moreover, we did not make any assumptions on authentication tables $AUTH\_TAB$;
for example, a requirement like ``there is someone that can authenticate as
$cvs\_admin$'' (i.e.  $ cvs\_admin \in \dom at$ where $at : AUTH\_TAB$) would be
helpful but is omitted since the real CVS-Server does not make this assumption.
This has far reaching consequences; in particular, it is possible to bring the
CVS server into a state where all operations block since all authentications
fail, which resembles a \emph{denial of service
  attack}\index{attack!denial of service} at the \posix-layer.

In the remainder of this section, we define the states of the client and the
server component.  To motivate the details of our specification, we present an
excerpt from the informal description of the CVS \cvscmd{login} command:
\begin{quote}
  \emph{Contacts a CVS server and confirms authentication information for a
    particular repository. This command does not affect either the working copy
    or the repository; it just confirms a password with a repository and stores
    the password for later use in the .cvspass file in your home directory.
    Future commands accessing the same repository with the same username will
    not require you to rerun login, because the client-side CVS will just
    consult the .cvspass file for the password.}~\cite{fogel:cvs:1999}
\end{quote}
The state of the client ($ClientState$) captures the concept of a working copy
in the variable $wc$, a set of files, and also introduces $wc\_uidtab$ which is
used to record for each file in the working copy the user ID that was used to
access this file in the repository.  The above mentioned file ``.cvspass'' is
modeled by $abs\_passwd$ which is a set of pairs of user IDs and passwords the
client has logged in with.  In addition, we introduce a set of focused files
($wfiles$) which will be the implicit arguments for most CVS operations.
\begin{doc}{ClientState}
  \begin{schema}{ClientState}
    wfiles: \power Abs\_Name \\
    wc: ABS\_DATATAB \\
    wc\_uidtab: ABS\_UIDTAB \\
    abs\_passwd: PASSWD\_TAB \\
  \end{schema}
\end{doc}
It is straightforward to motivate the state $RepositoryState$ of the server
component: The central concept of CVS is the repository, a set of files, which
is defined as $rep$.  Since the access to each file is checked individually, we
also define $rep\_permtab$, a table that associates the required access
permissions to each file.

As mentioned above, CVS-Server stores the authentication data inside the
repository --- thus it can be accessed and modified with CVS operations.  To
model this, we introduce a name $abs\_cvsauth$ and auxiliary functions.  In the
server state, we now require that $abs\_cvsauth$ is a file in the repository,
and that only the CVS administrator has sufficient permissions to access the
authentication data, i.e.\ we set the necessary permission to the greatest
element $cvs\_admin$ of our permission/role hierarchy.
\begin{doc}{RepositoryState}
  \begin{schema}{RepositoryState}
    rep: ABS\_DATATAB \\  
    rep\_permtab: ABS\_PERMTAB \\
    \where
    abs\_cvsauth \in \dom rep \\
    \dom rep = \dom rep\_permtab \\
    rep\_permtab(abs\_cvsauth) = cvs\_admin \\
    rep(abs\_cvsauth) \in \dom abs\_auth\_of \\ 
  \end{schema}
\end{doc}


\section{The Operations of the System Architecture}

\zsection[AbstractState]{AbsOperations}\vspace{1ex}\noindent

Now we define the operations of the system that model combined state transitions
of the client and the repository.  We begin with the \cvscmd{login} operation of
our CVS-Server system architecture.  Logging in marks the begin of a session, in
which the client authenticates for permissions that control his access to the
repository.  The operation \cvscmd{cd} sets the focused files ($wfiles$) for the
following operations.
\begin{zedgroup}
  \begin{doc}{absops_login}
    \begin{schema}{abs\_login}
      \Delta ClientState \\
      \Xi RepositoryState \\
      passwd?: Cvs\_Passwd \\
      uid?: Cvs\_Uid \\
      \where
      (uid?,passwd?) \in \dom (authtab~rep) \\
      abs\_passwd' = abs\_passwd \\
      \t1 \oplus \{ uid? \mapsto passwd? \} \\
      wc' = wc \\
      wc\_uidtab' = wc\_uidtab \\
      wfiles' = wfiles \\
    \end{schema}
  \end{doc}
  \begin{doc}{absops_cd}
    \begin{schema}{abs\_cd}
      \Delta ClientState \\
      \Xi RepositoryState \\
      wfiles?: \power Abs\_Name \\
      \where
      wfiles' = wfiles? \\
      wc' = wc \\
      wc\_uidtab' = wc\_uidtab \\
      abs\_passwd' = abs\_passwd \\
    \end{schema}
  \end{doc}
\end{zedgroup}

The next three operations allow for a synchronization of the working copy with
the operations. Recall that our model deliberately neglects all aspects related
to version control and merging of files; our model focuses on the security
aspects of the CVS-Server.

The \cvscmd{add} operation mirrors the fact that the user has created new
content (new filenames as well as new associated data). The operation adds this
new data to the local working copy and makes sure that the new names are
actually new in the repository.
\begin{doc}{absops_add}
  \begin{schema}{abs\_add}
    \Delta ClientState \\
    \Xi RepositoryState \\
    newfiles?: ABS\_DATATAB \\
    \where
    \dom newfiles? \cap \dom rep = \emptyset \\
    wc' = wc \oplus newfiles?  \\
    wc\_uidtab'= wc\_uidtab \oplus \{ n : \dom newfiles? | n \notin \dom wc\_uidtab \\
    \t2  \land (\exists id : \dom abs\_passwd @ \\
    \t3 (id, (abs\_passwd~id)) \in \dom (authtab~rep)) @\\
    \t2 n \mapsto (\mu id:
    \dom abs\_passwd | (id, (abs\_passwd~id)) \in \dom (authtab~rep)) \} \\
    wfiles' = wfiles \\
    abs\_passwd' = abs\_passwd \\
  \end{schema}
\end{doc} 
The \cvscmd{commit} and \cvscmd{checkin} operations usually take a set if files
as arguments (here denoted by $files?$).  However, if no arguments are provided
then these operations use the set of currently focused files ($wfiles$) as
implicit arguments.  This is modeled by restricting $files?$ by 
$wfiles$\footnote{Note that in order to operate on the
complete set of focused files, $files?$ must be equal to $wfiles$.}.

In our model the \cvscmd{commit} (or \cvscmd{checkin}) operation assumes that 
for all focused files (restricted by the input parameter), the user actually has access
to these files in the repository, i.e.\ his actual permission is larger than the
required one in the repository\footnote{We slightly oversimplified the
  situation for the sake of the presentation: the real implementation actually
  filters out the non-accessible files, while in our model, the operation
  blocks.}. For these files, the content of the working copy is transferred to
the repository. Note the use of $wc'$ here to mirror the fact that the data in
the working copy may have changed non-deterministically by user interaction and
merge operations which are excluded from our model.
\enlargethispage{\baselineskip}
\begin{doc}{absops_ci}
  \begin{schema}{abs\_ci}
    \Xi ClientState \\
    \Delta RepositoryState  \\
    files?: \power Abs\_Name \\
    \where
    (wfiles \cap files?) \subseteq \dom wc \\

    rep' = rep \oplus (\{ n: wfiles \cap files? | n \notin \dom rep \land n \in
    \dom wc\_uidtab \\
    \t1 \land (wc\_uidtab(n), abs\_passwd(wc\_uidtab~n)) \in
    \dom(authtab(rep))\} \dres wc) \\  
    \t2 \oplus (\{n : wfiles \cap files? | n \in \dom rep \land n \in \dom
    wc\_uidtab \\
    \t2 \land has\_access(rep\_permtab,rep,abs\_passwd,n,wc\_uidtab(n))\}
    \dres wc) \\

    rep\_permtab' = rep\_permtab \oplus \{ n: wfiles \cap files? |  n \notin
    \dom rep \land n \in \dom wc\_uidtab \\
    \t1 \land (wc\_uidtab(n), abs\_passwd(wc\_uidtab~n)) \in \dom(authtab~rep) @
    \\
    \t2 n \mapsto authtab(rep)(wc\_uidtab(n), abs\_passwd(wc\_uidtab~n)) \} \\

  \end{schema}
\end{doc}

The \cvscmd{update} operation updates every file in the working copy by the
corresponding file in the repository if the client has sufficient permissions to
access the file in the repository, i.e., is in a senior enough role. This
operation does not block if the client does not have sufficient permissions, but
silently ignores those files.  Additionally, this operation also updates the
permission table in the working copy.  If an accessible file does not exist in
the working copy $wc$ of the client, then $abs\_up$ simply inflates it from the
repository. Otherwise, CVS \cvscmd{update} originally distinguishes between four cases. The
working copy is \ldots
\begin{itemize}
  \item up-to-date and not locally modified. There is no action.
  \item up-to-date and locally modified. The message \glqq locally
    modified\grqq{} is displayed.
  \item outdated but not locally modified. The old version is replaced by the
    new one.
  \item both outdated and locally modified. An RCS~\cite{tichy85rcs}\index{RCS} merge
    operation (a 3-way-merge) is used to combine the working copy, the version
    in the repository and the latest ancestor which they have in common. This is
    the version of the file in the repository with the revisionnumber of the
    corresponding file of the working copy.
\end{itemize}

In our model we simplify this procedure using a non-deterministic merge
operation which implicitly handles all four cases.

\begin{doc}{3}
  \begin{axdef}
    merge : (Abs\_Data \cross Abs\_Data) \fun Abs\_Data \\
    \where
    \forall x : Abs\_Data @ merge(x,x) = x \\
  \end{axdef}
\end{doc} 

\begin{doc}{absops_up}
  \begin{schema}{abs\_up}
    \Delta ClientState \\
    \Xi RepositoryState  \\
    files?: \power Abs\_Name \\
    \where 
%wc' = wc \oplus (\{ n: wfiles \cap files? | n \in \dom rep \land n \in \dom
%    wc\_uidtab \\
%    \t1 \land (wc\_uidtab(n), abs\_passwd(wc\_uidtab~n)) \in \dom(authtab~rep) \\
%    \t1 \land (rep\_permtab(n),authtab(rep)(wc\_uidtab(n), \\
%    \t2 abs\_passwd(wc\_uidtab~n)))\in cvs\_perm\_order\} \dres rep) \\
wc' = wc \oplus (\{ n: wfiles \cap files? | n \notin \dom wc \land n \in \dom rep \land n \in \dom
    wc\_uidtab' \\
    \t1 \land has\_access(rep\_permtab,rep,abs\_passwd,n,wc\_uidtab'(n))\} \dres rep) \\    
    \oplus \{ n: wfiles \cap files? | n \in \dom wc \land n \in \dom rep \land n \in \dom
    wc\_uidtab \\
    \t1 \land has\_access(rep\_permtab,rep,abs\_passwd,n,wc\_uidtab(n)) @ n \mapsto (merge(rep(n), wc(n)))\}\\

    wc\_uidtab' = wc\_uidtab \oplus \{ n: wfiles \cap files? |  n \in \dom rep \\
    \t1 \land n \notin \dom wc\_uidtab \land (\exists id : \dom abs\_passwd @ \\
    \t2 has\_access(rep\_permtab,rep,abs\_passwd,n,id)) @\\
    \t2 n \mapsto (\mu id: \dom abs\_passwd | has\_access(rep\_permtab,rep,abs\_passwd,n,id)) \} \\

    abs\_passwd' = abs\_passwd \land wfiles' = wfiles \\
  \end{schema}
\end{doc}

In addition to these abstract models of the CVS operations we provide a modify operation which explicitly models interactions of users with their files via modifying the files of the working copy of the client state.

\begin{doc}{absops_modify}
  \begin{schema}{modify}
    \Delta ClientState \\
    \Xi RepositoryState  \\
    filename? : Abs\_Name \\
    newData? : Abs\_Data \\
    \where 
    filename? \in \dom wc\\
    wc' = wc \oplus\{ filename? \mapsto merge(wc~filename?, newData?)\} \\
    wfiles' = wfiles \\
    wc\_uidtab' = wc\_uidtab \\
    abs\_passwd' = abs\_passwd \\
  \end{schema}
\end{doc}

%%% Local Variables:
%%% TeX-master: "arch"
%%% fill-column:80
%%% x-symbol-8bits:nil
%%% End:
