\rcsInfo $Id: intro.tex,v 1.11 2003/12/04 15:03:06 brucker Exp $

\chapter{Introduction}\label{chapter:introduction}
These days, the \emph{Concurrent Versions System}\index{Concurrent
  Versions System|see{CVS}}\index{CVS} (CVS) is a widely used tool for
version management in many industrial software development projects
and plays a key role in open source projects usually performed by
highly distributed
teams~\cite{cederqvist.ea:cvs:2000,fogel:cvs:1999,cvshome:2001}.  CVS
provides a central data base (the \emph{repository}\index{repository})
and means to synchronize local partial copies (the \emph{working
  copies}\index{working copy}) and their local modifications with the
repository\index{repository}. CVS can be accessed via a network; this
requires a security architecture establishing authentication, access
control and non-repudiation.  A further complication of the CVS
security architecture stems from the fact that the administration of
authentication and access control is done via CVS itself; i.e.\ the
relevant data is accessed and modified via standard CVS operations
and, thus, access to objects may change dynamically.
 
The current standard CVS-server distribution (based on the
\texttt{pserver}-protocol)\index{pserver@\texttt{pserver}} is known to
be quite insecure: since passwords are not encrypted during
communication, usual password-sniffing attacks can be applied;
moreover, a bug in some CVS-operations can allow an arbitrary user to
modify the password authentication table.  Further, any CVS user (with
write access) can execute any program on a server with the userid he
is authenticated to. This means he can send himself a command shell.
Via buffer-overflow attacks, there is the inherent danger that the
CVS-server can be crashed leaving a command shell under root
permissions.  Additionally, the repository in itself is not protected
against direct user manipulations via file system operations.

Moreover, the current CVS-server distribution inherits the usual \unix{}
permission\index{Unix!permission} concept --- this may be too low-level and
inadequate to cope with the demands of many user-groups and their work
organization. In particular, we found it useful to provide a special
configuration of the CVS-server that enforces a hierarchical \emph{role based
  access control}\index{role based access model} model as described
in~\cite{sandhu.ea:role-based:1996}. This model associates to users one or more
\emph{roles} (e.g.\ project supervisor, test engineer, programmer, project
member, etc) that are organized in a partial order. Further, \emph{permissions}
are associated to objects in the repository; a user may authenticate for a
particular role in order to get appropriate permissions for object access.

In order to install the current CVS-server in security conscious way,
the traditional architecture requires the server to run on a
secured machine that has exclusive access to an own file-system
hosting the repository. This \emph{closed architecture} has a number
of administrative and technical costs (additional hardware,
efficiency).

In order to overcome some of these problems, we propose a number of improvements
of the standard CVS-server, either on the level of its implementation (via
patches), its configurations (i.e.\ the file system, including the initial state
of the repository; this extends to group-tables, file permissions, scripts
started when check in or check out data into the repository) or its architecture
(i.e.\ the particular setup of a cvs-server and its configuration in a network).
Our first main goal of our work is to provide a particular configuration of
CVS-server that enforces a role-based access control security policy. Our second
main goal is to develop an \emph{open} CVS-server architecture, where the
repository is part of the usual shared file-system of an intranet and the server
a regular process on some machine in it. While an open architecture has
undoubtedly a number of advantages, the correctness and trustworthiness of the
security mechanisms becomes a major concern.  In order to meet these concerns,
we will present in this paper formal models and an analysis of the open
CVS-server security architecture.


Our contribution in this paper is fivefold:
\begin{enumerate}
\item We provide a formal model of the CVS-server \emph{system
    architecture}\index{system architecture}\index{architecture!system}, that
  provides a \emph{hierarchical} role based access control model for a fixed
  preconceived hierarchy of CVS-roles. Our CVS-server security architecture
  model contains \cvscmd{checkout} and \cvscmd{checkin} operations of CVS with respect to the
  access control policy resulting from the CVS-role hierarchy.
\item We provide a formal model of the \emph{implementation
    architecture}\index{implementation
    architecture}\index{architecture!implementation} based on the UNIX/POSIX
  file-systems and its security mechanisms.  Methodologically, the
  implementation architecture is a refinement of the system architecture.
\item We provide semi-formal and formal (machine-assisted) proofs for the
  consistency of the specification and for the refinement.  Moreover, we provide
  proofs for the security properties both on the system and the implementation
  level of the architecture.  The proofs take into account that on the
  implementation level UNIX-operations like \unixcmd{cd}, \unixcmd{chmod},
  \unixcmd{chown} can be used within attacks.
\item We provide insight into the limits of refinement based,
  well-known abstraction techniques for the security analysis of
  real-world applications.  In particular, realistic security analysis
  involves the formal analysis of \emph{attacks against the
    implementation}; thus, models have to provide sufficient detail of
  the security mechanisms of the implementation architecture.
\item We provide a particular \emph{configuration} of our CVS-server
  architecture, that implements our open, intranet compliant architecture. Our
  implementation is based on standard CVS extended by \emph{cvsauth}\index{cvsauth} that
  incorporates SSL/TSL-support~\cite{itef:rfc2246:1999}\index{secure socket
    layer|see{SSL}}\index{SSL} for secure communication of remote CVS access.
\end{enumerate}

The overall benefit of our implemented architecture consists in
security against most of the known attacks as listed above: Password
sniffing is prevented by encryption, gaining root access by
cvs-command attacks is prohibited, the danger of gaining root access
by buffer-overflow attacks is limited to the authentication phase,
which is the only remaining phase where the server runs under root.
Additionally, administration of our CVS-server architecture is
independent from system administration (except during initialization).
And, last but not least, our open CVS-server can run on an arbitrary
machine (say, the file server) with standard access to the
file-system, which has become possible due to our formal modeling and
analysis.
   
As formal specification language, we use the Z-Notation~\cite{iso:z:2000} 
for specifying our model. For theorem-proving, we compile our
\LaTeX-based specification via a own translator to 
HOL-Z~\cite{kolyang.ea:hol:96,brucker.ea:hol-z:2002} based on
Isabelle98~\cite{paulson:isabelle:1994,nipkow.ea:isabelle:2002}.  Since Z is typed set-theory
and as such semantically equivalent to higher-order logic (HOL; modulo
minor differences w.r.t.\ polymorphism; see~\cite{santen:HOL:1998} for
details.), we treat Z as a mere syntactic front-end to HOL\@. As such,
Z offers a compact syntax geared towards data-modeling, which is
normed~\cite{iso:z:2000} and for which many excellent text-books for
software-engineers are available. For Z, an excellent \LaTeX-based
literate specification environment is available (\cite{zeta}; with own
type-checker etc.).

We proceeds as follows: After a discussion of our notion of
architecture and architectural refinement, we come to the core
chapters of this paper: the system architecture model (or: abstract
layer of the refinement), the \unix{} model as infrastructure for the
implementation architecture, and the implementation architecture
itself. Further, we describe the refinement relation between these
two, the security properties formalizable at the different layers and
their analysis.

%%% Local Variables:
%%% TeX-master: "arch"
%%% fill-column:80
%%% x-symbol-8bits:nil
%%% End:
