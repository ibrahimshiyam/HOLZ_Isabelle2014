\rcsInfo $Id: conclusion.tex,v 1.14 2003/12/04 15:03:06 brucker Exp $

\chapter{Conclusion}

We presented a case-study where an access control security problem for a
realistic system has been made amenable to formal, machine-based analysis. As a
hidden theme, we also presented a \emph{method} for analyzing security in
off-the-shelve operating system environments: First, specify the security
architecture (as a framework for formal security properties), second, specify
the implementation architecture (validated by inspecting informal specifications
or code and by testing), third, set up the security technology mapping as a
refinement problem, forth, establish a formal security analysis over the
transitive closure over the system transition relation, and fifth, prove
refinements and security properties by mechanized, gap-less proofs. This method
is applicable for a wider range of security problems and may be relevant, e.g.,
in mission critical e-commerce applications or in e-government applications,
where high-level certifications are mandatory.

More precisely, we have mapped an abstract model of the CVS-Server (with the
access-control model RBAC inside) to a more concrete implementation model based
on the concrete security mechanisms as offered by the traditional UNIX/POSIX
security architecture. Moreover, we also provided an implementation in form of a
configuration of the standard off-the-shelf implementation of
CVS~\cite{cvshome:2001}.  Thus, we showed an application of formal methods for
the field of security implementations, both from the technical and the
conceptual side. In the sequel, we will discuss these aspects in more detail.

\section{The Technical Side: A Secure CVS Configuration}
Our formal model (and its analysis) built the foundation of ``our'' CVS server
setup in our group at the University of Freiburg. At present, it is applied for
a repository containing $2.5$ Gigabyte of data for over $80$ users categorized
in five roles.

The presented setup addresses our main requirements: it does not monopolize an
additional machine with the CVS service, which would result in special
administration and installation of it, and it enforces a highly desirable access
control model. Moreover, any machine in our network can be assigned to our CVS
service, without any loss in security. Our CVS-server implementation requires
only the standard access control policy provided by the Unix standard, i.e.\ there
is no need for e.g.\ access control lists (ACL) as provided by some Unix
variants. Further, our implementation guarantees the encrypted transmission of
the passwords and the versioned data (not discussed in this report).

The main disadvantage for the users is the need for a special (patched) CVS
client, which also leads to some additional administrative work, because we have
to distribute these clients in source code and as binary for different operation
systems. Further, the security policy is `hard-wired' into our setup, e.g.\ 
support for write exclusion within one CVS role is realized via a different
mechanism.

\subsection{Related Work}
Other approaches for securing the standard CVS client/server model (using the
\texttt{pserver}-protocol) can roughly be divided into four classes:
\begin{enumerate}
\item Securing authentication and network traffic by providing SSL
  support, e.g.\ via tunneling~\cite{berezin:tunnel:2001} or wrapping
  the standard CVS server~\cite{vogt:cvsauth:2001},
\item Protecting the local network by running the standard CVS server in a
  sandbox (chrooted) environment~\cite{hess:anonymous:2001,idealx:chroot:2001},
\item Re-engineering the implementation of the
  \texttt{pserver}-protocol~\cite{nserver},
\item Setting up large scale closed servers providing CVS functionality, together with
  a project administration ~\cite{fsf:savannah:2001,osdn:sourceforge:2001}.
\end{enumerate}
These solutions mainly attempt to fix the known problems with the standard
CVS implementation by either providing encryption or minimizing the harms of a
intruder.  Whereas this is clearly one part of our problem, we also wanted to
provide an access control model, which lead to our choice
of~\cite{vogt:cvsauth:2001} as basis for our work, which solves the vital
problem of authentication and encryption of data transmission. Our DAC based
RBAC setup is an add-on on top of that, such that our setup can be used
together with most of the other approaches listed above.
\subsection{Future Work}
In our opinion, the greatest improvement of our CVS-server setup would be the
support of more flexible access control policies. Our implementation supports
already write exclusion within a CVS permission via a script\footnote{This script,
  called \texttt{cvs\_acls}, is part of the standard CVS distribution and can be
  found in the \texttt{contrib} directory.} implementing ACLs (access control
lists). Whereas such a write exclusion support clearly opens the door for
generic access policies its success is clearly limited by the lack of generic
hooks for implementing security checks (see for example~\cite{Linux-LSM} for
such an interface). Therefore any effort implementing a generic access control
into CVS (without rewriting the CVS system) would either lead to a complex and
intransparent configuration or it would heavily depend on the
underlying operation system (e.g.\ ACLs of the underlying filesystem), or both.

Taking all the well known hassles (e.g.\ moving of files, \ldots) of CVS into
account, we see a great future for projects creating a \emph{successor} of
CVS\@. Candidates for a replacement --- we only mention
Subversion~\cite{subversion} and OpenCM~\cite{opencm} here --- are taking long
strides toward production use. Subversion can be considered as the natural choice,
since it aims to solve the problems of CVS on the version control side
and supports a fine grain access control model using WebDAV~\cite{} in the future.
On the other side, OpenCM was designed with a refined access control mechanism
beyond RBAC from the beginning, but is weaker as a version control system.

Concluding, when setting up a \emph{new} version control system in the near
future, we would probably use one the CVS successors. We would favor
Subversion over OpenCM because of its better support in the free software
community and our relatively modest demands with respect to access
control.

\section{The Conceptual Side: Formal Methods for Security Technology}

It has been widely recognized that security properties can not be easily refined
--- actually, finding refinement notions that preserve security properties are a
hot research topic. However, standard refinement proof technology has still its
value here since it checks that abstract security requirements (which can be
seen as \emph{security against unintentional misuse}) are indeed achieved by a
mapping to concrete security technology, and that implicit assumptions on this
implementation have been made explicit. With respect to security against
\emph{intentional exploits of security leaks}, we believe that specialized
refinement notions will be limited to restricted aspects of a system. For this
problem, in most cases the answer will be to analyze the security on the
implementation level, possibly by reusing results from the abstract level. From
the methodological viewpoint this simply means that for an analysis of
\emph{intentional exploits of security leaks} implementation architectures must
simply be taken more seriously, which implies that more detailed and
implementation oriented models deserve more attention as before, where more
abstract models have been preferred. But in security analysis, more abstract
models are not necessarily better ones.

It is characteristic for our approach to consider security properties like
access control as ordinary functional properties; as a consequence, the
classical distinction between security and safety is very often difficult to
establish within our model. When refining a security architecture down
to an
implementation architecture, it is merely tradition to consider one form of
exceptional behavior (e.g.\ precondition that $has\_w\_access$ holds for a user
and a file) as security and another form (e.g.\ a path has the right format and
enables indeed to access a file in the filesystem) as safety.

\subsection{Related Work}

\begin{enumerate}
\item The common reference for stating that security properties (i.e.\ 
  information flow) are incompatible with refinement
  is~\cite{jacobs:derivation:1998}, where also a first method for stepwise
  development is proposed. More recent work for information flow
  is~\cite{Mantel:inp:2001} (where also a nice overview can be found). Other
  recent work on security-preserving refinements
  are~\cite{juerjens:secrecy-preserving:2001} (secrecy)
  and~\cite{santen:confidentiality-preserving:2002} (confidentiality).  To all
  these approaches, the critique above applies: these notion are focused on one
  particular security notion (so the question on combination arises), the
  transition to ``realistic'' implementations is not handled, such that these
  techniques remain limited to special aspects of a system.
\item Sandhu already described in~\cite{sandhu.ea:decentralized:1998} a method
  for embedding Role-Based Access Control with the Discretionary Access Control
  provided by standard \unix{} systems. Our implementation used this
  construction for providing the static role.
\item To our knowledge, there is amazingly few work on formalizing
  the UNIX/POSIX filesystem in the literature. Some approaches focus
  on the functional
  aspects~\cite{heisel:specification:1995,morgan.ea:unix-filesystem:1984}.
  The model in the Isabelle/HOL distribution by Markus Wenzel copes
  with security aspects and has inevitably many similarities with ours;
  however, it does not handle the execution flag, groups, not to
  mention the set-bits which play an important role in the role
  propagation when creating subdirectories.
 \end{enumerate}
\subsection{Future Work}
The formal proofs we did so far represent in our opinion a ``proof of
technology'' for this type of reasoning, but we do not claim that they represent
a thorough and complete analysis of the (real) CVS-server. So far, most
consistency, but only selected refinement and security properties have been proven.

However, from our experience gained by our proofs, we believe that specialized
proof technologies for certain types of proofs (refinement, security-properties)
in our methodological frame and concrete proof-technology can be improved and
will lead --- together with more and extended standard models for operating
system interfaces --- will dramatically increase the effectivity and feasibility
of our approach.


%%% Local Variables:
%%% TeX-master: "arch"
%%% fill-column:80
%%% x-symbol-8bits:nil
%%% End:
