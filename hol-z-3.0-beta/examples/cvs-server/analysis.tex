\rcsInfo $Id: analysis.tex,v 1.34 2003/12/29 09:30:25 hiss Exp $

\chapter{Formal Analysis}

In this chapter, we formally investigate functional and security properties of
both abstraction layers of the CVS-Server. By their nature, these are
\emph{behavioral properties}, i.e.\ it is necessary to consider the set of possible
\emph{sequences} of operations the system may engage in --- the \emph{traces}
--- and state requirements on the possible states the system reaches after these
traces. Hence, the specification of the safety properties in the system
architecture and the implementation architecture motivate two Z sections that
contain classical \emph{behavioral} specifications.

As functional properties, we describe requirements of what the system should do:
Give a client access to the objects in the repository it has permission to. In
contrast, security properties in our setting state that the client \emph{must
  not} access data that he has no permission to, whatever combination of
cvs-commands he applies.

Methodically, we need an interface between the data-oriented part of the
specification in previous chapters and the behavioral part as presented in the
following chapter. The trick is done by converting the operation schemas of both
system layers into relations over the underlying state. Traces are represented
as transitive closures over the union of these relations; in principle, all
sorts of modal operators can be used to specify properties over them.


\section[System Architecture]{Properties of the System Architecture}

\subsection{Security Properties}\label{sec:secprop1}

\zsection[AbsOperations]{SysArchSec}\vspace{1ex}\noindent

We assume that a user ``knows'' a set of pairs of IDs and passwords, and that
the user ``invents'' only files from a given set of pairs from names to data in
the \cvscmd{add}-operation.
%
% ---- Aufbauendes Prinzip
%
%%\begin{axdef}
%%  MH : \power (Abs\_Name \cross Abs\_Data) 
%%  \fun \power (Abs\_Name \cross Abs\_Data)\\
%%  \where
%%    \forall S : \power (Abs\_Name \cross Abs\_Data); 
%%    name : Abs\_Name ; data : Abs\_Data @\\ 
%%    \t1 (name, data) \in MH(S) \iff \\
%%    \t2 ((name, data) \in S \\
%%    \t2 \lor (\exists mdata : Abs\_Data @ \\
%%    \t3 (name, merge(data, mdata)) \in MH(S)\\
%%    \t3 \lor (name, merge(mdata, data)) \in MH(S)))\\
%%\end{axdef}
%
% ---- Abbauendes Prinzip
%
\begin{axdef}
  MH : \power (Abs\_Name \cross Abs\_Data) 
  \fun \power (Abs\_Name \cross Abs\_Data)\\
  \where
    \forall S : \power (Abs\_Name \cross Abs\_Data); 
    x : (Abs\_Name \cross Abs\_Data)@\\ 
    \t1 x \in MH(S) \iff \\
    \t2 (x \in S \\
    \t2 \lor (\exists a : Abs\_Name; y : Abs\_Data; mdata : Abs\_Data @ \\
    \t3 (a,y)\in MH(S) \land \\
    \t4 (x = (a, merge(y, mdata))\\

    \t4 \lor x = (a, merge(mdata, y)))))\\
\end{axdef}
%
% ----------------------------
%
\begin{axdef}
  Aknows: \power (Cvs\_Uid \cross Cvs\_Passwd) \\
  Ainvents\_Base, Ainvents: \power (Abs\_Name \cross Abs\_Data)\\
    \where
    \forall x : Abs\_Name @ \forall xd, yd : Abs\_Data @ \\
    \t1 (x \mapsto xd \in Ainvents\_Base \land x \mapsto yd \in Ainvents\_Base)\\
    \iff  x \mapsto merge(xd, yd) \in Ainvents\_Base\\
    Ainvents = MH(Ainvents\_Base)\\
  \end{axdef}

The merge hull function allows us to adjust the set $Ainvents$ to both the
actions of users on files in the working copy and the synchronization of the
working copy with the repository. 

In the following, we assume that the IDs and passwords that the client uses to
log in are restricted to IDs and passwords the user ``knows'' and that the name
and data that a client adds or modifies are restricted to names and data that
the user ``invents''.
\begin{zed}
%
% Login
%
  abs\_loginR == abs\_login \land \\
  \t1 [ \< uid? : Cvs\_Uid;passwd? : Cvs\_Passwd | \\
  \t2 (uid?, passwd?) \in Aknows] \> \\
\end{zed}
%
% Add
%
\begin{zed}
  abs\_addR == abs\_add \land \\
  \t1 [ newfiles? : ABS\_DATATAB | \\
  \t2 newfiles? \subseteq Ainvents] \\
\end{zed}
%
% Modify
%
\begin{zed}
  modifyR == modify \land \\
  \t1 [ \< filename? : Abs\_Name; newData? : Abs\_Data | \\ 
  \t2 filename? \mapsto newData? \in Ainvents] \> \\
\end{zed}

For the commit operation we restrict the security model on users without
administrative permission. Possibly the security statements can be weakened so
that they can be proven without this restriction. We require trust in users
getting administrative rights. Therefore we prefer to give more precise security
statements while accepting the restriction of $abs\_ci$.
\begin{zed}
%
% Commit
%
  abs\_ciR == abs\_ci \land \\
  \t1 [ \< rep : ABS\_DATATAB; abs\_passwd : PASSWD\_TAB | \\
  \t2 \forall role : \dom abs\_passwd @ \\
  \t3 authtab~rep~(role, abs\_passwd~role)\\
  \t3 \neq cvs\_admin] \> \\
%
%
\end{zed}
The schemas are embedded in a relation, and the ``sequence of operations'' is
captured by a transitive closure over the union of the operation relations.
\begin{zed}
  AService == abs\_loginR \lor abs\_addR \lor abs\_ciR \lor abs\_up \lor abs\_cd \lor modifyR\\
  AbsState == ClientState \land RepositoryState \\
  AtransR == \{ AService @ ( \theta AbsState, \theta AbsState ') \} \star
\end{zed}
We define InitAbsState as a restricted variant of the AbsState schema. For
example, we require that the working copy is initially empty.
%
%
  \begin{zed}
    InitAbsState == AbsState \land \\
    \t1 [ wc: ABS\_DATATAB; rep : ABS\_DATATAB; \\
    \t2 wc\_uidtab: ABS\_UIDTAB; abs\_passwd : PASSWD\_TAB| \\
    \t2 wc = \emptyset \land \\
    \t2 wc\_uidtab = \emptyset \land \\
    \t2 abs\_passwd \neq \emptyset \land \\
    \t2 (\forall x : abs\_passwd @ x \in Aknows)\land \\
    \t2 (\forall x : Aknows @ authtab~rep~x \neq cvs\_admin)]\\
  \end{zed}  
  With the schema $Areachable$ we relate the hull of commands to the desired
  initial states.
  \begin{zed}
    Areachable == InitAbsState \dres AtransR \\
  \end{zed}
  Now we can postulate the following list of security
  properties of our architecture:
\begin{enumerate}
\item Any sequence of CVS operations starting from an empty working copy does
  not lead to a working copy with data from the repository that the user has no
  permission to access in the repository (except if he was able to ``invent''
  it).

%
%
%
%%prerel AReadProp
\begin{doc}{3}
  \begin{axdef}
    AReadProp \_ : \power (ABS\_DATATAB \cross ABS\_DATATAB \cross ABS\_PERMTAB) \\
    \where \forall wc : ABS\_DATATAB; rep : ABS\_DATATAB; rep\_permtab :
    ABS\_PERMTAB @ \\
    \t1 AReadProp(wc, rep, rep\_permtab) \iff \\
    \t2 (\forall n : \dom wc @ \\
    \t3(n,wc(n)) \in Ainvents \lor \\
    \t3 ((\exists m: Aknows @ \\
    \t4 (rep\_permtab(n),authtab(rep)(m))\in cvs\_perm\_order)))\\
%    \t3 ((wc(n) = rep(n)) \implies (\exists m: Aknows @ \\
  \end{axdef}
\end{doc}
%
%
%
%  
\begin{zed}
    AReadSec == \forall AbsState @ \forall  AbsState ' @ \\
    \t1 [\Delta AbsState |(\theta AbsState,\theta AbsState ')\in Areachable \\
    \t2 \implies AReadProp(wc', rep', rep\_permtab')]\\
  \end{zed}
%
%
%
\item No sequence of CVS operations leads to a repository with ``invented''
  data, except the user ``knows'' the ID and password that corresponds to the
  data.
%
%
%%prerel AWriteProp
\begin{doc}{3}
  \begin{axdef}
    AWriteProp \_ : \power (ABS\_DATATAB \cross ABS\_DATATAB \cross ABS\_PERMTAB) \\
    \where \forall rep : ABS\_DATATAB; rep' : ABS\_DATATAB; rep\_permtab' :
    ABS\_PERMTAB @ \\
    \t1 AWriteProp(rep, rep', rep\_permtab') \iff \\
    \t2 (\forall f : \dom rep' @\\
    \t2 (rep(f)\neq rep'(f)\\
    \t2 \implies (\exists m: Aknows @ \\
    \t2 (rep\_permtab'(f),authtab(rep')(m)) \in cvs\_perm\_order)))\\
  \end{axdef}
\end{doc}
  \begin{zed}
%
%
    AWriteSec == \forall AbsState @ \forall  AbsState ' @ \\
    \t1 [\Delta AbsState |(\theta AbsState,\theta AbsState ')\in Areachable \\
    \t2 \implies AWriteProp(rep, rep', rep\_permtab')]\\ 
  \end{zed}
%
  If we can grant that the dominant permission $cvs\_admin$ cannot be obtained
  by any tuple of $Cvs\_Uid \cross Cvs\_Passwd$ from $(authtab~rep)$ (which is
  equal to the parsed authentication file $abs\_cvsauth$), then we can show that
  AWriteSec holds.
%
 
  The base cases for this proof can be shown trivially using the following
  lemma:

%
  \[
  \bigwedge u~v.\;
%
%
  \left.
    \begin{array}{l}
      u : ABS\_DATATAB\\
      v : ABS\_PERMTAB\\
    \end{array}
  \right\}
%
%
  \Longrightarrow
  AWriteProp(u, u, v)
  \]
%

  The induction step of our proof is characterized by a case distinction on the
  commands. The operation $abs\_ci$ is the only Operation which changes the
  repository state.
%

    For all other commands, the goal can be trivially solved with the induction
    hypothesis. The fact $\Xi RepositoryState$ implies invariance of $rep$ and
    $rep\_permtab$. So the goal can be unified with the induction hypothesis
    after substituting $rep$ and $rep\_permtab$. 
%
    
    The proof for the command commit uses a case distinction on the top level.

    \begin{center}
      \setlength\unitlength{1cm}
      \begin{picture}(7,5)
        \thicklines
%
% Obere Verzweigung
%
        \put(2.6,4.7){\line(4,-5){1}}
        \put(2.6,4.7){\line(-4,-5){1}}
%
% (1a)
%
        \put(0.5, 3){case (1a)}
        \put(0, 2.5){$f \notin \dom y.rep$}
%
% (1b)
%
        \put(3.2, 3){case (1b)}
        \put(3.0, 2.5){$f \in \dom y.rep$}
%
% untere Verzweigung
%
        \put(4.0,2.2){\line(4,-5){1}}
        \put(4.0,2.2){\line(-4,-5){1}}
%
% (2)
%
        \put(1.9, 0.5){case (2)}
        \put(1.0, 0){$y.rep~f \neq z.rep~f$}
%
% (3)
%
        \put(4.7, 0.5){case (3)}
        \put(4.4, 0){$y.rep~f = z.rep~f$}
%
%
      \end{picture}
    \end{center}
%
    The first two cases require a differentiated analysis of the behaviour of
    the command $abs\_ci$. The third case can be shown using the induction
    hypothesis as a main argument. Cases one and two use the lemma $LoginAknows$
    which allows us to show that an id and its corresponding password are
    members of the set $Aknows$.
%
% Lemma LoginAknows
%
    \[
    \bigwedge x.\;
    \left.  
      \parbox{0.25\textwidth}{$
        \begin{array}{l}   
          InitAbsState \\
          AtransR \\
          x \in \dom abs\_passwd' \\
        \end{array}$}
    \right\}
%
%
%
    \Longrightarrow
    (x, abs\_passwd~x) \in Aknows
    \]
% 

    If the administrator changed the file $abs\_cvsauth$, it would be
    possible that he meanwhile reduces his permission status. It can happen that
    a valid permission exists before changing a file, but vanishes afterwards.
    This makes a proof of our goal $AWriteSec$ allowing administrative
    permissions in $abs\_ci$ impossible. Alternatives to our decision which
    involve the administrator are e.g.\ weakening the statement to: ``there
    exists a state where the user has the permission to change the file'', or we
    could restrict the administrator to only extend the set of existing
    role-permission assignments.

%
%
  
\item Moreover, there is the obvious safety-property that the user can access
  all data he has the permission to.  With access meaning reading (i.e.\ cvs
  update) the repository and writing (i.e.\ cvs commit) the repository, and with
  ignoring merges, we can formulate this property as follows:
  \begin{schema}{AlwaysUpdateProp}
    wfiles: \power Abs\_Name \\
    wc': ABS\_DATATAB \\
    rep: ABS\_DATATAB \\
    rep\_permtab: ABS\_PERMTAB \\
    abs\_uid: Cvs\_Uid \\
    abs\_pwd: Cvs\_Passwd \\
    \where
    \forall f: wfiles | f \in \dom rep \land f \in \dom wc' \land \\
    \t1 (rep\_permtab(f), authtab(rep)(abs\_uid, abs\_pwd)) \in cvs\_perm\_order
    @ \\ 
    \t2 wc'(f) = rep(f) \\
  \end{schema}
  \begin{zed}
    AlwaysUpdate == \forall ClientState; ClientState '; RepositoryState; \\
    \t1 RepositoryState ' @ abs\_up \implies AlwaysUpdateProp \\
  \end{zed}
  The commit operation is analogous.
\end{enumerate}
 
A proof of safety properties can be based on an induction over the transitive
closure; in the base-case, a case distinction has to be made: either the user
``knows'' the authentication for the $cvs\_admin$ (then he can access everything
anyway) or not (then the authentication table remains constant and the user will
always have the same access rights. 

Due to the refinement, these safety properties carry over to the implementation
level. To be more precise, for all compatible states (i.e.\ the $Cvs\_Client$ is
not $root$ has normal access to his home account and his working copy in it),
any combination of cvs commands on the implementation level will not violate the
security properties.  However, this proof breaks down when the cvs commands can be
interleaved with \unix-commands. For this purpose, an own induction over refined
states has to be done, which is subject of the next section.


%\subsection{Functional Properties}

%\begin{itemize}
%\item checkout in allowed areas: $Init ; (cd | cvs\_login)* | cvs\_co$
%\item checkin in allowed areas: $Init ; (cd | cvs\_login)* | cvs\_ci$
%\item read/write of cvs\_auth if cvs\_admin
%\end{itemize}


\section[Implementation Architecture]{Properties of the Implementation
  Architecture}\label{sec:secprop2}

With respect to ``knows'' and ``invents'', we make the same assumptions as in
section~\ref{sec:secprop1}. Moreover, we have to assume specialties due to the
implementation architecture level: for instance, the user must have write access
to his own working copy, the working directory must point into the working copy,
etc.


\subsection{Security Properties}

\zsection[CVSServer]{ImplArchSec}\vspace{1ex}\noindent

We approach the formalization of the security properties in the same way as for
the abstract case.
\begin{axdef}
  Cknows: \power (Cvs\_Uid \cross Cvs\_Passwd) \\
  Cinvents: FILESYS\_TAB \\
\end{axdef}
\begin{zed}
  CService == cvs\_cd \lor cvs\_mkdir \lor cvs\_mkfile \lor cvs\_access \lor
  cvs\_write \lor cvs\_rm \\
  \t1 \lor cvs\_rmdir \lor cvs\_mv \lor cvs\_chown \lor cvs\_chmod \lor
  cvs\_setumask \\
  \t1 \lor cvs\_login \lor cvs\_co \lor cvs\_update \lor cvs\_ci \lor cvs\_add
  \\ 
  ConcState == ProcessState \land Cvs\_FileSystem \\
  CtransR == \{ CService @ ( \theta ConcState, \theta ConcState ') \} \star \\
\end{zed}
Then we postulate the following list of security properties of our
implementation architecture:
\begin{enumerate}
\item No sequence of CVS operations \emph{and} \unix-operations starting from an
  empty working copy leads to a working copy with data from the repository that
  the user has no permission to access in the repository (except if he was able
  to ``invent'' it).
  \begin{zed}
    InitConcState1== ConcState \land [ wcs\_attributes: CVS\_ATTR\_TAB | \\
    \t1 wcs\_attributes = \emptyset ] \\ 
    Creachable1 == CtransR \limg InitConcState1 \rimg \\
  \end{zed}
  \begin{schema}{CProp1}
%    uid: Uid \\
    wcs\_attributes: CVS\_ATTR\_TAB \\
    files: FILESYS\_TAB \\
    attributes: FILEATTR\_TAB \\
    \where
    \forall n: \dom wcs\_attributes @ (n, files(n)) \in Cinvents \lor \\
    \t1 ((files(n) = files((wcs\_attributes~n).rep)) \land (\exists m: Cknows @ \\
    \t2 has\_r\_access(cvsperm2uid(get\_auth\_tab(files)(m)), \\
    \t3 cvs\_rep \cat (wcs\_attributes~n).rep, attributes))) \\
  \end{schema}
  \begin{zed}
    CSec01 == \forall Creachable1 @  ConcState \land CProp1 \\
  \end{zed}
  
\item Any sequence of CVS-operations \emph{and} \unix-operations does not lead
  to a repository with ``invented'' data, except the user ``knows'' the id and
  password that corresponds to it.

  \begin{zed}
    Creachable2 == CtransR \limg ConcState \rimg \\
  \end{zed}
  \begin{schema}{CProp2}
%    uid: Uid \\
    wcs\_attributes: CVS\_ATTR\_TAB \\
    files: FILESYS\_TAB \\
    attributes: FILEATTR\_TAB \\
    \where
    \forall n: \dom files | cvs\_rep \prefix n @ (n, files(n)) \in Cinvents
    \implies \\
    \t1 (\exists m: Cknows @
    has\_w\_access(cvsperm2uid(get\_auth\_tab(files)(m)), \\ 
    \t2 cvs\_rep \cat (wcs\_attributes~n).rep, attributes)) \\
  \end{schema}
  \begin{zed}
    CSec2 == \forall Creachable2 @  ConcState \land CProp2 \\
  \end{zed}
  
\item No user can directly access or manipulate (chmod) the repository, except
  he is the system administrator ``root'' (this is essentially achieved by
  mapping CVS user IDs to \unix user IDs and group IDs that are disjoint from
  all user IDs of ``real'' users).
  
  \emph{Here, we have the choice to formulate this property as a general state
    invariant over the system state or over all possible traces of
    \unix-operations.}

  Formulating the property as a general state invariant:
  \begin{schema}{CProp3}
    uid: Uid \\
    cvs\_rep: Path \\
    attributes: FILEATTR\_TAB \\
    \where
    uid \notin \{ cvsperm2uid(cvs\_admin), root \} \\
    \lnot has\_w\_access(uid, cvs\_rep, attributes) \\
  \end{schema}
  \begin{zed}
    CSec3 == \forall ConcState @ ConcState \land CProp3 \\
  \end{zed}
  
\item The CVS administrator can withdraw the permissions for a user (after a
  suitable \cvscmd{update} of the authentication table); i.e.\ after a withdraw,
  the user does not get data into his working copy except by ``inventing'' it
  (e.g.\ the user cannot get data from the repository into his working copy).
  \label{item:withdraw}
  
  We define a relation $CWithdrawId$ that defines the withdrawal of a user id
  and a relation $ChangeWC$ which defines the possible actions of a user to
  alter his working copy.
  \begin{axdef}
    CWithdrawId: ConcState \rel ConcState \\
    ChangeWC: ConcState \rel ConcState \\
    \where
%    \forall a,b: ConcState @ (a,b) \in CWithdrawId \iff \\ 
%    \t1 b.files = a.files \oplus \{ cvs\_rep \cat \langle CVSROOT, cvsauth
%    \rangle \mapsto \\ 
%    \t2 Inl(data\_of(Cknows \ndres get\_auth\_tab(a.files))) \} \\ 
    \forall a,b: ConcState @ (a,b) \in CWithdrawId \iff \\ 
    \t1 b.files = a.files \oplus \{ cvs\_rep \cat \langle CVSROOT, cvsauth
    \rangle \mapsto \\ 
    \t2 Inl(data\_of(a.cvs\_passwd \ndres get\_auth\_tab(a.files))) \} \\ 

% the working copy in the post state contains data, that is not invented and
% wasn't in the working copy before any operation
    \forall a,b: ConcState @ (a,b) \in ChangeWC \iff (\forall n: \dom
    b.wcs\_attributes @ \\
    \t1 (n, b.files(n)) \notin Cinvents \\
    \t2 \land (n \notin \dom a.wcs\_attributes \lor a.files(n) \neq
    b.files(n))) \\
  \end{axdef}
  \begin{zed}
    CSec4 == [ a,b: ConcState | (a,b) \in CtransR \land (a,b) \in
    CWithdrawId ] \\
    \t1 \implies [b,c: ConcState | (b,c) \in CtransR \land (b,c) \notin
    ChangeWC ] \\
  \end{zed}
\end{enumerate}

Proofs have the same structure as in section~\ref{sec:secprop1}, but are far
more complex since concepts such as paths, the distinction between files and
directories, and their permissions (in particular when being created) are
involved.


%\begin{itemize}
%\item read, write, access attacks via UNIX commands on the repository (Approach
%  - use Invariants of Cvs\_FileSystem, show that access impossible on any state
%  fulfilling the invariant)
%\item checkout attacks in unallowed areas: Init ; (cd | cvs\_login)* | cvs\_co
%\item checkin attacks in unallowed areas: Init ; (cd | cvs\_login)* | cvs\_ci
%\item read/write attacks against cvs\_auth
%\end{itemize} 

%\subsection{Functional Properties}


\section[Perspective]{Setting both Security Models into Perspective}
\label{sec:analysis}

\zsection[SysArchSec,ImplArchSec]{CombinedSec}\vspace{1ex}\noindent

This section serves to discuss the relation between the system level security
and the implementation level security.

Some security properties cannot be expressed on both abstraction levels.  For
example, property~\ref{item:withdraw} of the implementation model cannot be
expressed on the abstract level.

On the implementation level, we could require the following property: No user
can access the working copy of other users --- except if the others set their
umask too liberally or the user is the sysadmin ``root''.  This property cannot
be expressed in our single user model!



\vspace{4ex} The following Z-section is only a virtual section that depends on
all other sections, and is therefore used to load the complete specification: \\
\zsection[Refinement,CombinedSec,SysArchConsistency,ImplArchConsistency]{all}


%%% Local Variables:
%%% TeX-master: "arch"
%%% fill-column:80
%%% x-symbol-8bits:nil
%%% End:
