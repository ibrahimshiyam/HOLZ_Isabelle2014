\documentclass[a4paper]{article}

\usepackage{fullpage}
\usepackage{zeta}

\newcommand{\zcomment}[1]{#1}

\title{Specification of a Class Scheduling System (CSS)}
\author{Burkhart Wolff and Frank Rittinger}

\begin{document}

\maketitle

This specification is part of an exercise for a lecture on software engineering.
The purpose of the \emph{Class Scheduling System} CSS is to calculate a schedule
for all sorts of lectures, seminars, etc. for our computer science department,
according to various constraints on rooms, overlapping classes that should be
taken in the same term, etc.


\section{Specification of the Basis-Entities} 

\zsection{basis}

\vspace{1ex}\noindent
We introduce some elementary data-types (basis and free types), and later give
schemas for the basis-entities, namely $DayBeginDuration$, $Instructor$, $Room$
and $EventOffer$.

%%macro \zcomment 1

\begin{zed}
  [TEXT, EventOfferID, RoomID, ClassScheduleID] \\
  DAY ::= Mo | Tu | We | Th | Fr | Sa | Su \\
  RANK ::= Prof | Pd | Dr | Dipl | Stud  \\
  RESEARCH\_GROUP ::= ST | PS | KI | DB | ME | AS | ML | AT | RA | CG \\
  ROOM\_FEATURE ::= Multimedia | Beamer | Projector | Computer \\
  EVENT\_TYPE & ::= & RequiredClass3 \\
  & | & RequiredClass5 \\
  & | & RequiredClass7 \\
  & | & SpecialClass \\
  & | & Lab | Seminar | Exercises \\
\end{zed}

The natural way to define $RequiredClass$ would be the following definition, but
this is not yet supported by HOL-Z!
\zcomment{
  \begin{zed}
    EVENT\_TYPE ::= RequiredClass \ldata \nat \rdata | SpecialClass | Lab |
    Seminar | Exercises
  \end{zed}}
Additionally we need a predicate that decides whether the event type is
$RequiredClass_i$ or something different.
\begin{axdef}
  isRequiredClass: \power EVENT\_TYPE \\
  \where
  isRequiredClass = \{ RequiredClass3, RequiredClass5, RequiredClass7 \} \\
\end{axdef}

\paragraph{Remarks.} $RequiredClass \ldata \nat \rdata$ expresses the
semester in which the event must take place; this will be important for
computing the conflicts in the schedules!  Note also that the free types could
contain further constants, e.g.~$Videorecorder$ in $ROOM\_FEATURE$.

The following are the schemas for the four basis-entities, including constraints
on data; of course, alternative (equivalent) formulations are possible.
\begin{schema}{DayBeginDuration}
  Day: DAY \\
  Begin: ( 8 \upto 17 ) \setminus \{ 13 \} \\
  Duration: \nat
  \where
  Day \neq Sa \land Day \neq Su \\
  Begin < 13 \implies Begin + Duration \leq 13 \\
  Begin > 13 \implies Begin + Duration \leq 18 \\
  Duration \geq 1
\end{schema}
\begin{schema}{Instructor}
  Name: TEXT  \\
  Rank: RANK \\
  Group: RESEARCH\_GROUP
\end{schema}
\begin{schema}{Room}
  Capacity: \nat  \\
  Features: \power ROOM\_FEATURE 
  \where
  Capacity \geq 3
\end{schema}

\noindent
Note that the $RoomID$s are not explicitly included in the schema, but
are used in the table of $Room$s below.

In order to give a schema for $EventOffer$, we first introduce an auxiliary
function which computes the ``set of hours'' contained in a given
$DayBeginDuration$, so that we can express the constraint that all
$PreferredDayBeginDurations$ must have the same overall duration.
\begin{axdef}
  TimeFrame: \power DayBeginDuration \fun \power (DAY \cross \nat) \\
  \where 
  \forall dbd : \power DayBeginDuration @ TimeFrame(dbd) =
  \{day: DAY; x: \nat | \\
  \t1 \exists t: dbd @ (day = t.Day) \land (t.Begin \leq x) \land (x
  < t.Begin + t.Duration)\}
\end{axdef}
\begin{schema}{EventOffer}
  Instructors: \power Instructor \\
  Title: TEXT \\
  Type: EVENT\_TYPE \\
  Features: \power ROOM\_FEATURE \\
  ExpectedAttendants: \nat \\
  PreferredDayBeginDurations: \power (\power DayBeginDuration) \\
  \where
  Instructors \neq \emptyset \\
  PreferredDayBeginDurations \neq \emptyset \land PreferredDayBeginDurations
  \neq \{\emptyset\} \\ 
  ExpectedAttendants \geq 3 \\
  \forall x, y: PreferredDayBeginDurations @ \#(TimeFrame~x) = \#(TimeFrame~y)
  \\ 
  \forall d: PreferredDayBeginDurations @ \forall x,y: d @ x \neq y \implies
  x.Day \neq y.Day \\ 
\end{schema}


\section{Specification of the Relations on the Entities} 

\zsection[basis]{relations}

\vspace{1ex}\noindent
A $PreferredDayBeginDurations$ is a set of $DayBeginDuration$s, and an
$EventAssignment$ consists of a triple of an $EventOffer$ (actually, its ID,
$EventOfferID$), a $PreferredDayBeginDurations$, and a collection (set, list,
etc.) of $Room$s (actually, their IDs, $RoomID$), which are assigned to an
event.
\begin{zed}
  PreferredDayBeginDuration ==  \power DayBeginDuration \\
\end{zed}

\noindent
For each $EventOfferID$, an $EventAssignment$ associates a $DayBeginDuration$
from the set of $DayBeginDuration$s to a particular room (characterized by its
$RoomID$). There is a natural linear ordering on $DayBeginDuration$s based on
their $Day$ and $Begin$, such that an element of the $DayBeginDuration$s-set may
be associated to a room in the room sequence $S$ according its position.
\begin{zed} 
  EventAssignment == \{ea : EventOfferID \cross \power DayBeginDuration \cross
  \seq RoomID | \\ 
  \t1 \exists eoid:EventOfferID @ \exists S:\power DayBeginDuration @ \exists
  ridS:\seq RoomID @ \\ 
  \t2 ((eoid, S,ridS) = ea) \land (\# S = \# ridS) \} \\

  EventOfferTable == EventOfferID \pinj EventOffer \\

  RoomTable == RoomID \pinj Room \\
\end{zed}
  
\noindent
Now we turn to the $ClassSchedule$-tables.
\begin{zed}
  ClassScheduleTable == ClassScheduleID \pinj (EventOfferID \pinj \power
  (DayBeginDuration \cross RoomID)) \\
\end{zed}

\noindent
It will turn out to be useful to have the following abbreviation:
\begin{zed} 
  ClassSchedule == EventOfferID \pinj \power (DayBeginDuration \cross RoomID) \\
\end{zed} 


\section{Specification of the $ClassSchedule$ table} 

\zsection[relations]{classSchedule}

\vspace{1ex}\noindent
Not all elements of the $ClassSchedule$ table are `legal', as there are a number
of constraints that must be satisfied.  Given the tables of $EventOffer$s and
$Room$s, we define a function $ClassSchedules$, which identifies the set of
legal class-schedules as follows.
\begin{axdef}
  ClassSchedules: EventOfferTable \cross RoomTable \fun \power ClassSchedule
  \where
  \forall eo : EventOfferTable; r : RoomTable @ \\
  \t1 ClassSchedules (eo,r) = \{ s : ClassSchedule | \dom s = \dom eo \land \\
  \t2 ( \forall eoid : \dom s @ \exists dbd:
  (eo~eoid).PreferredDayBeginDurations @ \\ 
  \t3 ( \{x : s~eoid @ first~x\} = dbd ) \land ( \forall x: s~eoid @  second~x
  \in \dom r)) \} \\
\end{axdef}

\noindent
The second formula constraining the class-schedule $s$ expresses that the
$DayBeginDuration$s must be contained in the $PreferredDayBeginDurations$
(i.e.~$\{x : s~eoid @ first~x\} = dbd$), and that the $Room$s must be defined
(i.e.~$\forall x : s~eoid @ second~x \in \dom r$).

The following auxiliary function allows us to compute a schedule of the rooms,
i.e.~a $RoomOccupancyPlan$, from a given $ClassSchedule$ and a $RoomID$.
\begin{axdef}
  RoomOccupancyPlan: ClassSchedule \cross RoomID \fun \power (EventOfferID
  \cross DayBeginDuration) \\
  \where
  \forall s: ClassSchedule; rid: RoomID @ \\
  \t1 RoomOccupancyPlan (s,rid) = \{eoid: EventOfferID; dbd: DayBeginDuration |
  \\ 
  \t2 (dbd,rid) \in s~eoid\}
\end{axdef}

\noindent
We can also define a similar schedule for the instructors:
\begin{axdef}
  InstructorsSchedule: EventOfferTable \cross ClassSchedule \cross Instructor
  \fun \power (EventOfferID \cross DayBeginDuration) \\ 
  \where
  \forall eot: EventOfferTable ; s: ClassSchedule; i: Instructor @ \\
  \t1 InstructorsSchedule (eot,s,i) = \{eoid: EventOfferID; dbd:
  DayBeginDuration |  \exists r: RoomID @ \\ 
  \t2 (dbd,r) \in s~eoid \land i \in (eot~eoid).Instructors\}
\end{axdef}


\section{Specification of the Conflicts} 
\zsection[classSchedule]{conflicts}

\vspace{1ex}\noindent
We now formalize auxiliary functions that compute room-features conflicts
(violations of the requirements enforced on a room) and instructor conflicts
(violations of the fact that an instructor cannot take part in two or more
events during the same time-frame).

The main idea is to compute the cardinality of sets of tuples, which
characterize the conflicts.

The system assigns, within a schedule, a pair of room and date to an event.  A
room-features conflict occurs when the required features (e.g.~Beamer and
Projector) of an event are not a subset of the features of the assigned room, or
when the number of expected attendants is higher than the capacity of the room.

% %%inrel \nsubseteq
% %% \begin{axdef}[X]
% %%    \_ \nsubseteq \_ : \assumed \power(\power X \cross \power X)
% %% \end{axdef}

We define the Z-function $RoomFeatureConflicts$ as follows:
\begin{axdef}
  RoomFeatureConflicts: ClassSchedule \cross RoomTable \cross EventOfferTable
  \fun \nat \\
  \where
  \forall s: ClassSchedule; r: RoomTable; eo: EventOfferTable @ \\
  RoomFeatureConflicts(s,r,eo) = \\
  \t1 \# \{eoid: \dom s; rid: \dom r | \exists dbd: DayBeginDuration @ (dbd,rid)
  \in (s~eoid) \land \\
  \t2 (\lnot(((eo~eoid).Features) \subseteq (r~rid).Features) \\
  \t2 \lor ((eo~eoid).ExpectedAttendants) > (r~rid).Capacity)\} \\
\end{axdef}

\noindent
An instructor conflict occurs when a class-schedule requires an instructor to
take part in two (or more) events at the same time (i.e.~during the same
time-frame). In order to define formally the function $InstructorConflicts$ we
exploit the auxiliary function $TimeFrame$.
\begin{axdef}
  InstructorConflicts: ClassSchedule \cross EventOfferTable \fun \nat \\
  \where
  \forall s: ClassSchedule; eo: EventOfferTable @ \\
  InstructorConflicts(s, eo) = \\
  \# \{eoid1: \dom s; eoid2: \dom s; i: Instructor | \\
  \t1 eoid1 \neq eoid2 \\
  \t1 \land i \in (eo~eoid1).Instructors \\
  \t1 \land i \in (eo~eoid2).Instructors \\
  \t1 \land TimeFrame \{ sid: s~eoid1 @ first~sid \} \cap TimeFrame \{ sid:
  s~eoid2 @ first~sid \} \neq \emptyset \} \div 2 \\
\end{axdef}

\noindent
{\bf Remark}: we have to divide by 2 since $eoid1$ and $eoid2$ both occur in $s:
ClassSchedule$, so that the instructor conflicts within the $DayBeginDurations$
of $eoid1$ and $eoid2$ are counted twice.

We now formalize auxiliary functions that compute further conflicts: room
conflicts (violations of the fact that a room cannot be assigned to more than
one event during a time-frame) and required-class conflicts (violations of the
fact that no two classes required for the same semester $n$ can take place
during the same time-frame).

As for the other conflicts, the main idea is to compute the cardinality of sets
of tuples, which characterize the conflicts.

\begin{axdef}
  RoomConflicts: ClassSchedule \cross RoomTable \fun \nat \\
  \where
  \forall s:ClassSchedule; r: RoomTable @ RoomConflicts(s,r) = \\
  \t1 \# \{eoid1: \dom s; dbd1: DayBeginDuration; \\
  \t2 eoid2: \dom s; dbd2: DayBeginDuration; rid: \dom r | \\
  \t3 eoid1 \neq eoid2 \land \\
  \t3 (dbd1,rid) \in s~eoid1 \land \\
  \t3 (dbd2,rid) \in s~eoid2 \land \\
  \t3 TimeFrame(\{dbd1\}) \cap TimeFrame(\{dbd2\}) \neq \emptyset\} \div 2 \\
\end{axdef}

\noindent
{\bf Remark}: as for $InstructorConflicts$, we divide by 2, since $eoid1$ and
$eoid2$ both occur in $s: ClassSchedule$, so that the room conflicts between the
$DayBeginDurations$ of $eoid1$ and $eoid2$ are counted twice. A similar
observation holds for the function $RequiredClassConflicts$.
\begin{axdef}
  RequiredClassConflicts: ClassSchedule \cross EventOfferTable \fun \nat \\
  \where 
  \forall s:ClassSchedule; eo: EventOfferTable @ RequiredClassConflicts(s,eo) = \\
  \t1 \# \{eoid1: \dom s; dbd1: DayBeginDuration; \\
  \t2 eoid2: \dom s; dbd2: DayBeginDuration | \\
  \t3 eoid1 \neq eoid2 \land \\
  \t3 (eo~eoid1).Type = (eo~eoid2).Type \land \\
  \t3 (((eo~eoid1).Type) \in isRequiredClass) \land \\ 
  \t3 (\exists rid1: RoomID @ (dbd1,rid1) \in s~eoid1) \land \\
  \t3 (\exists rid2: RoomID @ (dbd2,rid2) \in s~eoid2) \land\\
  \t3 TimeFrame(\{dbd1\}) \cap TimeFrame(\{dbd2\}) \neq \emptyset\} \div 2 \\
\end{axdef}


\section{Specification of the Planning-Algorithm} 

\zsection[conflicts]{planner}

\vspace{1ex}\noindent 
We have defined the four most important functions for counting conflicts and we
can now define a lexicographic ordering on conflict-tuples in a class-schedule;
this will allow us to determine if a schedule is `better' than another
one.\footnote{Of course, in a more refined specification one may wish to
  consider also other conflicts that can occur in a class-schedule
  (e.g.~conflicts between required classes and seminars); the definition of the
  corresponding counting functions will be along the lines of the four we
  presented.}

\begin{zed}
  ConflictTuple == (\nat \cross \nat \cross \nat \cross \nat) \\
\end{zed}
\begin{axdef}
  Conflicts: ClassSchedule \cross EventOfferTable \cross RoomTable \fun
  ConflictTuple \\
  \where
  \forall s:ClassSchedule; eo:EventOfferTable; r:RoomTable @ \\
  \t1 Conflicts(s,eo,r) = (RequiredClassConflicts(s,eo), RoomConflicts(s,r), \\
  \t2 InstructorConflicts(s,eo), RoomFeatureConflicts(s,r,eo)) \\
\end{axdef}

\zrelation{(\_ lessConflicts \_)}
\begin{axdef}
  \_ lessConflicts \_: ConflictTuple \rel ConflictTuple \\
  \where
  \forall a,b,c,d,a',b',c',d': \nat @ \\
  \t1 (a,b,c,d)~lessConflicts~(a',b',c',d') \iff \\
  \t2 a < a' \lor (a = a' \land (b < b' \lor (b = b' \land (c < c' \lor(c = c'
  \land d \leq d'))))) \\
\end{axdef}

\noindent
{\bf Remark:} Note that we have chosen to order conflicts by giving
required-class conflicts the highest ranking, followed by room conflicts,
instructor conflicts and room-features conflicts. Of course, other orderings are
possible.

Given $lessConflicts$, we define an ordering on class-schedules by means of the
function $lessClassSchedules$ as follows. Note that ours is a
\emph{quasi-ordering} (i.e.~reflexive, transitive and antisymmetric), so that
sorting using $lessClassSchedules$ will never be unique.
\begin{axdef}
  lessClassSchedules : EventOfferTable \cross RoomTable \fun (ClassSchedule \rel
  ClassSchedule) \\
  \where
  \forall eo:EventOfferTable; r:RoomTable @ \\
  \t1 lessClassSchedules (eo,r) = \{ s: ClassSchedule; s': ClassSchedule | \\
  \t2 Conflicts(s,eo,r)~lessConflicts~Conflicts(s',eo,r) \} \\
\end{axdef}

\noindent
The following auxiliary function $isSorted$ allows us to define a function
$Hitlists$, which yields an ordered, $k$-ary list of class-schedules having less
than $max$ conflicts.
\begin{axdef}
  isSorted: (ClassSchedule \rel ClassSchedule) \fun \power (\seq ClassSchedule) 
  \where
  \forall ord: (ClassSchedule \rel ClassSchedule) @ \\
  \t1 isSorted~ord = \{ cs: \seq ClassSchedule | \forall i: \dom cs; j: \dom cs
  @ i \leq j \implies ((cs~i),(cs~j)) \in ord\} \\
\end{axdef}
\begin{axdef}
  Hitlists: EventOfferTable \cross RoomTable \cross ConflictTuple \cross \nat
  \fun \power (\seq ClassSchedule) \\
  \where 
  \forall eo:EventOfferTable; r:RoomTable; max:ConflictTuple; k:\nat @ \\
  \t1 Hitlists(eo,r,max,k) = \{sseq: \seq (ClassSchedules(eo,r)) | \\
  \t2 (\forall s: \ran sseq @ Conflicts(s,eo,r)~lessConflicts~max \land \\
  \t2 sseq \in isSorted~(lessClassSchedules(eo,r))) @ (1 \upto k) \dres sseq\} \\
\end{axdef}



\section{The System-State of CSS}

\zsection[planner]{state}

\vspace{1ex}\noindent
The following schema $ClassSchedulingSystem$ provides a preliminary model
(version V1) of the state-space of our class-scheduling-system, which we will
extend and refine in the final specification.
\begin{zed}
  INIT ::= Valid | Invalid \\
\end{zed}
\begin{schema}{ClassSchedulingSystem}
  eventOfferTable: EventOfferTable \\
  roomTable: RoomTable \\
  classScheduleTable: ClassScheduleID \pfun ClassSchedule \\
  classScheduleTableInit: INIT \\
  actVersion: ClassScheduleID \\
  maxVersions: \nat \\
  maxConflicts: ConflictTuple \\
  \where
  classScheduleTable \in ClassScheduleID \pfun \bigcup \{eosub: \power (\dom
  eventOfferTable) @ \\
  \t1 ClassSchedules(eosub \dres eventOfferTable,roomTable)\} \\
\end{schema}

\noindent
The slightly complex body of the schema is due to our decision that
$ClassSchedules$ requires the domain of a class-schedule to be equal to
the domain of the $eventOfferTable$, i.e.~$ClassSchedules$ admits only
\emph{complete} solutions as ``legal''. This had to be relaxed to the
set of all partial approximations.

To initialize $ClassSchedulingSystem$ we define:
%\begin{schema}{InitClassSchedulingSystem}
%  ClassSchedulingSystem \\
%  \where
%  classScheduleTableInit = Invalid \\
%  eventOfferTable = \emptyset \\
%  roomTable = \emptyset \\
%  maxVersions = 3 \\
%  maxConflicts = (3,3,3,3) \\
%\end{schema}
\noindent
$classScheduleTableInit$ is a flag that indicates whether class-schedules have
been generated or not.

Note that the equation for $actVersion$ has been omitted. This mean that
$InitClassSchedulingSystem$ admits an arbitrary value for it; this is legal
because of our strategy of setting and unsetting $classScheduleTableInit$.


\section{The Operations of the Class-Scheduling-System}
\zsection[state]{operations}

\vspace{1ex}\noindent 
In order to simplify the definitions of the operations, we introduce an
auxiliary function $reduceSchedule$ that takes a class-schedule and eliminates
all entries containing $RoomId$s or $EventOfferId$s that are not contained in
two sets given as input.
\begin{axdef}
  reduceSchedule: \power EventOfferID \cross \power RoomID \fun ClassSchedule
  \fun ClassSchedule \\ 
  \where
  \forall eoids: \power EventOfferID @ \forall rids: \power RoomID @ \forall cs:
  ClassSchedule @ \\
  \t1 reduceSchedule(eoids, rids) (cs) = \{e:eoids; S: \power (DayBeginDuration
  \cross rids) | e \mapsto S \in cs \} \\
\end{axdef}

\begin{schema}{AddRoom}
  \Delta ClassSchedulingSystem \\
  r?: Room \\
  \where 
  \exists rid: RoomID @ (rid \notin \dom roomTable) \land (roomTable' =
  roomTable \cup \{rid \mapsto r?\}) \\ 
  eventOfferTable' = eventOfferTable \\
  classScheduleTable' = classScheduleTable \\
  classScheduleTableInit = Valid \\
  actVersion' =  actVersion \\
  maxVersions' = maxVersions \\
  maxConflicts' = maxConflicts
\end{schema}
\begin{schema}{DeleteRoom}
  \Delta ClassSchedulingSystem \\
  rid?: RoomID \\
  \where
  rid? \in \dom roomTable \\
  roomTable' = roomTable \setminus \{rid? \mapsto roomTable~rid?\} \\
  eventOfferTable' = eventOfferTable \\
  classScheduleTable' = classScheduleTable \comp reduceSchedule(\dom
  eventOfferTable, \dom roomTable') \\ 
  classScheduleTableInit'= Valid \\
  actVersion' =  actVersion \\
  maxVersions' = maxVersions \\
  maxConflicts' = maxConflicts
\end{schema}
\begin{schema}{UpdateRoom}
  \Delta ClassSchedulingSystem \\
  rid?: RoomID \\
  r?: Room \\
  \where
  rid? \in \dom roomTable \\
  roomTable' = roomTable \oplus \{rid?  \mapsto r?\} \\
  eventOfferTable' = eventOfferTable \\
  classScheduleTable' = classScheduleTable \comp reduceSchedule(\dom
  eventOfferTable, \dom roomTable \setminus \{rid?\}) \\
  classScheduleTableInit' = Valid \\
  actVersion' =  actVersion \\
  maxVersions' = maxVersions \\
  maxConflicts' = maxConflicts \\
\end{schema}
\begin{schema}{AddEventOffer}
  \Delta ClassSchedulingSystem \\
  eo?: EventOffer \\
  \where 
  \exists eoid: EventOfferID @ (eoid \notin \dom eventOfferTable) \land \\
  \t1 (eventOfferTable' = eventOfferTable \cup \{eoid \mapsto eo?\}) \\ 
  roomTable' = roomTable \\
  classScheduleTable' = classScheduleTable \\
  classScheduleTableInit = Valid \\
  actVersion' =  actVersion \\
  maxVersions' = maxVersions \\
  maxConflicts' = maxConflicts \\
\end{schema}
\begin{schema}{DeleteEventOffer}
  \Delta ClassSchedulingSystem \\
  eoid?: EventOfferID \\
  \where
  eoid? \in \dom eventOfferTable \\
  eventOfferTable' = eventOfferTable \setminus \{eoid? \mapsto
  eventOfferTable~eoid?\} \\
  roomTable' = roomTable \\
  classScheduleTable' = classScheduleTable \comp reduceSchedule(\dom
  eventOfferTable', \dom roomTable) \\
  classScheduleTableInit = Valid \\
  actVersion' =  actVersion \\
  maxVersions' = maxVersions \\
  maxConflicts' = maxConflicts \\
\end{schema}
\begin{schema}{UpdateEventOffer}
  \Delta ClassSchedulingSystem \\
  eoid?: EventOfferID \\
  e?: EventOffer \\
  \where
  (eoid? \in \dom eventOfferTable) \\
  eventOfferTable' = eventOfferTable \setminus \{eoid? \mapsto
  eventOfferTable~eoid?\} \cup \{eoid? \mapsto e?\} \\
  roomTable' = roomTable \\
  classScheduleTable' = classScheduleTable \comp reduceSchedule(\dom
  eventOfferTable \setminus \{eoid?\}, \dom roomTable) \\
  classScheduleTableInit' = Valid \\
  actVersion' =  actVersion \\
  maxVersions' = maxVersions \\
  maxConflicts' = maxConflicts \\
\end{schema}

\noindent
We now formalize the operations for creating class-schedules and for extending
class-schedule-versions by sets of event-offers to be scheduled.

\vspace{1ex}
\zsection[state]{creation}
\vspace{1ex}

\begin{schema}{CreateClassSchedules}
  \Delta ClassSchedulingSystem \\
  eoid?: \power EventOfferID \\
  \where
  \exists sol: ClassScheduleID \pfun ClassSchedules(eoid? \dres eventOfferTable,
  roomTable) @ \\ 
  \t1 \exists hit: Hitlists(eoid? \dres eventOfferTable, roomTable,
  maxConflicts, maxVersions) @ \\ 
  \t2 \ran sol = \ran hit \land \\
  \t2 classScheduleTable' = sol \land \\
  \t2 classScheduleTable'~actVersion' = hit(1) \\
  classScheduleTableInit' = Valid \\
  eventOfferTable' = eventOfferTable \\
  roomTable' = roomTable \\
%actVersion' = actVersion \\
  maxVersions' = maxVersions \\
  maxConflicts' = maxConflicts \\
\end{schema}

\noindent
The following operation $AddEventsInClassSchedule$ can be seen as a more
technical variant of the operation above: it extends the actual version of the
class-schedule with a set of event-offers (given as input).
\begin{schema}{AddEventsInClassSchedule}
  \Delta ClassSchedulingSystem \\
  eoid?: \power EventOfferID \\
  \where
  eoid? \cap \dom (classScheduleTable~actVersion) = \emptyset \\
  \exists hit: Hitlists((eoid? \cup \dom (classScheduleTable~actVersion)) \dres
  eventOfferTable, \\ 
  \t1 roomTable,maxConflicts,maxVersions) @ \\
  \t2 \exists sol: ClassScheduleID \pfun ClassSchedule @ \\
  \t3 (\dom sol \cap \dom classScheduleTable = \emptyset) \land \\
  \t3 (\forall i: (1 \upto \# hit) @ classScheduleTable~actVersion \subseteq
  (hit~i) \land \\ 
  \t3 (hit~i \in \ran sol)) \land \\
  \t3 classScheduleTable' = classScheduleTable \cup sol \\
  classScheduleTableInit' = Valid \\
  eventOfferTable' = eventOfferTable \\
  roomTable' = roomTable \\
  actVersion' = actVersion \\
  maxVersions' = maxVersions \\
  maxConflicts' = maxConflicts \\
\end{schema}

\noindent
We conclude by giving an operation for removing events from the actual
class-schedule version (i.e.~from the class-schedule actually selected), and an
operation for removing events from all class-schedules.
\begin{schema}{RemoveEventsFromActVersion}
  \Delta ClassSchedulingSystem \\
  eoid?: \power EventOfferID \\
  \where
  eoid? \subseteq \dom (classScheduleTable~actVersion) \\
  \dom classScheduleTable' = \dom classScheduleTable \\
  \forall csid: \dom classScheduleTable @ \\
  \t1 (csid \neq actVersion) \implies (classScheduleTable'~csid =
  classScheduleTable~csid) \\
  (classScheduleTable'~actVersion) = (eoid? \ndres
  (classScheduleTable~actVersion)) \\
  classScheduleTableInit' = Valid \\
  eventOfferTable' = eventOfferTable \\
  roomTable' = roomTable \\
  actVersion' = actVersion \\
  maxVersions' = maxVersions \\
  maxConflicts' = maxConflicts \\
\end{schema}
\begin{schema}{RemoveEventsFromClassSchedules}
  \Delta ClassSchedulingSystem \\
  eoid?: \power EventOfferID \\
  \where
  eoid? \subseteq \dom (classScheduleTable~actVersion) \\
  \forall csid: \dom classScheduleTable @ \\
  \t1 (classScheduleTable'~csid) = (eoid? \ndres
  (classScheduleTable~csid)) \\ 
  classScheduleTableInit' = Valid \\
  eventOfferTable' = eventOfferTable \\
  roomTable' = roomTable \\
  actVersion' = actVersion \\
  maxVersions' = maxVersions \\
  maxConflicts' = maxConflicts
\end{schema}

This completes the specification of the data-model of CSS. 

\end{document}
