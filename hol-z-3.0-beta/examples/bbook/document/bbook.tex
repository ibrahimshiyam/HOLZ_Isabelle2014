
\section{The Specification}

This document cantains specification and analysis of Spivey's 
classical BirthdayBook example. It is intended to demonstrate
the use of the ZETA-frontend to write and typecheck
specifications in Z, and the Isabelle backend that allows for
stating and proving proof obligations over this specification.

\subsection{Basic Datatypes and State Schemas}
\zsection{BBSpec}

\begin{zedgroup}
  \begin{zed}
    [NAME,DATE]
  \end{zed}

  \begin{schema}{BirthdayBook}
    known: \power NAME \\
    birthday: NAME \pfun DATE
    \where
    known = \dom birthday
  \end{schema}
\end{zedgroup}

\subsection{Operations}
\begin{zed}
 InitBirthdayBook == [BirthdayBook | known = \emptyset] 
\end{zed}

\begin{zedgroup}
  \begin{schema}{AddBirthday}
    \Delta BirthdayBook \\
    name?: NAME; date?: DATE
    \where
    name? \notin known \\
    birthday' = birthday \cup \{name? \mapsto date?\}
  \end{schema}
  \begin{schema}{FindBirthday}
    \Xi BirthdayBook \\
    name?: NAME; date!: DATE 
    \where
    name? \in known \\
    date! = birthday(name?)
  \end{schema}
  \begin{schema}{Remind}
    \Xi BirthdayBook \\
    today?: DATE; cards!: \power NAME
    \where
    cards! = \{ n: NAME | n \in known; birthday(n) =  today? \}\\
  \end{schema}
\end{zedgroup}


\subsection{Strengthening the Specification}

%\zsection[BBSpec]{BBStrength}

\begin{zed}
  REPORT ::= ok | already\_known | not\_known
\end{zed}

\begin{schema}{Success}
  result!: REPORT \\
  \where
  result! = ok
\end{schema}

\begin{schema}{AlreadyKnown}
  \Xi BirthdayBook \\
  name?: NAME \\
  result!: REPORT \\
  \where
  name? \in known \\
  result! = already\_known \\
\end{schema}

\begin{zed}
  RAddBirthday == ( AddBirthday \land Success ) \lor AlreadyKnown
\end{zed}


\subsection{Implementing the Birthday Book}

%\zsection[BBSpec]{BBImpl}

\begin{schema}{BirthdayBook1}
  names: \nat \fun NAME \\
  dates: \nat \fun DATE \\
  hwm:   \nat \\
  \where
  \forall i,j: 1 \upto hwm @ i \neq j \implies names(i) \neq names(j) \\
\end{schema}

\begin{zed}
 InitBirthdayBook1 == [ BirthdayBook1 | hwm = 0 ] 
\end{zed}

\begin{schema}{Abs}
  BirthdayBook \\
  BirthdayBook1 \\
  \where
  known = \{ i: 1 \upto hwm @ names(i) \} \\
  \forall i: 1 \upto hwm @ birthday( names(i) ) = dates(i)
\end{schema}

\begin{schema}{AddBirthday1}
  \Delta BirthdayBook1 \\
  name?: NAME \\
  date?: DATE \\
  \where
  \forall i: 1 \upto hwm @ name? \neq names(i) \\
  hwm' = hwm + 1 \\
  names' = names \oplus \{ hwm' \mapsto name? \} \\
  dates' = dates \oplus \{ hwm' \mapsto date? \}
\end{schema}


\subsection{Specification}
\subsection{Stating Conjectures}
At times, the designer of a specification might want to state
a certain property that he has in mind when writing the specification
document. Such properties can be stated as \emph{conjecture}. ZETA
can type-check them and export them to HOL-Z; the latter will
consider a conjecture as definition of an internal constant symbol.
For proving the conjecture, one states a lemma in the analysis that
this internal constant symbol is actually equivalent to $True$.

The statement of a conjecture is simply done by:

\begin{zed}
 \forall BirthdayBook; BirthdayBook1; name?:NAME; date?:DATE @  \\
    name? \notin known \land (known = \{i : 1 \upto hwm @ names(i)\}) \\
    \implies (\forall i : 1\upto hwm @ name? \neq names(i))
\end{zed}


\subsection{Conclusion}
In the following, we discuss two versions of analysis: one based on relational refinement,
another on functional refinement, which leads to simpler proof.
See corresponding \verb+BBSpec.thy+ file for the relational refinement,
and \verb+BBSpec_Functional.thy+ for the functional refinement.

The following ISAR command starts a new Isabelle 
theory based on Z, including all libraries and setups. 

%%% Local Variables: 
%%% mode: latex
%%% TeX-master: t
%%% End: 
